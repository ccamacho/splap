\section{Equivalence between the operational and denotational semantics}\label{sec:equivalence}

\begin{itemize}
\item We prove the equivalence of the operational semantics and
  denotational semantics.
\item The proofs are in the appendix
\end{itemize}

\bprop\label{prop:pref}
  Dado el término $P\in\fodaPAp$, y la característica $\feature{A}\in\calF$, entonces
  $$\prodp(\feature{A};P)=\semdenp{\feature{A};\cdot}(\prodp(P))$$
\begin{proof}
  This proposition is an inmmediate consequence of
  Lemma~\ref{lem:pref} in the Appendix~\ref{app:proofs}
\end{proof}
\eprop


\bprop\label{prop:prefopt}
  Dado el término $P\in\fodaPAp$, la característica $\feature{A}\in\calF$ y la probabilidad $q\in (0,1)$, entonces
  $$\prodp(\ofeature{A};_qP)=\semdenp{\ofeature{A};_q\cdot}(\prodp(P))$$
  \bprf
    Esto es consecuencia de la proposición~\ref{prop:prefopt}.
  \eprf
\begin{proof}
  This proposition is an inmmediate consequence of
  Lemma~\ref{lem:prefopt} in the Appendix~\ref{app:proofs}
\end{proof}
\eprop


\bprop\label{prop:choice}
  Dados los términos $P,Q\in\fodaPAp$, y la probabilidad $q\in (0,1)$, entonces
  $$\prodp(P\choice_q Q)=\semdenp{\cdot\choice_q\cdot}\bigl(\prodp(P),\prodp(Q)\bigr)$$
\begin{proof}
  This proposition is an inmmediate consequence of
  Lemma~\ref{lem:choice} in the Appendix~\ref{app:proofs}
\end{proof}
\eprop


\bprop\label{prop:paral}
  Dados los términos $P,Q\in\fodaPAp$, y la probabilidad $q\in (0,1)$, entonces
  $$\prodp(P\paral Q)=\semdenp{\cdot\paral\cdot}\bigl(\prodp(P),\prodp(Q)\bigr)$$
\begin{proof}
  This proposition is an inmmediate consequence of
  Lemma~\ref{lem:paral} in the Appendix~\ref{app:proofs}
\end{proof}
\eprop

\bprop\label{prop:mand}
  Dado el término   $P\in\fodaPAp$  y la característica  $\feature{A}\in\calF$, entonces $(pr,p)\in$ si y sólo si
  $$\prodp(\mandatory{A}{P})=\semdenp{\mandatory{A}{\cdot}}(\prodp(P))$$
\begin{proof}
  This proposition is an inmmediate consequence of
  Lemma~\ref{lem:mand} in the Appendix~\ref{app:proofs}
\end{proof}
\eprop

\bprop\label{cor:forb}
  Dado el término $P\in\fodaPAp$ y la característica $\feature{A}\in\calF$, entonces 
  $$\prodp(\forbid{A}{P})=\semdenp{\forbid{A}{\cdot}}(\prodp(P))$$
\begin{proof}
  This proposition is an inmmediate consequence of
  Lemma~\ref{lem:forb} in the Appendix~\ref{app:proofs}
\end{proof}
\eprop

\bprop\label{prop:req}
  Dado el término $P\in\fodaPAp$ y las características $\feature{A},\feature{B}\in\calF$,
  entonces
  \begin{displaymath}
    \prodp(\require{A}{B}{P})=\semdenp{\require{A}{B}{\cdot}}(\prodp(P))
  \end{displaymath}
\begin{proof}
  This proposition is an inmmediate consequence of
  Lemma~\ref{lem:req} in the Appendix~\ref{app:proofs}
\end{proof}
\eprop

\bprop\label{prop:excl}
  Dado el término $P\in\fodaPAp$ y las características $\feature{A},\feature{B}\in\calF$, entonces
  \begin{displaymath}
    \prodp(\exclude{A}{B}{P})=\semdenp{\exclude{A}{B}{\cdot}}(\prodp(P))
  \end{displaymath}
\begin{proof}
  This proposition is an inmmediate consequence of
  Lemma~\ref{lem:excl} in the Appendix~\ref{app:proofs}
\end{proof}
\eprop

\bthm\label{prop:equivprob}
  Dado el término $P\in\fodaPAp$ y el producto $pr\in\prod(\np(P))$. $(pr,p)\in\semdenp{P}$ si y sólo si
  $(pr,p)\in\prodp(P)$.
  \textit{Demostración.}
    La demostración es trivial por inducción estructural sobre $P$
    al usar los corolarios~\ref{prop:pref}, \ref{prop:prefopt}, \ref{prop:choice},
    \ref{prop:paral},  \ref{prop:mand}, \ref{prop:forb}, \ref{prop:req}, y~\ref{prop:excl}.
\ethm


  
%%% Local Variables: 
%%% mode: latex
%%% TeX-master: "main"
%%% ispell-local-dictionary: "english"
%%% End: 
