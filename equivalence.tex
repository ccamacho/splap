\section{Equivalence between the operational and denotational semantics}\label{sec:equivalenceMain}
We have defined two different semantics for our language:
the products derived from the operational semantics and the products
obtained from the denotational semantics. It is important that
both semantics are consistent, so that we can chose the approach that suits
better in any moment.

\bprop\label{prop:DenoOpe}
%\label{prop:pref}
  Let $P,Q\in\fodaPAp$ be terms, $\feature{A},\feature{B}\in\calF$ be features and $q\in (0,1)$, be a probability. We have the following results:
  $$
  \setcounter{enumi}{0}
  \begin{array}{rlll<{\stepcounter{enumi}(\arabic{enumi})}}
  \prodp(\feature{A};P)&=&\semdenp{\feature{A};\cdot}(\prodp(P)) &\\
  \prodp(\ofeature{A};_qP)&=&\semdenp{\ofeature{A};_q\cdot}(\prodp(P))& \\
  \prodp(P\choice_q Q)&=&\semdenp{\cdot\choice_q\cdot}\bigl(\prodp(P),\prodp(Q)\bigr)& \\
  \prodp(P\paral Q)&=&\semdenp{\cdot\paral\cdot}\bigl(\prodp(P),\prodp(Q)\bigr)& \\
  \prodp(\mandatory{A}{P})&=&\semdenp{\mandatory{A}{\cdot}}(\prodp(P))& \\
  \prodp(\forbid{A}{P})&=&\semdenp{\forbid{A}{\cdot}}(\prodp(P))& \\
  \prodp(\require{A}{B}{P})&=&\semdenp{\require{A}{B}{\cdot}}(\prodp(P))& \\
  \prodp(\exclude{A}{B}{P})&=&\semdenp{\exclude{A}{B}{\cdot}}(\prodp(P)) &
    \end{array}$$
    \begin{proof}
      The full proof of this Proposition is in
      Appendix~\ref{proof:DenoOpe}. Each equality above
      is proved in a different Lemma:
      \begin{enumerate*}[label=(\arabic{enumi})]
      \item is consequence of Lemma~\ref{lem:pref},
      \item is
        consequence of Lemma~\ref{lem:prefopt},
      \item is consequence of
        Lemma~\ref{lem:choice},
      \item is consequence of
        Lemma~\ref{lem:paral},
      \item  is consequence of
        Lemma~\ref{lem:mand},
      \item is consequence of
        Lemma~\ref{lem:excl},
      \item is consequence of
        Lemma~\ref{lem:req}, and
      \item is consequence of
        Lemma~\ref{lem:excl}.
      \end{enumerate*}
    \end{proof}
\eprop
The definition of the operator \(\semdenp{\cdot\paral\cdot}\) is
clearly associative and commutative. Then, as consequence of the
previous proposition, the semantics of the
conjunction operator $\paral$ is associative and commutative.

Finally, we have the previously announced result. The proof, by structural induction on $P$, is easy from
Proposition~\ref{prop:DenoOpe}.

%  \ref{prop:pref}, \ref{prop:prefopt}, \ref{prop:choice},
%    \ref{prop:paral},  \ref{prop:mand}, \ref{prop:forb}, \ref{prop:req}, y~\ref{prop:excl}.

\bthm\label{prop:equivprob}
  Let $P\in\fodaPAp$ be a term, $pr\subseteq\calF$ be a product, and
  $p\in(0,1]$ be a probability. We have that $ (pr,p)\in\semdenp{P}$ if and only if
  $(pr,p)\in\prodp(P)$.
\ethm

\begin{theorem}\mbox{ }
  \begin{itemize}
  \item Let $P,Q\in\fodaPAp$, then $P\equivprob Q$ iff
    $\semden{P}=\semden{Q}$.
  \item $\equivprob$ is a congruence.
  \end{itemize}
  \begin{proof}
    This is a direct consequence of Theorem~\ref{prop:equivprob}
  \end{proof}
\end{theorem}


%%% Local Variables:
%%% mode: latex
%%% TeX-master: "main"
%%% ispell-local-dictionary: "english"
%%% End:
