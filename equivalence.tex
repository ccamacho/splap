\section{Equivalence between the operational and denotational semantics}\label{sec:equivalenceMain}
We have defined two different semantics for our language:
the products derived from the operational semantics and the products
obtained from the denotational semantics. It is important that
both semantics coincide, so that we can chose the approach that suits
better in any moment. The proof of the following result is an immediate consequence of Lemmas~\ref{lem:pref}--\ref{lem:excl} (see Appendix~\ref{app:proofs} of the paper).

\bprop\label{prop:DenoOpe}
%\label{prop:pref}
  Let $P,Q\in\fodaPAp$ be terms, $\feature{A},\feature{B}\in\calF$ be features and $q\in (0,1)$, be a probability. We have the following results:
  $$\begin{array}{lll}
  \prodp(\feature{A};P)&=&\semdenp{\feature{A};\cdot}(\prodp(P))\\
  \prodp(\ofeature{A};_qP)&=&\semdenp{\ofeature{A};_q\cdot}(\prodp(P))\\
  \prodp(P\choice_q Q)&=&\semdenp{\cdot\choice_q\cdot}\bigl(\prodp(P),\prodp(Q)\bigr)\\
  \prodp(P\paral Q)&=&\semdenp{\cdot\paral\cdot}\bigl(\prodp(P),\prodp(Q)\bigr)\\
  \prodp(\mandatory{A}{P})&=&\semdenp{\mandatory{A}{\cdot}}(\prodp(P))\\
  \prodp(\forbid{A}{P})&=&\semdenp{\forbid{A}{\cdot}}(\prodp(P))\\
  \prodp(\require{A}{B}{P})&=&\semdenp{\require{A}{B}{\cdot}}(\prodp(P))\\
  \prodp(\exclude{A}{B}{P})&=&\semdenp{\exclude{A}{B}{\cdot}}(\prodp(P))
\end{array}$$
\eprop

Finally, we have the previously announced result. The proof, by structural induction on $P$, is easy from
Proposition~\ref{prop:DenoOpe}.

%  \ref{prop:pref}, \ref{prop:prefopt}, \ref{prop:choice},
%    \ref{prop:paral},  \ref{prop:mand}, \ref{prop:forb}, \ref{prop:req}, y~\ref{prop:excl}.

\bthm\label{prop:equivprob}
  Let $P\in\fodaPAp$ be a term, $pr\subseteq\calF$ be a product, and
  $p\in(0,1]$ be a probability. We have $ (pr,p)\in\semdenp{P}$ if and only if
  $(pr,p)\in\prodp(P)$.
\ethm



%%% Local Variables:
%%% mode: latex
%%% TeX-master: "main"
%%% ispell-local-dictionary: "english"
%%% End:
