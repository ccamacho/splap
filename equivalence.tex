\section{Equivalence between the operational and denotational semantics}\label{sec:equivalence}
We have defined two different semantics for our language:
the products derived from the operational semantics and the products
obtained from the denotational semantics. It is important that
both semantics coincide, so that we can chose the approach that suits
better in any moment. The proof of the following result is an immediate consequence of Lemmas~\ref{lem:prefopt}--\ref{lem:excl} (see the appendix of the paper).

\bprop\label{prop:DenoOpe}
%\label{prop:pref}
  Let $P\in\fodaPAp$ be a term and $\feature{A}\in\calF$ be a feature, then
  $$\prodp(\feature{A};P)=\semdenp{\feature{A};\cdot}(\prodp(P))$$
%\begin{proof}
%  This proposition is an inmmediate consequence of
%  Lemma~\ref{lem:pref} in the Appendix~\ref{app:proofs}
%\end{proof}
%\eprop
%
%\bprop\label{prop:prefopt}

Let  $P\in\fodaPAp$ be a term, $\feature{A}\in\calF$ be a
  feature, and $q\in (0,1)$, be a a probability; then
  $$\prodp(\ofeature{A};_qP)=\semdenp{\ofeature{A};_q\cdot}(\prodp(P))$$
%\begin{proof}
%  This proposition is an inmmediate consequence of
%  Lemma~\ref{lem:prefopt} in the Appendix~\ref{app:proofs}
%\end{proof}
%\eprop
%
%
%\bprop\label{prop:choice}

Let $P,Q\in\fodaPAp$ be terms and $q\in (0,1)$ be a probability, then
  $$\prodp(P\choice_q Q)=\semdenp{\cdot\choice_q\cdot}\bigl(\prodp(P),\prodp(Q)\bigr)$$
%\begin{proof}
%  This proposition is an inmmediate consequence of
%  Lemma~\ref{lem:choice} in the Appendix~\ref{app:proofs}
%\end{proof}
%\eprop
%
%
%\bprop\label{prop:paral}

Let $P,Q\in\fodaPAp$, then
  $$\prodp(P\paral Q)=\semdenp{\cdot\paral\cdot}\bigl(\prodp(P),\prodp(Q)\bigr)$$
%\begin{proof}
%  This proposition is an inmmediate consequence of
%  Lemma~\ref{lem:paral} in the Appendix~\ref{app:proofs}
%\end{proof}
%\eprop
%
%\bprop\label{prop:mand}

Let $P\in\fodaPAp$  be a term and $\feature{A}\in\calF$ be a feature, then
  $$\prodp(\mandatory{A}{P})=\semdenp{\mandatory{A}{\cdot}}(\prodp(P))$$
%\begin{proof}
%  This proposition is an inmmediate consequence of
%  Lemma~\ref{lem:mand} in the Appendix~\ref{app:proofs}
%\end{proof}
%\eprop
%
%\bprop\label{prop:forb}

Let $P\in\fodaPAp$ be a term and $\feature{A}\in\calF$ be a feature, then
  $$\prodp(\forbid{A}{P})=\semdenp{\forbid{A}{\cdot}}(\prodp(P))$$
%\begin{proof}
%  This proposition is an inmmediate consequence of
%  Lemma~\ref{lem:forb} in the Appendix~\ref{app:proofs}
%\end{proof}
%\eprop
%
%\bprop\label{prop:req}

Let $P\in\fodaPAp$ be a term and $\feature{A},\feature{B}\in\calF$
  be features, then
$$\prodp(\require{A}{B}{P})=\semdenp{\require{A}{B}{\cdot}}(\prodp(P))$$
%\begin{proof}
%  This proposition is an inmmediate consequence of
%  Lemma~\ref{lem:req} in the Appendix~\ref{app:proofs}
%\end{proof}
%\eprop
%
%\bprop\label{prop:excl}

Let $P\in\fodaPAp$ be a term and $\feature{A},\feature{B}\in\calF$
  be features, then
$$    \prodp(\exclude{A}{B}{P})=\semdenp{\exclude{A}{B}{\cdot}}(\prodp(P))$$
%\begin{proof}
%  This proposition is an inmmediate consequence of
%  Lemma~\ref{lem:excl} in the Appendix~\ref{app:proofs}
%\end{proof}
\eprop

The proof of the following result is an immediate consequence, by structural induction on $P$, of
Proposition~\ref{prop:DenoOpe}.

%  \ref{prop:pref}, \ref{prop:prefopt}, \ref{prop:choice},
%    \ref{prop:paral},  \ref{prop:mand}, \ref{prop:forb}, \ref{prop:req}, y~\ref{prop:excl}.

\bthm\label{prop:equivprob}
  Let $P\in\fodaPAp$ be a term, $pr\subseteq\calF$ be a product, and
  $p\in(0,1]$ be a probability. We have $ (pr,p)\in\semdenp{P}$ if and only if
  $(pr,p)\in\prodp(P)$.
\ethm



%%% Local Variables:
%%% mode: latex
%%% TeX-master: "main"
%%% ispell-local-dictionary: "english"
%%% End:
