\subsection{Denotational Semantics}
\label{sec:stat:den}
\mncomment{Comprobad que he interpretado bien lo que ponia en los itemizes originales}

Next we define a denotational semantics for the terms of our language. The main features of the semantic domain are that we consider products (set of features) with probability,
each product appears one and the sum of all the probabilities associated with products belongs to the interval $[0,1]$. First, we precisely define the members of the semantic domain.

%\todo{Explicar el modelo con más detalle,  y el motivo por el que se necesitan multiconjnutos}

\mncomment{Me he bloqueado. Se quiere definir primero el dominio semantico, supongo. Asi que habria que definir $\calM$, creo. Luego, durante el resto de la seccion, en lugar de decir $M \subseteq \calP(\calF)\times [0,1]$ habria que decir $M\in \calM$, no? Supongo esto y continuo. Si he supuesto algo mal, corregidlo, please.}

\bdfn\label{def:den:pr}
We define the semantic domain $\calM$ as the largest set $\calM\subseteq\calP(\calP(\calF)\times
  (0,1]))$ such that if $M\in\calM$ then the following
  conditions hold:
  \begin{itemize}
  \item \mncomment{He quitado esta condicion porque es redundante.}
  %$M\subseteq \calP(\calF)\times (0,1]$
  \item If $(P,q)\in M$ and $(P,r)\in M$ then $q=r$.
  \item $0\leq \sum \lbag q \ | \ \exists P: (P,q)\in M\rbag \leq 1$.
  \end{itemize}
%\edfn

%\begin{itemize}
%\item The multisets in will appear in the operational semantics.
%\item We need to convert these multisets into sets of the model. It is
%  possible if the conditions of Proposition~\ref{prop:pr:accum} hold.
%\end{itemize}

%\bdfn
Let $M \in \cal$. \mncomment{Pone que es un multiset. Si es un objeto semantico, es un set, no??}
We define the operator $\accum$ as follows:
  $$\accum(M) = \left\{(P,p)\ \left| \ p=\sum_{(P,q)\in M}q \wedge p>0\right. \right\}$$
\edfn


\bprop\label{prop:pr:accum}
 Let $M\subseteq \calP(\calF)\times [0,1]$ be a multiset, if $1\geq\sum_{(P,q)\in M} q$
 then $\accum(M)\in\calM$.
\eprop


\begin{itemize}
\item Operators of the operational semantics.
\end{itemize}


\bdfn\label{def:semantic:operators}

  \begin{itemize}\mbox{ }

  \item $\semdenp{\nil}=\emptyset$

  \item $\semdenp{\checkmark}=\{(\emptyset,1)\}$

  \item
    $\semdenp{\feature{A};\cdot}(M)=
      \accum\bigl(\lbag(\{\feature{A}\}\cup P,p)\ |\ (P,p)\in M\rbag\bigr)
      % \accum\bigl(
      %    \{ (\{\feature{A}\} \cup P,p ) |\   (P,p) \in M\}\bigr)
         $

  \item
    $\semdenp{\ofeature{A};_{r}\cdot}(M)=
                \accum\bigl(\lbag(\emptyset , 1-r )\rbag\uplus\lbag (\{\feature{A}\}\cup P, r\cdot p ) \ |\ (P,p) \in M\rbag\bigr)$

  \item
    $\semdenp{\cdot\choice_{r}\cdot}(M_{1},M_{2})=\accum\bigl(\lbag(P,r\cdot p  ) \ |\ (P,p) \in
    M_{1}\rbag \uplus \lbag(Q, (1-r)\cdot q  ) \ |\ (Q,q) \in M_{2}\rbag\bigr) $
  \item
    $
        \semdenp{\cdot\paral\cdot}(M_{1},M_{2})= \accum\Bigl(
                \lbag (P\cup Q, p\cdot q)  |\ (P,p) \in M_{1},\ (Q,q) \in M_{2}\rbag\Bigr)
    $

  \item
    $\semdenp{\require{A}{B}{\cdot}}(M)=\accum\left(\begin{array}{l}
      \lbag\bigl(P,p\bigr)\ |\  (P,p)\in M, \feature{A}\not\in P\rbag\ \uplus\\
      \lbag\bigl(\{\feature{B}\}\cup P ,p\bigr)\ |\ (P,p)\in M, \feature{A}\in P\rbag \\
      \end{array}\right)$

  \item
    $\semdenp{\exclude{A}{B}{\cdot}}(M)=\begin{array}[t]{l}
      \{(P,p)\ |\ (P,p)\in M, \feature{A}\not\in P\}\cup\\
      \{(P,p)\ |\ (P,p)\in M, \feature{B}\not\in P\}
      \end{array}$

    \item $\semdenp{\mandatory{A}{\cdot}}(M) = \semdenp{\feature{A};{\cdot}}(M)$

    \item
      $\semdenp{\forbid{A}{\cdot}}(M) = \{(P,p)\ |\ (P, p) \in M, \feature{A}\not\in P      \}$

  \end{itemize}
\edfn

%\todo{explicar los operadores}





\bprop\label{prp:domain:prob}
  Dados  $M,M_{1}, M_{2}\in\calM$,
  la probabilidad $p\in(0,1]$ y
  las características $\feature{A},\feature{B}\in\calF$, entonces:
  \begin{itemize}
  \item $\semdenp{\feature{A};\cdot}(M)\in\calM$
  \item $\semdenp{\ofeature{A};_{r}\cdot}(M)\in\calM$
  \item $\semdenp{\cdot\choice_{r}\cdot}(M_{1},M_{2})\in\calM$
  \item $\semdenp{\cdot\paral\cdot}(M_{1},M_{2})\in\calM$
  \item $\semdenp{\require{A}{B}{\cdot}}(M)\in\calM$
  \item $\semdenp{\exclude{A}{B}{\cdot}}(M)\in\calM$
  \item $\semdenp{\mandatory{A}{\cdot}}(M)\in\calM$
  \item $\semdenp{\forbid{A}{\cdot}}(M)\in\calM$
  \end{itemize}


  \begin{proof}
    It is easy to check that all multisets appearing in
    Definition~\ref{def:semantic:operators}
    meet the conditions of Proposition~\ref{prp:domain:prob}.
  \end{proof}
\eprop





%%% Local Variables:
%%% mode: latex
%%% TeX-master: "main"
%%% ispell-local-dictionary: "english"
%%% End:
