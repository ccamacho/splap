\subsection{Denotational Semantics}
\label{sec:stat:den}
\mncomment{Comprobad que he interpretado bien lo que ponia en los itemizes originales}

Next we define a denotational semantics for the terms of our language. The main features of the semantic domain are that we consider products (set of features) with probability,
each product appears one and the sum of all the probabilities
associated with products belongs to the interval $(0,1]$. First, we
precisely define the members of the semantic domain. 

%\todo{Explicar el modelo con más detalle,  y el motivo por el que se necesitan multiconjnutos}

\mncomment{Me he bloqueado. Se quiere definir primero el dominio
  semantico, supongo. Asi que habria que definir $\calM$, creo. Luego,
  durante el resto de la seccion, en lugar de decir $M \subseteq
  \calP(\calF)\times [0,1]$ habria que decir $M\in \calM$, no? Supongo
  esto y continuo. Si he supuesto algo mal, corregidlo,
  please.}\lcomen{Ha veces que lo que hay son elementos 
$MM \subseteq \calP(\calF)\times [0,1]$ que no cumplen la condición de \calM.}

\bdfn\label{def:den:pr}
We define the semantic domain $\calM$ as the largest set $\calM\subseteq\calP(\calP(\calF)\times
  (0,1]))$ such that if $M\in\calM$ then the following
  conditions hold:
  \begin{itemize}
  \item If $(P,q)\in M$ and $(P,r)\in M$ then $q=r$.
  \item $0\leq \sum \lbag q \ | \ \exists P: (P,q)\in M\rbag \leq 1$.
  \end{itemize}
%\edfn

%\begin{itemize}
%\item The multisets in will appear in the operational semantics.
%\item We need to convert these multisets into sets of the model. It is
%  possible if the conditions of Proposition~\ref{prop:pr:accum} hold.
%\end{itemize}

%\bdfn
Let $M$ a multiset  with elements in the set $\calP(\calF)\times [0,1]$.
We define the operator $\accum$ as follows:
  $$\accum(M) = \left\{(P,p)\ \left| \ p=\sum_{(P,q)\in M}q \wedge p>0\right. \right\}$$
\edfn

The following result is immediate.
\bprop\label{prop:pr:accum}
 Let $M$ be a multiset with elements in the set $\calP(\calF)\times
 [0,1]$
 \mncomment{otra vez lo de multiset. Puede que yo no haya entendido
   algo? Pero vamos, si es un multiset, no tiene sentido que ponga
   $M\subseteq \calP(\calF) \times [0,1]$.}\lcomen{En la mayoría de
   los operadores semántics salen multiconjuntos. Por eso $M$ debe ser
 un multiconjunto.}
 %be a multiset, 
 If $1\geq\sum\lbag q\ |\ (P,q)\in M\rbag$
 then $\accum(M)\in\calM$.
\eprop


Next we define the operators of the denotational semantics. Multisets
meeting the conditions of the previous Proposition appear when
defining these operators. For instance, the prefix operator
$\semden{\fA;}(M)$ should add feature $\fA$ to any product in $M$. Let
suppose that 
$M=\{
   (\{\fB,\fA\},\frac{1}{2}), 
   (\{\fB\},\frac{1}{2})\}$. 
If we add $\fA$ to the products of $M$, we obtain the product
$\{\fA,\fB\}$ repeated twice with probability $\frac{1}{2}$. So we
need to apply the function $\accum$ to accumulate both probabilities
and obtain a single product with probability $1$. Also in this
definition we use the multiset union denoted by $\uplus$.


\bdfn\label{def:semantic:operators}
  Let $M,M_1,M_2\in\calM$, $\fA,\fB\in\calF$, and $p\in(0,1]$. For any operator appearing in
  Definition~\ref{sec:stat:sintax} we define its denotational operator
  as follows:
  \begin{itemize}

  \item $\semdenp{\nil}=\emptyset$

  \item $\semdenp{\checkmark}=\{(\emptyset,1)\}$

  \item
    $\semdenp{\feature{A};\cdot}(M)=
      \accum\bigl(\lbag(\{\feature{A}\}\cup P,p)\ |\ (P,p)\in M\rbag\bigr)
      % \accum\bigl(
      %    \{ (\{\feature{A}\} \cup P,p ) |\   (P,p) \in M\}\bigr)
         $

  \item
    $\semdenp{\ofeature{A};_{r}\cdot}(M)=
                \accum\bigl(\lbag(\emptyset , 1-r )\rbag\uplus\lbag (\{\feature{A}\}\cup P, r\cdot p ) \ |\ (P,p) \in M\rbag\bigr)$

  \item
    $\semdenp{\cdot\choice_{r}\cdot}(M_{1},M_{2})=\accum\bigl(\lbag(P,r\cdot p  ) \ |\ (P,p) \in
    M_{1}\rbag \uplus \lbag(Q, (1-r)\cdot q  ) \ |\ (Q,q) \in M_{2}\rbag\bigr) $
  \item
    $
        \semdenp{\cdot\paral\cdot}(M_{1},M_{2})= \accum\Bigl(
                \lbag (P\cup Q, p\cdot q)  |\ (P,p) \in M_{1},\ (Q,q) \in M_{2}\rbag\Bigr)
    $

  \item
    $\semdenp{\require{A}{B}{\cdot}}(M)=\accum\left(\begin{array}{l}
      \lbag\bigl(P,p\bigr)\ |\  (P,p)\in M, \feature{A}\not\in P\rbag\ \uplus\\
      \lbag\bigl(\{\feature{B}\}\cup P ,p\bigr)\ |\ (P,p)\in M, \feature{A}\in P\rbag \\
      \end{array}\right)$

  \item
    $\semdenp{\exclude{A}{B}{\cdot}}(M)=\begin{array}[t]{l}
      \{(P,p)\ |\ (P,p)\in M, \feature{A}\not\in P\}\cup\\
      \{(P,p)\ |\ (P,p)\in M, \feature{B}\not\in P\}
      \end{array}$

    \item $\semdenp{\mandatory{A}{\cdot}}(M) = \semdenp{\feature{A};{\cdot}}(M)$

    \item
      $\semdenp{\forbid{A}{\cdot}}(M) = \{(P,p)\ |\ (P, p) \in M, \feature{A}\not\in P      \}$

  \end{itemize}
\edfn

%\todo{explicar los operadores}


It is easy to check that all multisets appearing in the previous
definition meet the conditions of Proposition~\ref{prop:pr:accum}. So
the operators are actually well defined. This is formalized in the
following Proposition.
\bprop\label{prp:domain:prob}
  Let  $M,M_{1}, M_{2}\in\calM$,
  $p\in(0,1]$ a probability, and 
  $\feature{A},\feature{B}\in\calF$, then:
  \begin{itemize}
  \item $\semdenp{\feature{A};\cdot}(M)\in\calM$
  \item $\semdenp{\ofeature{A};_{r}\cdot}(M)\in\calM$
  \item $\semdenp{\cdot\choice_{r}\cdot}(M_{1},M_{2})\in\calM$
  \item $\semdenp{\cdot\paral\cdot}(M_{1},M_{2})\in\calM$
  \item $\semdenp{\require{A}{B}{\cdot}}(M)\in\calM$
  \item $\semdenp{\exclude{A}{B}{\cdot}}(M)\in\calM$
  \item $\semdenp{\mandatory{A}{\cdot}}(M)\in\calM$
  \item $\semdenp{\forbid{A}{\cdot}}(M)\in\calM$
  \end{itemize}
\eprop





%%% Local Variables:
%%% mode: latex
%%% TeX-master: "main"
%%% ispell-local-dictionary: "english"
%%% End:
