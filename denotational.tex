\section{Semántica denotacional}

De manera de definir la semántica denotacional para la extensión
probabilística es indispensable definir el \emph{dominio matemático}
donde serán representados los elementos sintácticos de $\fodaPA$. 

La semántica de cualquier expresión $\fodaPA$
se representa por su conjunto de productos y sus probabilidades,
y cada producto puede ser representado por  sus características.

\bdfn
  Se define el conjunto $M\in\calM$ si y sólo si
  \begin{itemize}
  \item $M\subseteq \calP(\calF)\times (0,1]$
  \item $(P,q),(P,r)\in M$ entonces $q=r$
  \item $1\geq\sum_{(P,q)\in M} q$
  \end{itemize}
\edfn

La notación $\lbag \cdots\rbag$ representa un multiconjunto.

\bdfn
  Dado un multiconjunto $M\subseteq \calP(\calF)\times [0,1]$, se define entonces
  $$\accum(M) = \left\{(P,p)\ | \ p=\sum_{(P,q)\in M}q\ ,\ p>0\right\}$$
\edfn
\bprop
 Dado un multiconjunto $M\subseteq \calP(\calF)\times [0,1]$, si $1\geq\sum_{(P,q)\in M} q$
 entonces $\accum(M)\in\calM$.
\eprop

El siguiente paso es definir un operador semántico para cada uno de los
operadores sintácticos en $\fodaPA$.
Esto es realizado mediante la siguiente definición.
\todo{Most of the sets in the following definition should be multisets}
\bdfn\label{def:semantic:operators}
  Dados los siguientes multiconjuntos $M,M_{1}, M_{2}\in\calM$,
  la secuencia de productos $P \in \calP(\calF)$,
  los valores de probabilidades $p,q,r \in (0,1)$ y
  las características $\feature{A},\feature{B}\in\calF$.

  Son definidos los siguientes operadores:

  \begin{itemize}
  \item $\semdenp{\nil}=\emptyset$
  \item $\semdenp{\checkmark}=\{(\emptyset,1)\}$
  \item 
    $\semdenp{\feature{A};\cdot}:\calM\mapsto\calM$
    como
    $$\semdenp{\feature{A};\cdot}(M)= 
      \accum\bigl(\lbag(\{\feature{A}\}\cup P,p)\ |\ (P,p)\in M\rbag\bigr)
      % \accum\bigl(
      %    \{ (\{\feature{A}\} \cup P,p ) |\   (P,p) \in M\}\bigr)
         $$

  \item 
    $\semdenp{\ofeature{A};_{r}\cdot}:\calM\mapsto\calM$
    como
    $$\semdenp{\ofeature{A};_{r}\cdot}(M)=
                \accum\bigl(\lbag(\emptyset , 1-r )\rbag\cup\lbag (\{\feature{A}\}\cup P, r\cdot p ) \ |\ (P,p) \in M\rbag\bigr)$$

  \item 
    $\semdenp{\cdot\choice_{r}\cdot}:\calM\times\calM\mapsto\calM$
    como 
    $$\semdenp{\cdot\choice_{r}\cdot}(M_{1},M_{2})=\accum\bigl(\lbag(P,r\cdot p  ) \ |\ (P,p) \in
    M_{1}\rbag \cup \lbag(Q, (1-r)\cdot q  ) \ |\ (Q,q) \in M_{2}\rbag\bigr) $$
  \item 
    $\semdenp{\cdot\paral\cdot}:\calM\times\calM\mapsto\calM$
    como 
    $$
        \semdenp{\cdot\paral\cdot}(M_{1},M_{2})= \accum\Bigl(
                \lbag (P\cup Q, p\cdot q)  |\ (P,p) \in M_{1},\ (Q,q) \in M_{2}\rbag\Bigr)
    $$
    
  \item  
    $\semdenp{\require{A}{B}{\cdot}}:\calM\mapsto\calM$
    como 
    $$\semdenp{\require{A}{B}{\cdot}}(M)=\accum\left(\begin{array}{l}
      \lbag\bigl(P,p\bigr)\ |\  (P,p)\in M, \feature{A}\not\in P\rbag\ \cup\\
      \lbag\bigl(\{\feature{B}\}\cup P ,p\bigr)\ |\ (P,p)\in M, \feature{A}\in P\rbag \\ 
      \end{array}\right)$$
  \item  
    $\semdenp{\exclude{A}{B}{\cdot}}:\calM\mapsto\calM$
    como 
    $$\semdenp{\exclude{A}{B}{\cdot}}(M)=\begin{array}[t]{l}
      \{(P,p)\ |\ (P,p)\in M, \feature{A}\not\in P\}\cup\\
      \{(P,p)\ |\ (P,p)\in M, \feature{B}\not\in P\}
      \end{array}$$
%
%
    \item $\semdenp{\mandatory{A}{\cdot}}:\calM\mapsto\calM$
    como 
    $$\semdenp{\mandatory{A}{\cdot}}(M) = \semdenp{\feature{A};{\cdot}}(M)
    $$
%
%
    \item   $\semdenp{\forbid{A}{\cdot}}:\calM\mapsto\calM$
    como 
    $$\semdenp{\forbid{A}{\cdot}}(M) = \{(P,p)\ |\ (P, p) \in M, \feature{A}\not\in P      \}$$
  \end{itemize}
\edfn

\bprop
  Dados los multiconjuntos $M,M_{1}, M_{2}\in\calM$,
  la probabilidad $p\in(0,1]$ y
  las características $\feature{A},\feature{B}\in\calF$, entonces:
  \begin{itemize}
  \item $\semdenp{\feature{A};\cdot}(M)\in\calM$
  \item $\semdenp{\ofeature{A};_{r}\cdot}(M)\in\calM$
  \item $\semdenp{\cdot\choice_{r}\cdot}(M_{1},M_{2})\in\calM$
  \item $\semdenp{\cdot\paral\cdot}(M_{1},M_{2})\in\calM$
  \item $\semdenp{\require{A}{B}{\cdot}}(M)\in\calM$
  \item $\semdenp{\exclude{A}{B}{\cdot}}(M)\in\calM$
  \item $\semdenp{\mandatory{A}{\cdot}}(M)\in\calM$
  \item $\semdenp{\forbid{A}{\cdot}}(M)\in\calM$
  \end{itemize}
\eprop

Una vez definidos los operadores semánticos sobre un conjunto de productos,
es posible definir la semántica denotacional para cualquier
expresión $\fodaPA$.
Esta es definida de manera inductiva.
\bdfn
  La semántica denotacional de $\fodaPA$ es representada por la función
  $\semdenp{\cdot}:\fodaPA\rightarrow\calP(\calP(\calF))$ definida de manera inductiva
  de la siguiente forma: para cualquier operador $n$-ario $\op\in\{\nil, \checkmark ,
  \feature{A};\cdot, \ofeature{A};_{p}\cdot , \cdot\choice_{p}\cdot,
  \cdot\paral\cdot, \require{A}{B}{\cdot},\exclude{A}{B}{\cdot},
  \mandatory{A}{\cdot}, \forbid{A}{\cdot}\}$%
\footnote{$\nil$ y $\checkmark$ son operadores 0-arios;
  $\feature{A};\cdot$, $\ofeature{A};\cdot$
  $\require{A}{B}{\cdot}$, $\exclude{A}{B}{\cdot}$,
  $\mandatory{A}{\cdot}$, $\forbid{A}{\cdot}$ son operadores 1-arios;
  $\cdot\choice_p\cdot$ y $\cdot\paral\cdot$ son operadores 2-arios.}:
  $$\semdenp{\op(P_1,\ldots P_n)}=\semdenp{\op}(\semdenp{P_1},\ldots,\semdenp{P_n})$$
\edfn



En la figura \ref{example:denot:1} y en la figura \ref{example:denot:2} se describe el cálculo
la semántica denotacional para las \SPLs\ presentadas en la
figura \ref{example:op}.


%%% Local Variables: 
%%% mode: latex
%%% TeX-master: "main"
%%% End: 
