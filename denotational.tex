\section{Semántica denotacional}
\label{sec:stat:den}

\begin{itemize}
\item The model: products (set of features) with probability.
\end{itemize}


%\todo{Explicar el modelo con más detalle,  y el motivo por el que se necesitan multiconjnutos}

\bdfn\label{def:den:pr}
  Let us consider the set $\calM\subseteq\calP(\calP(\calF)\times
  (0,1]))$ defined as follows: $M\in\calM$ if the following
  conditions hold:
  \begin{itemize}
  \item $M\subseteq \calP(\calF)\times (0,1]$ \\
  \item $(P,q),(P,r)\in M$ entonces $q=r$ \\
  \item $1\geq\sum_{(P,q)\in M} q$ \\
  \end{itemize}
\edfn

\begin{itemize}
\item The multisets in will appear in the operational semantics.
\item We need to convert these multisets into sets of the model.
\end{itemize}

\bdfn
  Let $M \subseteq \calP(\calF)\times [0,1]$ a multi set then we
  define the operator $\accum$ as follows:
  $$\accum(M) = \left\{(P,p)\ | \ p=\sum_{(P,q)\in M}q\ ,\ p>0\right\}$$

\edfn
\bprop\label{prop:pr:accum}
 Let $M\subseteq \calP(\calF)\times [0,1]$ be a multiset, if $1\geq\sum_{(P,q)\in M} q$
 then $\accum(M)\in\calM$.
\eprop


% \bdfn\label{def:pr:sum}
% Dados los multiconjuntos:
% \todo{Verificar agregada por carlos}

% $M = \lbag (M_{1},m_{1}),\ldots,(M_{n},m_{n})\rbag, 0 \leq m_{1}\ldots m_{n} \leq 1, M_{1}\ldots M_{n} \subseteq \calP(\calF) $

% $N = \lbag (N_{1},n_{1}),\ldots,(N_{n},n_{n})\rbag, 0 \leq n_{1}\ldots n_{n} \leq 1, N_{1}\ldots N_{n} \subseteq \calP(\calF) $
 
% se define $M \uplus N$ como:

% $\lbag (M_{1},m_{1}),\ldots,(M_{n},m_{n}),(N_{1},n_{1}),\ldots,(N_{n},n_{n})\rbag$
% \edfn

% \bprop\label{prop:pr:sum}
% \todo{Verificar agregada por carlos}
% Dados los multiconjuntos $M\in\calM$ y $N\in\calM$ entonces $M \uplus N \in\calM$
% \eprop



\bdfn\label{def:semantic:operators}

  \begin{itemize}
        
  \item $\semdenp{\nil}=\emptyset$
  
  \item $\semdenp{\checkmark}=\{(\emptyset,1)\}$
   
  \item 
    $\semdenp{\feature{A};\cdot}:\calM\mapsto\calM$
    como
    $$\semdenp{\feature{A};\cdot}(M)= 
      \accum\bigl(\lbag(\{\feature{A}\}\cup P,p)\ |\ (P,p)\in M\rbag\bigr)
      % \accum\bigl(
      %    \{ (\{\feature{A}\} \cup P,p ) |\   (P,p) \in M\}\bigr)
         $$
  
  Para el procesamiento de cualquier característica obligatoria
  será procesada con probabilidad $p$ y esta será añadida
  al multiconjunto de características procesadas previamente.
  

  \item 
    $\semdenp{\ofeature{A};_{r}\cdot}:\calM\mapsto\calM$
    como
    $$\semdenp{\ofeature{A};_{r}\cdot}(M)=
                \accum\bigl(\lbag(\emptyset , 1-r )\rbag\uplus\lbag (\{\feature{A}\}\cup P, r\cdot p ) \ |\ (P,p) \in M\rbag\bigr)$$

  Al procesar una característica opcional deben realizarse
  dos acciones. La primera es de no procesar la característica,
  terminando el procesamiento del término con probabilidad $1-r$.
  La segunda es de procesarla con probabilidad $r$ siendo esta
  multiplicada por la probabilidad acumulada
  del término restante. El resultado de ambas acciones es
  unido usando el operador $\uplus$ y las probabilidades 
  de cada conjunto sumadas utilizando la función $\accum(M)$.
  
  \item 
    $\semdenp{\cdot\choice_{r}\cdot}:\calM\times\calM\mapsto\calM$
    como 
    $$\semdenp{\cdot\choice_{r}\cdot}(M_{1},M_{2})=\accum\bigl(\lbag(P,r\cdot p  ) \ |\ (P,p) \in
    M_{1}\rbag \uplus \lbag(Q, (1-r)\cdot q  ) \ |\ (Q,q) \in M_{2}\rbag\bigr) $$
\todo{Deberiamos intercambiar de lado la probabilidad r y 1-r por consistencia en este operador.}
  Se crea un nuevo multiconjunto al procesar los términos (multiconjuntos) a cada
  lado del operador de selección única, este, generará dos conjuntos dentro del
  nuevo multiconjunto y sus probabilidades serán sumadas utilizando la función $\accum(M)$.

  \item 
    $\semdenp{\cdot\paral\cdot}:\calM\times\calM\mapsto\calM$
    como 
    $$
        \semdenp{\cdot\paral\cdot}(M_{1},M_{2})= \accum\Bigl(
                \lbag (P\cup Q, p\cdot q)  |\ (P,p) \in M_{1},\ (Q,q) \in M_{2}\rbag\Bigr)
    $$
    
  El operador de paralelo dentro de la semántica operacional unirá las características
  de ambos términos multiplicando su probabilidad.
     
  \item  
    $\semdenp{\require{A}{B}{\cdot}}:\calM\mapsto\calM$
    como 
    $$\semdenp{\require{A}{B}{\cdot}}(M)=\accum\left(\begin{array}{l}
      \lbag\bigl(P,p\bigr)\ |\  (P,p)\in M, \feature{A}\not\in P\rbag\ \uplus\\
      \lbag\bigl(\{\feature{B}\}\cup P ,p\bigr)\ |\ (P,p)\in M, \feature{A}\in P\rbag \\ 
      \end{array}\right)$$

  La restricción de requerimiento no modifica las probabilidades de los productos en
  construcción, sin embargo, si se cumple la restricción de requerimiento debe agregarse
  la característica requerida en el producto y deben ser sumadas las probabilidades utilizando
  los operadores $\uplus$ y $\accum(M)$.
      
  \item  
    $\semdenp{\exclude{A}{B}{\cdot}}:\calM\mapsto\calM$
    como 
    $$\semdenp{\exclude{A}{B}{\cdot}}(M)=\begin{array}[t]{l}
      \{(P,p)\ |\ (P,p)\in M, \feature{A}\not\in P\}\cup\\
      \{(P,p)\ |\ (P,p)\in M, \feature{B}\not\in P\}
      \end{array}$$

  Para el caso de la restricción de exclusión, si esta se cumple, se
  procede a eliminar las características que estén referenciadas por
  este operador y las probabilidades se mantienen inalteradas.
  
    \item $\semdenp{\mandatory{A}{\cdot}}:\calM\mapsto\calM$
    como 
    $$\semdenp{\mandatory{A}{\cdot}}(M) = \semdenp{\feature{A};{\cdot}}(M)
    $$

  El operador de inclusión se comporta de igual manera al operador de
  obligatoriedad.
  
    \item   $\semdenp{\forbid{A}{\cdot}}:\calM\mapsto\calM$
    como 
    $$\semdenp{\forbid{A}{\cdot}}(M) = \{(P,p)\ |\ (P, p) \in M, \feature{A}\not\in P      \}$$

  El operador de ocultamiento tiene el mismo comportamiento que el operador de
  exclusión, con la diferencia que se aplica a una sola característica.

  \end{itemize}
\edfn

%\todo{explicar los operadores}





\bprop\label{prp:domain:prob}
  Dados  $M,M_{1}, M_{2}\in\calM$,
  la probabilidad $p\in(0,1]$ y
  las características $\feature{A},\feature{B}\in\calF$, entonces:
  \begin{itemize}
  \item $\semdenp{\feature{A};\cdot}(M)\in\calM$
  \item $\semdenp{\ofeature{A};_{r}\cdot}(M)\in\calM$
  \item $\semdenp{\cdot\choice_{r}\cdot}(M_{1},M_{2})\in\calM$
  \item $\semdenp{\cdot\paral\cdot}(M_{1},M_{2})\in\calM$
  \item $\semdenp{\require{A}{B}{\cdot}}(M)\in\calM$
  \item $\semdenp{\exclude{A}{B}{\cdot}}(M)\in\calM$
  \item $\semdenp{\mandatory{A}{\cdot}}(M)\in\calM$
  \item $\semdenp{\forbid{A}{\cdot}}(M)\in\calM$
  \end{itemize}


\textit{Demostración.}
La demostración de la proposición \ref{prp:domain:prob}
permite comprobar que las operaciones sobre los multiconjuntos
dentro de la semántica denotacional de \fodaPAp
siempre tienen como resultado un único multiconjunto.
%
Para aquellos operadores que hagan uso de la función $\accum(M)$
por la proposición \ref{prop:pr:accum} se cumple que $\accum(M)\in\calM$.
De igual manera para los operadores que utilicen el 
operador $\uplus$ por la proposición \ref{prop:pr:sum} se cumple que 
$M \uplus N \in\calM$.
\eprop

Una vez definidos los operadores semánticos sobre un conjunto de productos,
es posible definir la semántica denotacional para cualquier
expresión $\fodaPA$.
Esta es definida de manera inductiva.
\bdfn
  La semántica denotacional de $\fodaPA$ es representada por la función
  $\semdenp{\cdot}:\fodaPA\rightarrow\calP(\calP(\calF))$ definida de manera inductiva
  de la siguiente forma: para cualquier operador $n$-ario $\op\in\{\nil, \checkmark ,
  \feature{A};\cdot, \ofeature{A};_{p}\cdot , \cdot\choice_{p}\cdot,
  \cdot\paral\cdot, \require{A}{B}{\cdot},\exclude{A}{B}{\cdot},
  \mandatory{A}{\cdot}, \forbid{A}{\cdot}\}$%
\footnote{$\nil$ y $\checkmark$ son operadores 0-arios;
  $\feature{A};\cdot$, $\ofeature{A};\cdot$
  $\require{A}{B}{\cdot}$, $\exclude{A}{B}{\cdot}$,
  $\mandatory{A}{\cdot}$, $\forbid{A}{\cdot}$ son operadores 1-arios;
  $\cdot\choice_p\cdot$ y $\cdot\paral\cdot$ son operadores 2-arios.}:
  $$\semdenp{\op(P_1,\ldots P_n)}=\semdenp{\op}(\semdenp{P_1},\ldots,\semdenp{P_n})$$
\edfn



En la figura \ref{example:denot:1} y en
la figura \ref{example:denot:2} se describe
el cálculo la semántica denotacional de 5 ejemplos,
así permitir analizar el proceso de cálculo de 
las probabilidades de las características dentro de
términos a medida que son procesados, estos,
viendo que el término es representado como un árbol,
desde los elementos terminales u hojas hasta la raíz. 




\begin{figure}[h]
        \hrule
        
        \vspace*{1em}
        
        \centering $P_{1} = \feature{A};\ofeature{B};_{p}\checkmark$\review
        
        \scalebox{0.8}{$
                \begin{array}{lll}
                \semdenp{\checkmark} &=& \{(\emptyset,1)\}\\
                \semdenp{\ofeature{B};_p\checkmark} &=&   \semdenp{\ofeature{B};_{p}\cdot}(\semdenp{\checkmark}) = 
                \semdenp{\ofeature{B};_{p}\cdot}(\{(\emptyset,1)\})=\\
                &&\{(\emptyset,(1-p)),(\feature{B},p)\}\\
                \semdenp{\feature{A};\ofeature{B};_p\checkmark} &=& \semdenp{\feature{A};\cdot}(\semdenp{\ofeature{B};_p\checkmark})=
                \semdenp{\feature{A};\cdot}(\{(\emptyset,(1-p)),(\feature{B},p)\})=\\
                &&\{(\feature{A},(1-p)),(\feature{A}\feature{B},p)\}
                \end{array}
                $}
        
        \vspace*{1em} \hrule\vspace*{1em}
        
        
        \centering 
        $P_{2} = \feature{A};(\ofeature{B};_{p}\checkmark \choice_{q} \ofeature{B};_{r}\checkmark)$ \review
        
        %\feature{A};(\ofeature{B};_{p}\checkmark \choice_{q} \ofeature{B};_{r}\checkmark)
        \scalebox{0.8}{$
                \begin{array}{lll}
                \semdenp{\checkmark} &=& \{(\emptyset,1)\}\\
                \semdenp{\ofeature{B};_{\frac{1}{3}}\checkmark} &=&   \semdenp{\ofeature{B};_{\frac{1}{3}}\cdot}(\semdenp{\checkmark}) = 
                \semdenp{\ofeature{B};_{\frac{1}{3}}\cdot}(\{(\emptyset,1)\})=\\
                &&\{(\emptyset,\frac{2}{3}),(\feature{B},\frac{1}{3})\}\\
                \semdenp{\ofeature{B};_{\frac{1}{3}}\checkmark \choice_{\frac{1}{2}} \ofeature{B};_{\frac{1}{3}}\checkmark} &=&
                \semdenp{\cdot  \choice_{\frac{1}{2}} \cdot}(\semdenp{\ofeature{B};_{\frac{1}{3}}\checkmark},\semdenp{\ofeature{B};_{\frac{1}{3}}\checkmark)}\\
                &=&
                \semdenp{\cdot  \choice_{\frac{1}{2}} \cdot}(\{(\emptyset,\frac{2}{3}),(\feature{B},\frac{1}{3})\}, \{(\emptyset,\frac{2}{3}),(\feature{B},\frac{1}{3})\}) =\\
                &=&     
                \accum\Bigl(\bigl\{(\emptyset,\frac{2}{6}),(\feature{B},\frac{1}{6})\bigr\}  \uplus \bigl\{(\emptyset,\frac{2}{6}),(\feature{B},\frac{1}{6})\bigr\}\Bigr)=\\
                &=&\bigl\{(\emptyset,\frac{2}{3}),(\feature{B},\frac{1}{3})\bigr\}
                \end{array}
                $}
        
        \vspace*{1em} \hrule\vspace*{1em}
        
        $P_{3} = \feature{A};(\feature{A};\checkmark\choice_{q} \feature{B};\checkmark)$\review
        
        %$\feature{A};(\feature{A};\checkmark\choice_{q} \feature{B};\checkmark)$
        \scalebox{0.8}{$
                \begin{array}{lll}
                \semdenp{\checkmark} &=& \{(\emptyset,1)\}\\
                \semdenp{\feature{A};\checkmark} &=& \semdenp{\feature{A};\cdot}(\semdenp{\checkmark})=
                \semdenp{\feature{A};\cdot}(\{(\emptyset,1)\})=\{(\feature{A},1)\}\\
                \semdenp{\feature{A};\checkmark\choice_{q} \feature{B};\checkmark} &=&
                \semdenp{\cdot \choice_{q} \cdot} (\semdenp{\feature{A};\checkmark},\semdenp{\feature{B};\checkmark})=\\
                &=& \semdenp{\cdot \choice_{q} \cdot} (\{(\feature{A},1)\},\{(\feature{B},1)\}) = \{(\feature{A},q),(\feature{B},(1-q))\}\\
                \semdenp{\feature{A};(\feature{A};\checkmark\choice_{q} \feature{B};\checkmark)} &=&
                \semdenp{\feature{A};\cdot}(\semdenp{\feature{A};\checkmark\choice_{q} \feature{B};\checkmark})=\\
                &=& \semdenp{\feature{A};\cdot}(\{(\feature{A},q),(\feature{B},(1-q))\}) = \{(\feature{A},q),(\feature{A}\feature{B},(1-q))\}
                \end{array}
                $}
        \vspace*{1em}
        
        \hrule
        
        \caption{Ejemplo de la ejecución de las reglas de la semántica denotacional (1/2).\label{example:denot:1}}
\end{figure}


\begin{figure}[h]
        \hrule
        \vspace*{1em}
        
        \centering $P_{4} = (\feature{A};\checkmark \choice_{p} \feature{B};\nil)$\review
        
        %(\feature{A};\checkmark \choice_{p} \feature{B};\nil)
        \scalebox{0.8}{$
                \begin{array}{lll}
                \semdenp{\nil} &=& \emptyset\\
                \semdenp{\feature{B};\nil} &=& \semdenp{\feature{B};\cdot}(\semdenp{\nil}) =\semdenp{\feature{B};\cdot}(\emptyset)= \emptyset\\
                \semdenp{\checkmark} &=& \{(\emptyset,1)\}\\
                \semdenp{\feature{A};\checkmark} &=& \semdenp{\feature{A};\cdot}(\semdenp{\checkmark})=
                \semdenp{\feature{A};\cdot}(\{(\emptyset,1)\})=\{(\feature{A},1)\}\\
                \semdenp{\feature{A};\checkmark \choice_{\frac{1}{7}} \feature{B};\nil} &=& 
                \semdenp{\cdot \choice_{\frac{1}{7}} \cdot}(\semdenp{\feature{A};\checkmark},\semdenp{\feature{B};\nil})\\
                &=&\semdenp{\cdot \choice_{\frac{1}{7}} \cdot}(\{(\feature{A},1)\},\emptyset)= \{(\feature{A},\frac{1}{7})\}
                \end{array}
                $}
        
        \vspace*{1em} \hrule\vspace*{1em}
        \centering $P_{5} = (\feature{A};\checkmark \paral \feature{B};\checkmark)$\review
        
        %$(\feature{A};\checkmark \paral \feature{B};\checkmark)$
        \scalebox{0.8}{$
                \begin{array}{lll}
                \semdenp{\checkmark} &=& \{(\emptyset,1)\}\\
                \semdenp{\feature{A};\checkmark} &=& \semdenp{\feature{A};\cdot}(\semdenp{\checkmark})=
                \semdenp{\feature{A};\cdot}(\{(\emptyset,1)\})=\{(\feature{A},1)\}\\
                \semdenp{\feature{A};\checkmark \paral \feature{B};\checkmark} &=& 
                \semdenp{\cdot \paral \cdot}(\semdenp{\feature{A};\checkmark},\semdenp{\feature{B};\checkmark})=
                \semdenp{\cdot \paral \cdot}(\{(\feature{A},1)\},\{(\feature{B},1)\})=\\
                && \{(\feature{A}\feature{B},\frac{1}{2}),(\feature{B}\feature{A},\frac{1}{2})\}\\
                \end{array}
                $}
        \vspace*{1em} \hrule
        
        
        \caption{Ejemplo de la ejecución de las reglas de la semántica denotacional (2/2).\label{example:denot:2}}
\end{figure}





%%% Local Variables: 
%%% mode: latex
%%% TeX-master: "main"
%%% ispell-local-dictionary: "english"
%%% End: 
