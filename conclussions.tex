\section{Conclusions}
\label{section:jstat:concs}

We have presented a probabilistic extension of our
formal framework to specify and analyze \SPLs. The main goal of this
proposal is to alleviate the combinatorial explosion issue, where a vast
number of combinations are generated by some of the algebra operators,
that making unpractical to process the entire \SPL. By including probabilistic information in our process algebra, we are able to generate significant information for determining the probability
of a given feature to be present in a valid product. We have provided two semantic frameworks for our language and have proved that they identify the same processes.
%
In order to show the applicability of our approach, a tool containing the
implementation of the denotational semantics for our probabilistic extension
has been developed. This tool has been used to conduct an experimental study.
%
The results of this study show that, using our approach, it is possible to compute
the probability of each feature in the \SPL\ to be present in a valid product.
Thus, the testing process can focus on those features having a high probability
of being included in a product.
%%Moreover, the implementation of our proposed algorithm
%%shows that the obtained performance is linear with respect to the total
%%length of the \SPL.
%\ccomen{Es cierto no lo hemos ni estudiado ni analizado.}
%\lcomen{¿qué significa eso?, no hemos demostrado
%  nada de complejidad.}\mncomment{Si que se habla de complejidad pero yo no tengo claro que lo que se dice este bien.}
% \acomen{Como en la implementación, directamente quitaba la frase}



%
%To do so we have defined new syntactical operators for the probabilistic
%extension. Once defined, we have represented the extension to give a logical
%meaning to these semantic rules based on the operational semantics
%and denotational semantics.
%
%Also this extension was proven to be consistent against the presented definitions
%and with the non probabilistic models.
%
%The implementation of this extension allows to compute and calculate the probability of the features among the valid products.
%
We have two main lines for future work. First, it is important to develop
mechanisms allowing us to simplify and/or optimize terms
based on the results of the probabilistic analysis. In addition,
we plan to find practical use cases to show the usefulness of having a probabilistic extension for \SPLs.



%%% Local Variables:
%%% mode: latex
%%% TeX-master: "main"
%%% End:
