\section{Conclusiones}
\label{section:jstat:concs}

Una vez descritos los problemas asociados a la explosión combinatoria de
algunos operadores del álgebra, se busca generar información suficiente
que permita determinar la probabilidad de que una característica cualquiera
se encuentre presente en el conjunto de productos válidos del término
a representar.
%
Para llevar a cabo esto, han sido descritos los nuevos operadores sintácticos
de la extensión probabilística. Una vez definidos, han de representarse las 
extensiones para las reglas de las semánticas descritas en secciones anteriores,
como lo son la semántica operacional y la semántica denotacional.
%
Al haber mostrado por completo la definición formal de la extensión probabilística
es imprescindible mostrar que la extensión es consistente con las definiciones de las
secciones anteriores.
%
Una vez definida la extensión, son mostradas las ventajas de ocultar aquellos elementos
sintácticos que no afectan la cardinalidad del conjunto de productos válidos del sistema.
Esto permite analizar el término de manera mas eficiente reduciendo la complejidad del
algoritmo a orden N, posiblemente siendo este término de una longitud considerable.
