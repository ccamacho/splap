\section{Conclusions}
\label{section:jstat:concs}

In this paper we have presented a probabilistic extension of our 
formal framework to specify and analyze \SPLs. The main goal of this 
proposal is to alleviate the combinatory explosion issue, where a vast
number of combinations are generated by some of the algebra operators,
it being unpractical to completely processing the entire \SPL. Hence, we
are able to generate relevant information for determining the probability 
of a given feature to be present in a valid product.

In order to show the applicability of our approach, a tool containing the 
implementation of the denotational semantics for the probabilistic extension
has been developed. This tool has been used to conduct a experimental study.
%
The results of this study show that, using our approach, it is possible to calculate 
the probability of each feature in the \SPL\ to be present in a valid product. 
Thus, the testing process can be focused in those features having a high probability 
of being included in a product. Moreover, the implementation of our proposed algorithm
show that the obtained performance is s linear with respect to the total length of the \SPL.
%
%To do so we have defined new syntactical operators for the probabilistic
%extension. Once defined, we have represented the extension to give a logical
%meaning to these semantic rules based on the operational semantics
%and denotational semantics.
%
%Also this extension was proven to be consistent against the presented definitions
%and with the non probabilistic models.
%
%The implementation of this extension allows to compute and calculate the probability of the features among the valid products.
%
As future work we plan to extend our approach for...\acomen{Ni idea de qué poner aquí :(}

%%% Local Variables: 
%%% mode: latex
%%% TeX-master: "main"
%%% End: 
