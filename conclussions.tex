\chapter{Conclusiones}
\label{chapter:conclusiones}

\begin{flushright}
\itshape{Where you start is not as important as where you finish.} \\

\texttt{Zig Ziglar}
\end{flushright}
\minitoc
%\newpage

A lo largo de los últimos 30 años han sido propuestos innumerables modelos para representar \SPLs. 
Sin embargo, el análisis formal y automático de los modelos de variabilidad sigue siendo un desafío, donde dependiendo de la información 
que sea necesario generar, los investigadores se encuentran con problemas combinatorios complejos de resolver en \emph{tiempo y espacio}.
Este trabajo de investigación propone y muestra un formalismo matemático capaz de definir modelos de variabilidad lo suficientemente
flexibles como para poder ser aplicados a otros modelos, ya que su aplicación no se basa exclusivamente en las reglas de \FODA. Además, se muestran distintas extensiones e implementaciones que permitirán generar información útil para la toma de decisiones.
 





%%% Local Variables: 
%%% mode: latex
%%% TeX-master: "../main"
%%% End: 
