% $Id: introduction.tex,v 1.8 2013/12/03 09:17:27 ccamacho Exp $
\section{Introduction}
\label{ref:introduction}


The main purpose of Software Product Lines (in short, \SPLs) is to produce products
while increasing productivity and shortening the time-to-market period. \SPLs\ depend
on which software products are being produced and which of them are better for
a specific criterion. When products are represented in a product line organization,
several modeling approaches can be used to increase both quality and productivity.
In most cases this is represented in the form of features, relationships and
constraints. For instance, some of these approaches are FODA~\cite{kchnp90},
RSEB~\cite{mj98} and PLUSS~\cite{k05,ebb06}.

Automatic and formal approaches arise from these graphical representations to model
variability and commonality of systems. Formal models allow for detecting errors in the early
stages of the production process. Some of the existing approaches use algebras and
semantics~\cite{szw05,kkm06,prb11,acl13}, while others use propositional or first order
logic~\cite{man02,ka07,abgf10,atfg10,nnz14}.

\begin{figure}[t]

\linefigure

\centering


\begin{tikzpicture}[>=stealth',shorten >=1pt,auto,node distance=2cm, semithick]

  \node[rectangle,draw] (A)   {\feature{A}};
  \node[rectangle,draw] (B) [below of=A]   {\feature{B}};

  \node[rectangle,draw] (C) [right of=A]   {\feature{A}};
  \node[rectangle,draw] (D) [below of=C]   {\feature{B}};

%  \node[] (E) [right of=C]   {};
  \node[rectangle,draw,node distance=4cm] (G) [right of=C]   {\feature{A}};
  \node[rectangle,draw] (F) [below left of=G]   {\feature{B}};
%  \node[] (H) [below of=G]   {};
  \node[rectangle,draw] (I) [below right  of=G]   {\feature{C}};



  \path (A) edge [-o,shorten >=-0.05em] node[draw=none] {Optional} (B.north)
        (C) edge [-*,shorten >=-0.05em] node[draw=none] {Mandatory} (D.north)

        (G) edge [->]  node[draw=none, shift={(80:-0.5)}]{Choose 1} (F)
            edge [->]  node[draw=none] {} (I)  ;


\draw [<-] (6.8,-0.2) arc (-36:-140:20pt);

  \node[rectangle,draw,node distance=2.5cm] (J) [right of=G]   {\feature{A}};
  \node[rectangle,draw] (K) [below of=J]   {\feature{B}};

  \node[rectangle,draw] (L) [right of=J]   {\feature{A}};
  \node[rectangle,draw] (M) [below of=L]   {\feature{B}};


  \path
        (J) edge [->,dashed] node[draw=none] {Excludes} (K)
        (L) edge [->,dotted] node[draw=none] {Implies} (M)  ;

\end{tikzpicture}

\linefigure

\caption{\FODA\ Diagram representation.\label{fig:foda:relations}}

\end{figure}


% \begin{figure}[t]
% \linefigure
% \centering
% \begin{minipage}{0.2\hsize}
%         \centering
%         \EX{a}


%         \centering
%         \begin{tikzpicture}[>=stealth',shorten >=1pt,auto,node distance=1.5cm, semithick]

%           \node[rectangle,draw] (A)   {\feature{A}};
%           \node[rectangle,draw,node distance=1.1cm] (B) [below of=A]   {\feature{B}};

%           \path (A) edge [-o] node[draw=none] {} (B.north)
%           ;
%         \end{tikzpicture}

% \end{minipage}
% %
% \begin{minipage}{0.1\hsize}

%         \centering

%         \EX{b}

%         \begin{tikzpicture}[>=stealth',shorten >=1pt,auto,node distance=1.5cm, semithick]


%           \node[rectangle,draw] (C) [right of=A]   {\feature{A}};
%           \node[rectangle,draw,node distance=1.1cm] (D) [below of=C]   {\feature{B}};
%           \path (C) edge [] node[draw=none] {} (D)
%           ;
%         \fill (D.north) circle (0.1);
%         \end{tikzpicture}

% \end{minipage}
% %
% \begin{minipage}{0.33\hsize}
% \centering

%         \EX{c}

%         \begin{tikzpicture}[>=stealth',shorten >=1pt,auto,node distance=1.5cm, semithick]


%           \node[rectangle,draw] (G)    {\feature{A}};
%           \node[rectangle,draw] (F) [below left of=G]   {\feature{B}};
%           \node[rectangle,draw] (I) [below right  of=G]   {\feature{C}};

%           \path
%                 (G) edge [->]  node[draw=none] {} (F)
%                     edge [->]  node[draw=none] {} (I)

%           ;

%         \draw [<-] (0.7,-0.1) arc (360:180:14pt);

%         \end{tikzpicture}
% \end{minipage}
% \begin{minipage}{0.33\hsize}
% \centering

%         \EX{d}


%         \begin{tikzpicture}[>=stealth',shorten >=1pt,auto,node distance=1.5cm, semithick]
%           \node[rectangle,draw] (G)   {\feature{A}};

%           \node[rectangle,draw] (F) [below left of=G]   {\feature{B}};
%           \node[rectangle,draw] (I) [below right  of=G]   {\feature{C}};
%           \path
%                 (G) edge []  node[draw=none] {} (F.north)
%                     edge []  node[draw=none] {} (I.north)
%           ;
%         \fill (F.north) circle (0.1);
%         \fill (I.north) circle (0.1);
%         \end{tikzpicture}
% \end{minipage}


% \linefigure



% \begin{minipage}{0.33\hsize}
% \centering
%         \EX{e}
% \vspace*{0.3em}

%         \begin{tikzpicture}[>=stealth',shorten >=1pt,auto,node distance=1.5cm, semithick]
%           \node[rectangle,draw] (G)    {\feature{A}};
%           \node[rectangle,draw] (F) [below left of=G]   {\feature{B}};
%           \node[rectangle,draw] (I) [below right  of=G]   {\feature{C}};
%           \path
%                 (G) edge [-o]  node[draw=none] {} (F.north)
%                     edge []  node[draw=none] {} (I.north) ;
%         \fill (I.north) circle (0.1);
%         \end{tikzpicture}
% \end{minipage}
% \begin{minipage}{0.3\hsize}
%         \centering
%  \EX{f}

%         \begin{tikzpicture}[>=stealth',shorten >=1pt,auto,node distance=1.5cm, semithick]


%           \node[rectangle,draw] (A)    {\feature{A}};
%           \node[rectangle,draw] (B) [below left   of=A]   {\feature{B}};
%           \node[rectangle,draw] (C) [below right  of=A]   {\feature{C}};

%           \path
%                 (A) edge [-o]  node[draw=none] {} (B.north)
%                         edge [-o]  node[draw=none] {} (C.north)

%     (B) edge [->,dashed] node[draw=none] {} (C);
%         \end{tikzpicture}
% \end{minipage}
% \begin{minipage}{0.3\hsize}
%         \centering
%  \EX{g}

%         \begin{tikzpicture}[>=stealth',shorten >=1pt,auto,node distance=1.5cm, semithick]


%           \node[rectangle,draw] (A)    {\feature{A}};
%           \node[rectangle,draw] (B) [below left   of=A]   {\feature{B}};
%           \node[rectangle,draw] (C) [below right  of=A]   {\feature{C}};

%           \path
%                 (A) edge [-o]  node[draw=none] {} (B.north)
%                         edge [-o]  node[draw=none] {} (C.north)

%         (B) edge [->,dotted] node[draw=none] {} (C)
% ;
%         \end{tikzpicture}
% \end{minipage}
% \linefigure

% \caption{Examples of \FODA\ Diagrams.\label{section:introduction:figure:examples}}
% \end{figure}

Feature Oriented Domain Analysis~\cite{kchnp90} (in short, \FODA) is a graphical
representation of commonality and variability of systems. Figure~\ref{fig:foda:relations}
shows all \FODA\ relationships and constraints.
% and Figure~\ref{section:introduction:figure:examples}
% shows some examples of how
% \FODA\ diagrams are built.
In order to perform automatic analysis,  graphical representations
must be transformed into mathematical entities~\cite{nak10}.
Thus, it is necessary to provide the original
\FODA\ graphical representation with formal semantics base, where automated analysis can be
performed~\cite{bhst04}. This  issue is solved by using \fodaPA~\cite{acl13}, a formal
framework to represent \FODA\ diagrams using process algebras. \fodaPA\
can only be applied not only to
\FODA, but also to represent other feature-related
problems and variability models.

Costs within our formal framework refer to the required effort to add a feature to a product
under construction. This cost refers to many factors depending on the context of the
product line organization. For example, the cost of adding a feature to a product can be
equal to the number of lines of code of a software component~\cite{n07,babc09}, or the
effort, in terms of human hours, to develop that module. This effort is usually measured by using
functional metrics~\cite{j04,cg08,hko13}. The cost of adding third-party modules, both
commercial and open source, to our \SPL\ could be the time associated with integrating it into
the product line organization.

The order in which features are computed is important in many software projects and
has an  important relevance to the final cost of the project. This order can be easily incorporated
into the operational semantics of~\fodaPA. In this paper we represent costs with
natural numbers. This is not a drawback of our formalism because we can assume a
minimum cost unit, and therefore, any cost can be represented as a multiple of this unit.

Formal methods and process algebras 
have demonstrated being a useful tool
to describe systems behavior.
They have evolved to extend their relations
to support probabilistic models~\cite{pp01,l19911}.

Figures like the displayed trees
describes how are assigned probabilities inside the
system behavior.
Is relevant to notice that the probability
of execute a transition \textit{t} 
is not necessarily equal to the other probabilities 
from the same level.

\begin{figure}[h]
	\hspace{50px}
	\begin{minipage}{0.4\hsize}       
		
		
		\begin{minipage}{0.2\hsize}
			\scalebox{0.6}{
				\begin{tikzpicture}[>=stealth',shorten >=1pt,auto,node distance=2cm, semithick]
				
				\node[circle,draw] (A)   {};
				\node[circle,draw] (B) [below left of=A]   {};
				\node[regular polygon,regular polygon sides=3,draw,scale=0.8] (F) [below of=B]   {T};
				\node[circle,draw] (C) [below right of=A]   {};
				\node[circle,draw] (D) [below left of=C]   {};
				\node[circle,draw] (E) [below right of=C]   {};
				\node[regular polygon,regular polygon sides=3,draw,scale=0.8] (G) [below  of=D]   {T};
				\node[regular polygon,regular polygon sides=3,draw,scale=0.8] (H) [below  of=E]   {T};			
				
				\path (A) edge [] node [pos=0.5, above left, draw=white, opacity=0, text opacity=1]{1/3} (B);
				\path (B) edge [] node [pos=0.5, left, draw=white, opacity=0, text opacity=1]{a} (F);
				\path (A) edge [] node [pos=0.5, above right, draw=white, opacity=0, text opacity=1]{1/2} (C);
				\path (C) edge [] node [pos=0.5, above left, draw=white, opacity=0, text opacity=1]{1/2} (D);
				\path (C) edge [] node [pos=0.5, above right, draw=white, opacity=0, text opacity=1]{1/2} (E);
				\path (D) edge [] node [pos=0.5, right, draw=white, opacity=0, text opacity=1]{b} (G);
				\path (E) edge [] node [pos=0.5, left, draw=white, opacity=0, text opacity=1]{c} (H);
				\end{tikzpicture}
			}
		\end{minipage}
	\end{minipage}
	\begin{minipage}{0.4\hsize}       
		
		
		
		
		\begin{minipage}{0.2\hsize}
			\scalebox{0.6}{
				\begin{tikzpicture}[>=stealth',shorten >=1pt,auto,node distance=4cm, semithick]
				\node[circle,draw] (A)   {};
				\node[regular polygon,regular polygon sides=3,draw,scale=0.8] (B) [below left of=A]   {T};
				\node[regular polygon,regular polygon sides=3,draw,scale=0.8] (C) [below  of=A]   {T};			
				\node[regular polygon,regular polygon sides=3,draw,scale=0.8] (D) [below right of=A]   {T};			
				
				\path (A) edge [] node [pos=0.5, above left, draw=white, opacity=0, text opacity=1]{1/3} (B);
				\path (A) edge [] node [pos=0.5, left, draw=white, opacity=0, text opacity=1]{1/3} (C);
				\path (A) edge [] node [pos=0.5, above right, draw=white, opacity=0, text opacity=1]{1/3} (D);
				
				\end{tikzpicture}}
		\end{minipage}
		
	\end{minipage}
	
\end{figure}

These models allow to assign a real value
between 0 and 1 to each transition in order to execute
an event.

Several research projects have proven that
statistic analysis over Software Product
Lines allow to determine several SPL characteristics
like the certainty of finding valid products among
complex models ~\cite{tllv15,tlll15,chssgl13,dpcslsh17}.
Some of them~\cite{tllv15,tlll15} describe models
to run statistical analysis to represent Software
Product Lines by defining an specific set of operational
rules to compute their defined syntax elements.
These models demonstrate their ability by being
integrated to standard tools like QFLan, Microsoft's
SMT Z3 and MultiVeStA. Others like ~\cite{chssgl13}
focuses on testing SPLs properties like reliability,
this research article defines three verification
techniques like a probabilistic model checker on
each product, on a model range or testing the behavior
relations with other models. Other articles focus on
describing use cases~\cite{dpcslsh17} analyzing the
probability of finding features inside valid products.
An interesting feature is that any of the referenced
research articles describe their approaches using multisets.




%%% Local Variables:
%%% mode: latex
%%% TeX-master: "main"
%%%   ispell-local-dictionary: "american"
%%% End:



% LocalWords:  formalisms
