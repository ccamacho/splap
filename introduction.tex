El objetivo principal de éste capítulo es el de
estudiar el comportamiento de las líneas de productos
software basado en el análisis probabilístico.
Este análisis muestra cómo identificar aquella
característica mas utilizada dentro del modelo 
calculando la probabilidad de encontrar
una característica dentro de una línea de productos.
%
La sección~\ref{sec:stat:sintax} muestra la descripción
formal de la extensión de \fodaPA\ para soportar
probabilidades en los elementos sintácticos que lo
requieran.
%
Luego, la sección~\ref{sec:stat:oper} muestra los
cambios necesarios en las reglas semánticas de \fodaPA\
para soportar los cambios realizados en la sintaxis
del lenguaje.
%
Una vez modificadas tanto la sintaxis como la semántica 
de \fodaPA\ la sección~\ref{sec:stat:consis} demuestra
que todos los teoremas y lemas planteados para esta extensión
son correctos.
%
La sección~\ref{sec:stat:den} describe los cambios realizados a las
reglas de la semántica denotacional.
%
Seguidamente la sección~\ref{sec:stat:hid} muestra una
característica novedosa de la extensión probabilística, que
es la de ocultar aquellas características que no afecten 
a la cardinalidad total de los productos validos a generar.
%
La sección~\ref{sec:equivalence} muestra que los productos
generados por ambas semánticas (operacional y denotacional)
son equivalentes.
%
Para finalizar la sección~\ref{sec:stat:impl} muestra la
implementación de la semántica denotacional de la extensión
probabilística.

%\todo{Creo que es mejor lo que que se cuenta en cada sección del
%  capítulo como en el cap'itulo 2}


%%% Local Variables: 
%%% mode: latex
%%% TeX-master: "../../main"
%%% End: 
