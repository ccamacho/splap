% $Id: introduction.tex,v 1.8 2013/12/03 09:17:27 ccamacho Exp $
\section{Introduction}
\label{ref:introduction}


The main purpose of Software Product Lines (in short, \SPLs) is to produce products 
while increasing productivity and shortening the time-to-market period. \SPLs\ depend 
on which software products are being produced and which of them are better for 
a specific criterion. When products are represented in a product line organization, 
several modeling approaches can be used to increase both quality and productivity.
In most cases this is represented in the form of features, relationships and 
constraints. For instance, some of these approaches are FODA~\cite{kchnp90}, 
RSEB~\cite{mj98} and PLUSS~\cite{k05, ebb06}.

Automatic and formal approaches arise from these graphical representations to model 
variability and commonality of systems. Formal models allow for detecting errors in the early
stages of the production process. Some of the existing approaches use algebras and 
semantics~\cite{szw05, kkm06, prb11,acl13}, while others use propositional or first order 
logic~\cite{man02,ka07, abgf10, atfg10,nnz14}.


\begin{figure}[t]

\linefigure

\centering 


\begin{tikzpicture}[>=stealth',shorten >=1pt,auto,node distance=2cm, semithick]

  \node[rectangle,draw] (A)   {\feature{A}};
  \node[rectangle,draw] (B) [below of=A]   {\feature{B}};

  \node[rectangle,draw] (C) [right of=A]   {\feature{A}};
  \node[rectangle,draw] (D) [below of=C]   {\feature{B}};

%  \node[] (E) [right of=C]   {};
  \node[rectangle,draw,node distance=4cm] (G) [right of=C]   {\feature{A}};
  \node[rectangle,draw] (F) [below left of=G]   {\feature{B}};
%  \node[] (H) [below of=G]   {};
  \node[rectangle,draw] (I) [below right  of=G]   {\feature{C}};



  \path (A) edge [-o,shorten >=-0.05em] node[draw=none] {Optional} (B.north)
        (C) edge [-*,shorten >=-0.05em] node[draw=none] {Mandatory} (D.north)

        (G) edge [->]  node[draw=none, shift={(80:-0.5)}]{Choose 1} (F)
            edge [->]  node[draw=none] {} (I)  ;


\draw [<-] (6.8,-0.2) arc (-36:-140:20pt);   

  \node[rectangle,draw,node distance=2.5cm] (J) [right of=G]   {\feature{A}};
  \node[rectangle,draw] (K) [below of=J]   {\feature{B}};

  \node[rectangle,draw] (L) [right of=J]   {\feature{A}};
  \node[rectangle,draw] (M) [below of=L]   {\feature{B}};


  \path 
        (J) edge [->,dashed] node[draw=none] {Excludes} (K)
        (L) edge [->,dotted] node[draw=none] {Implies} (M)  ;

\end{tikzpicture}

\linefigure

\caption{\FODA\ Diagram representation.\label{fig:foda:relations}}

\end{figure}


% \begin{figure}[t]
% \linefigure
% \centering
% \begin{minipage}{0.2\hsize}
%         \centering
%         \EX{a}


%         \centering
%         \begin{tikzpicture}[>=stealth',shorten >=1pt,auto,node distance=1.5cm, semithick]

%           \node[rectangle,draw] (A)   {\feature{A}};
%           \node[rectangle,draw,node distance=1.1cm] (B) [below of=A]   {\feature{B}};

%           \path (A) edge [-o] node[draw=none] {} (B.north)
%           ;
%         \end{tikzpicture}

% \end{minipage}
% %
% \begin{minipage}{0.1\hsize}

%         \centering 

%         \EX{b}

%         \begin{tikzpicture}[>=stealth',shorten >=1pt,auto,node distance=1.5cm, semithick]


%           \node[rectangle,draw] (C) [right of=A]   {\feature{A}};
%           \node[rectangle,draw,node distance=1.1cm] (D) [below of=C]   {\feature{B}};
%           \path (C) edge [] node[draw=none] {} (D)
%           ;
%         \fill (D.north) circle (0.1);
%         \end{tikzpicture}

% \end{minipage}
% %
% \begin{minipage}{0.33\hsize}
% \centering

%         \EX{c}

%         \begin{tikzpicture}[>=stealth',shorten >=1pt,auto,node distance=1.5cm, semithick]


%           \node[rectangle,draw] (G)    {\feature{A}};
%           \node[rectangle,draw] (F) [below left of=G]   {\feature{B}};
%           \node[rectangle,draw] (I) [below right  of=G]   {\feature{C}};

%           \path
%                 (G) edge [->]  node[draw=none] {} (F)
%                     edge [->]  node[draw=none] {} (I)

%           ;

%         \draw [<-] (0.7,-0.1) arc (360:180:14pt);

%         \end{tikzpicture}
% \end{minipage}
% \begin{minipage}{0.33\hsize}
% \centering

%         \EX{d}


%         \begin{tikzpicture}[>=stealth',shorten >=1pt,auto,node distance=1.5cm, semithick]
%           \node[rectangle,draw] (G)   {\feature{A}};

%           \node[rectangle,draw] (F) [below left of=G]   {\feature{B}};
%           \node[rectangle,draw] (I) [below right  of=G]   {\feature{C}};
%           \path
%                 (G) edge []  node[draw=none] {} (F.north)
%                     edge []  node[draw=none] {} (I.north)
%           ;
%         \fill (F.north) circle (0.1);
%         \fill (I.north) circle (0.1);
%         \end{tikzpicture}
% \end{minipage}


% \linefigure



% \begin{minipage}{0.33\hsize}     
% \centering
%         \EX{e}
% \vspace*{0.3em}

%         \begin{tikzpicture}[>=stealth',shorten >=1pt,auto,node distance=1.5cm, semithick]
%           \node[rectangle,draw] (G)    {\feature{A}};
%           \node[rectangle,draw] (F) [below left of=G]   {\feature{B}};
%           \node[rectangle,draw] (I) [below right  of=G]   {\feature{C}};
%           \path
%                 (G) edge [-o]  node[draw=none] {} (F.north)
%                     edge []  node[draw=none] {} (I.north) ;
%         \fill (I.north) circle (0.1);
%         \end{tikzpicture}
% \end{minipage}
% \begin{minipage}{0.3\hsize}
%         \centering
%  \EX{f}

%         \begin{tikzpicture}[>=stealth',shorten >=1pt,auto,node distance=1.5cm, semithick]


%           \node[rectangle,draw] (A)    {\feature{A}};
%           \node[rectangle,draw] (B) [below left   of=A]   {\feature{B}};
%           \node[rectangle,draw] (C) [below right  of=A]   {\feature{C}};

%           \path
%                 (A) edge [-o]  node[draw=none] {} (B.north)
%                         edge [-o]  node[draw=none] {} (C.north)

%     (B) edge [->,dashed] node[draw=none] {} (C);
%         \end{tikzpicture}
% \end{minipage}
% \begin{minipage}{0.3\hsize}
%         \centering
%  \EX{g}

%         \begin{tikzpicture}[>=stealth',shorten >=1pt,auto,node distance=1.5cm, semithick]


%           \node[rectangle,draw] (A)    {\feature{A}};
%           \node[rectangle,draw] (B) [below left   of=A]   {\feature{B}};
%           \node[rectangle,draw] (C) [below right  of=A]   {\feature{C}};

%           \path
%                 (A) edge [-o]  node[draw=none] {} (B.north)
%                         edge [-o]  node[draw=none] {} (C.north)

%         (B) edge [->,dotted] node[draw=none] {} (C)
% ;
%         \end{tikzpicture}
% \end{minipage}
% \linefigure

% \caption{Examples of \FODA\ Diagrams.\label{section:introduction:figure:examples}}
% \end{figure}

Feature Oriented Domain Analysis~\cite{kchnp90} (in short, \FODA) is a graphical 
representation of commonality and variability of systems. Figure~\ref{fig:foda:relations} 
shows all \FODA\ relationships and constraints.
% and Figure~\ref{section:introduction:figure:examples} 
% shows some examples of how 
% \FODA\ diagrams are built.
In order to perform automatic analysis,  graphical representations 
must be transformed into mathematical entities~\cite{nak10}.
Thus, it is necessary to provide the original
\FODA\ graphical representation with formal semantics base, where automated analysis can be 
performed~\cite{bhst04}. This  issue is solved by using \fodaPA~\cite{acl13}, a formal 
framework to represent \FODA\ diagrams using process algebras. \fodaPA\
can only be applied not only to 
\FODA, but also to represent other feature-related 
problems and variability models.
 
Costs within our formal framework refer to the required effort to add a feature to a product 
under construction. This cost refers to many factors depending on the context of the 
product line organization. For example, the cost of adding a feature to a product can be 
equal to the number of lines of code of a software component~\cite{n07, babc09}, or the 
effort, in terms of human hours, to develop that module. This effort is usually measured by using 
functional metrics~\cite{j04,cg08,hko13}. The cost of adding third-party modules, both 
commercial and open source, to our \SPL\ could be the time associated with integrating it into 
the product line organization. 

The order in which features are computed is important in many software projects and 
has an  important relevance to the final cost of the project. This order can be easily incorporated 
into the operational semantics of~\fodaPA. In this paper we represent costs with 
natural numbers. This is not a drawback of our formalism because we can assume a 
minimum cost unit, and therefore, any cost can be represented as a multiple of this unit. 


\begin{itemize}
\item Hablar de las probabilidades en métodos formales.
\item Carlos debe buscar alguna referencia de SPL con probabilidades
\item Las probabilidades ayudan en los SPL. Por ejemplo en testing
  se pueden comprobar con mayor fiabilidad los productos o componentes
  que aparecen con mayor probabilidad.
\end{itemize}

El objetivo principal de éste capítulo es el de
estudiar el comportamiento de las lineas de productos
software basado en el análisis probabilístico.
Este análisis muestra cómo identificar aquella
característica mas utilizada dentro del modelo 
calculando la probabilidad de encontrar
una característica dentro de una línea de productos software.

Los objetivos del presente capítulo se encuentran
clasificados de acuerdo a la siguiente estructura.
La definición de una nueva sintaxis para soportar
probabilidades basado en la sintaxis de \fodaPA.
Una nueva definición de la semántica operacional
y de la semántica denotacional que permitan
soportar probabilidades basadas en \fodaPA.
Describir las relaciones de equivalencia entre
la semántica operacional y la semántica denotacional.
Mostrar la implementación de la semántica
denotacional del modelo probabilístico determinando
la probabilidad de una característica en un modelo
de características,
sin calcular todos los productos.
Para finalizar el capítulo, se muestran los resultados
de la implementación en comparación
con los resultados obtenidos previamente en
los modelos no probabilísticos \cite{acl13, clc16}.



%%% Local Variables: 
%%% mode: latex
%%% TeX-master: "main"
%%%   ispell-local-dictionary: "american"
%%% End: 

  

% LocalWords:  formalisms
