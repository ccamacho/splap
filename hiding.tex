\section{Ocultando conjuntos de características}

El cálculo de la probabilidad de una característica dentro
de una línea de productos software, permite determinar
cuan utilizado será este componente dentro de la familia
de productos. Esta probabilidad se ve afectada por
aquellos operadores sintácticos que influyan en la 
cardinalidad del número de productos válidos a generar,
es decir, en los operadores de selección única y en los
operadores de selección opcional ($P \choice_{p} Q$ y $\ofeature{A};_{p} P$ respectivamente)
así como sus dependencias y restricciones.

Esta condición es de suma importancia para la optimización de
los términos del álgebra, ya que de esa manera 
todo aquel elemento sintáctico que no influya en la cardinalidad
del conjunto de productos válidos puede ser eliminada del modelo.

%Why we do this?
%\begin{itemize}
%\item The probability of a feature
%\item Consequences of make visible only one feature (plus its
%  dependencies and restrictions).
%\end{itemize}

\begin{definition}
  Dado el conjunto de características $\calA\subseteq \calF$ y un término $P\in\fodaPAp$,
  el operador $P\hide{\calA}$
  representa el ocultamiento sintáctico de las características en el conjunto $\calA$
  para el término $P$.
\end{definition}

De esta manera se necesita una nueva característica $\bot\not\in\calF$. $\calFbot =
\calF\cup\{\bot\}$. $\calA\subseteq\calF$.

\subsection{Semántica operacional}

A continuación se describen las nuevas reglas de la semántica operacional
para el procesamiento de la característica de ocultamiento.

\begin{displaymath}
  \begin{array}{ccccc}
    \nombreRegla{hid1} & 
    \frac{P\tran{\fA}_p P', \fA\in\calA}{P\hide{\calA}\tran{\bot}_p P'\hide{\calA}} &
    \qquad \qquad \qquad&
    \nombreRegla{hid2} &     
    \frac{P\tran{\fA}_p P', \fA\not\in\calA}{P\hide{\calA}\tran{\fA}_p P'\hide{\calA}}
  \end{array}
\end{displaymath}

\subsection{Semántica denotacional}

\begin{definition}
  Dado el producto $pr\subseteq\calF$ y el conjunto de características ocultas $\calA\subseteq\calF$
  se define:
  
  \begin{displaymath}
    pr\hide{\calA} = \{\fA\ |\ \fA\in pr,\
    \fA\not\in\calA\}\cup
    \begin{cases}
      \{\bot\} & \mbox{si } pr\cap\calA\neq\emptyset\\
      \emptyset & \mbox{si } pr\cap\calA=\emptyset\\
    \end{cases}
  \end{displaymath}
  Dada la traza $s\in\calF^{*}$, $s\hide\calA$ representará la substitución
  de cada ocurrencia en la traza $s$ de cualquier característica en $\calA$ por $\bot$.
\end{definition}

\begin{definition}
  Dado el multiconjunto $M\in\calM$ y el conjunto de características ocultas $\calA\subseteq\calF$. Se define:
  \begin{displaymath}
    \semdenp{\cdot\hide{\calA}} = \accum\Bigl(\lbag(pr\hide{\calA},p)\
    |\ (pr,p)\in M\rbag \Bigr)
  \end{displaymath}
\end{definition}


\begin{proposition}
  $P\hide\calA\vtran{s}_r$ si y sólo si $r=\sum\lbag p \ |\ P\vtran{s'}_p Q,\
  s=s'\hide\calA\rbag$
  \begin{proof}
    La demostración es realizada por inducción sobre la longitud
    de la traza $s$. Si la longitud es cero el resultado es trivial. Entonces
    se supone que
    $s=\fA\cdot s_1$. Si $\fA=bot$ entonces cualquier transición
    $P\hideA\tran{s}{p}Q\hideA$ puede ser dividida en transiciones,
    posiblemente mas de una, como por ejemplo.
    \begin{displaymath}
      P\hideA\tran{\bot}_{r_1}P_1\hideA\vtran{s_1}_{r2}Q
    \end{displaymath}
    Entonces se tiene
    \begin{displaymath}
      \begin{split}
        r = \sum\lbag p\ |\ P\hideA\vtran{s}_p Q\rbag = 
        \sum\lbag r_1\cdot r2\ |\
        P\hideA\tran{\bot}_{r_1}P_1\hideA\vtran{s_1}_{r2}Q\rbag =\\
        \sum\lbag r_1'\cdot r2\ |\
        P\hideA\tran{\fB}_{r_1'}P_1'\hideA\vtran{s_1}_{r2}Q,\ \fB\in\calA\rbag
      \end{split}
    \end{displaymath}
    Ahora por cada una de las $r_1'$, se puede aplicar el método inductivo a cada una de las
    transiciones $P_1'\hideA\vtran{s_1}_{r2}Q$ para obtener 
    $r_2=\sum\lbag r2'\ |\ P_1=\vtran{s_1'}Q,\
    s_1=s_1'\hideA\rbag$. Continuando la ecuación anterior :
    \begin{displaymath}
      \begin{split}
        \sum\lbag r_1'\cdot r2\ |\
        P\hideA\tran{\fB}_{r_1'}P_1'\hideA\vtran{s_1}_{r2}Q,\
        \fB\in\calA\rbag=\\
        \sum\lbag r_1'\cdot r2'\ |\ P\hideA\tran{\fB}_{r_1'}P_1'\vtran{s_1'}_{r2'}Q,\
        \fB\in\calA.\ s_1=s_1'\hideA\rbag = \\
        \sum\lbag r_1\cdot r2'\ |\ P\hideA\tran{\bot}_{r_1}P_1\vtran{s_1'}_{r2'}Q,\
        \fB\in\calA.\ s_1=s_1'\hideA\rbag =\\
        \sum\lbag r\ |\ P\vtran{s'}_{r}Q,\ s=s'\hideA\rbag
      \end{split}
    \end{displaymath}
    El caso cuando $\fA\not\in\calA$ es similar al anterior, solo necesita saltar el paso de $\fB$ a $\bot$.
  \end{proof}
\end{proposition}

\begin{proposition}
  \begin{displaymath}
    \prodp(P\hideA)  = \semdenp{\hideA}(\prodp(P))
  \end{displaymath}
  \begin{proof}
    $(pr,p)\in \prodp(P\hideA)$ si y sólo si 
    \begin{displaymath}
      \begin{split}
        p = \sum\lbag r\ |\
          P\hideA\tran{s\checkmark}_rP'\hideA.\ pr=\product{s}\rbag =\\
        \sum\lbag r\ |\ P\tran{s'\checkmark}_rP',\ s=s'\hideA,\
          pr=\product{s}\rbag=\\
        \sum\lbag r\ |\ P\tran{s'\checkmark}_rP',\ s=pr\hideA\rbag =
        \\
        \sum\lbag r\ |\ (pr',r)\in\prodp(P),\ pr'=pr\hideA\rbag =
      \end{split}
    \end{displaymath}
    De tal manera 
    $(pr,p)\in\prodp(P\hideA)$ si y sólo si 
    $(pr,p)\in\semdenp{\hideA}(\prodp(P))$ 
  \end{proof}
\end{proposition}
\begin{proposition}
  Dados los términos $P,Q\in\fodaPAp$ y el conjunto de acciones ocultas $\calA\subseteq\calF$, entonces
  las siguientes proposiciones mantienen
  \begin{itemize}
  \item $\semdenp{\checkmark\hide\calA}=\semdenp{\checkmark}$
  \item $\semdenp{\nil\hide\calA}=\semdenp{\nil}$
  \item
      $\semdenp{(\fA;P)\hide\calA}=
      \begin{cases}
        \semdenp{\bot;(P\hide\calA)} & \mbox{si } A\in\calA\\
        \semdenp{\fA;(P\hide\calA)} & \mbox{si } A\not\in\calA\\
      \end{cases}$
  \item
      $\semdenp{(\ofA;_rP)\hide\calA}=
      \begin{cases}
        \semdenp{\overline{\bot};_r(P\hide\calA)} & \mbox{si } A\in\calA\\
        \semdenp{\ofA;_r(P\hide\calA)} & \mbox{si } A\not\in\calA\\
      \end{cases}$
  \item $\semdenp{(P\choice_P Q)\hide\calA}=\semdenp{(P\hide\calA)\choice_P (Q\hide\calA)}$
  \item $\semdenp{(P\paral Q)\hide\calA}=\semdenp{(P\hide\calA)\paral (Q\hide\calA)}$
  \item If $\fA,\fB\not\in\calA$ entonces 
    $\semdenp{(\require{A}{B}{P})\hide\calA}=\semdenp{\require{A}{B}{(P\hide\calA)}}$
  \item If $\fA,\fB\not\in\calA$ entonces 
    $\semdenp{(\exclude{B}{P})\hide\calA}=\semdenp{\exclude{A}{B}{(P\hide\calA)}}$
  \end{itemize}
  \begin{proof}
    Consequencia del teormea~\ref{prop:equivprob} y la
    proposición anterior.
  \end{proof}
\end{proposition}
%%% Local Variables: 
%%% mode: latex
%%% TeX-master: "main"
%%% End: 
