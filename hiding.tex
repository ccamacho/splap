\section{Hiding sets of features}
\label{sec:stat:hid}

Computing the probability of a single features in a software product line
allows is a measure of the occurrences of this features in the set of
products. For instance, in case of testing it is interesting to know
the most frequent components to focus the testing in these components. 
This probability is affected by the  dependencies and restrictions of
the feature. 


In order to compute the probability of a set of features we are going
to use a hiding operator: we are going to hide other features. The
reason of doing this is that it maybe not feasible to compute all the
products of the software product line. But we expect that we can
achieve this if we restrict ourselves to a subset of features. So the
non interesting features are considered a new feature that we will
denote as $\bot\not\in\calF$, we consider the set $\calFbot =
\calF\cup\{\bot\}$.

\begin{figure}
  \centering
\begin{displaymath}
    \begin{array}{ccccc}
      \nombreRegla{hid1} & 
      \frac{P\tran{\fA}_p P', \fA\in\calA}{P\hide{\calA}\tran{\bot}_p P'\hide{\calA}} &
      \qquad \qquad \qquad&
      \nombreRegla{hid2} &     
        \frac{P\tran{\fA}_p P', \fA\not\in\calA}{P\hide{\calA}\tran{\fA}_p P'\hide{\calA}}
    \end{array}
  \end{displaymath}
  
  \caption{Operational semanitcs for the hiding operator}
  \label{fig:oper-hid}
\end{figure}



So we extend the set of operators with a new one: hiding a set of
features in a term. 

\bdfn
  Let $\calA\subseteq \calF$ be a subset of features and
  $P\in\fodaPAp$ be a term,
  the $P\hide{\calA}$
  hiding operator of the features in $\calA$
  for the term $P$.
\edfn

Next we need to give the semantics of the new operator. First, the
operational semantics is given by the rules appearing in
Figure~\ref{fig:oper-hid}. 

In order to define the denotational semantics of the new operator,
first we need an auxiliary function that hides all some features
from a given product.

\bdfn
  Let $pr\subseteq\calF$ be a product and $\calA\subseteq\calF$
  be a set of features. The \emph{hidden operation of the set $\calA$
    in $pr$}, denoted by $pr\hide{\calA}$, is defined as follows:
  \begin{displaymath}
    pr\hide{\calA} = \{\fA\ |\ \fA\in pr,\
    \fA\not\in\calA\}\cup
    \begin{cases}
      \{\bot\} & \mbox{si } pr\cap\calA\neq\emptyset\\
      \emptyset & \mbox{si } pr\cap\calA=\emptyset\\
    \end{cases}
  \end{displaymath}
  Analogously, for any trace $s\in\calF^{*}$, $s\hide\calA$ is the
  resulting
  trace of 
  substituting of any  occurrence of a feature $\fA\in\calA$ by
  the symbol $\bot$ in $s$. 
\edfn

\bdfn
  Let $M\in\calM$ and $\calA\subseteq\calF$. We define:
  \begin{displaymath}
    \semdenp{\cdot\hide{\calA}} = \accum\Bigl(\lbag(pr\hide{\calA},p)\
    |\ (pr,p)\in M\rbag \Bigr)
  \end{displaymath}
\edfn

As in the case of the other operators, we have to prove that the
operational semantics and the denotational semantics coincide. This is
proven by the next Proposition.
\bprop\label{prop:hid}
  \begin{displaymath}
    \prodp(P\hideA)  = \semdenp{\prodp(P)\hideA}
  \end{displaymath}
  \begin{proof}
    The proof is in the Appendix, page~\pageref{prof:prop:hid}.
  \end{proof}
\eprop


Finally, the next proposition allows to \emph{remove} the hiding
operator from any term. The idea is to substitute any occurrence of
the hidden actions by the symbol $\bot$. Let us note that we cannot
hide actions that appear in the restriction operators.  
\bprop
  Let $P,Q\in\fodaPAp$ be terms, $r\in (0,1]$ be a probability, and 
  $\calA\subseteq\calF$ be a set of hidden actions, then
  \begin{itemize}
  \item $\semdenp{\checkmark\hide\calA}=\semdenp{\checkmark}$
  \item $\semdenp{\nil\hide\calA}=\semdenp{\nil}$
  \item
      $\semdenp{(\fA;P)\hide\calA}=
      \begin{cases}
        \semdenp{\bot;(P\hide\calA)} & \mbox{si } A\in\calA\\
        \semdenp{\fA;(P\hide\calA)} & \mbox{si } A\not\in\calA\\
      \end{cases}$
  \item
      $\semdenp{(\ofA;_rP)\hide\calA}=
      \begin{cases}
        \semdenp{\overline{\bot};_r(P\hide\calA)} & \mbox{si } A\in\calA\\
        \semdenp{\ofA;_r(P\hide\calA)} & \mbox{si } A\not\in\calA\\
      \end{cases}$
  \item $\semdenp{(P\choice_P Q)\hide\calA}=\semdenp{(P\hide\calA)\choice_P (Q\hide\calA)}$
  \item $\semdenp{(P\paral Q)\hide\calA}=\semdenp{(P\hide\calA)\paral (Q\hide\calA)}$
  \item If $\fA,\fB\not\in\calA$ entonces 
    $\semdenp{(\require{A}{B}{P})\hide\calA}=\semdenp{\require{A}{B}{(P\hide\calA)}}$
  \item If $\fA,\fB\not\in\calA$ entonces 
    $\semdenp{(\exclude{B}{P})\hide\calA}=\semdenp{\exclude{A}{B}{(P\hide\calA)}}$
  \end{itemize}
  \begin{proof}
    In order to prove this properties it is enough to apply
    Proposition~\ref{prop:hid}.
  \end{proof}
\eprop


%%% Local Variables: 
%%% mode: latex
%%% TeX-master: "main"
%%% End: 
