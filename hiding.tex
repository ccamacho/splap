\section{Hiding sets of features}
\label{sec:stat:hid}

Computing the probability of a single features in a software product line
allows is a measure of the occurrences of this features in the set of
products. For instance, in case of testing it is interesting to know
the most frequent components to focus the testing in these components. 
This probability is affected by the  dependencies and restrictions of
the feature. 


In order to compute the probability of a set of features we are going
to use a hiding operator: we are going to hide other features. The
reason of doing this is that it maybe not feasible to compute all the
products of the software product line. But we expect that we can
achieve this if we restrict ourselves to a subset of features. So the
non interesting features are considered a new feature that we will
denote as $\bot\not\in\calF$, we consider the set $\calFbot =
\calF\cup\{\bot\}$.

\begin{figure}
  \centering
\begin{displaymath}
    \begin{array}{ccccc}
      \nombreRegla{hid1} & 
      \frac{P\tran{\fA}_p P', \fA\in\calA}{P\hide{\calA}\tran{\bot}_p P'\hide{\calA}} &
      \qquad \qquad \qquad&
      \nombreRegla{hid2} &     
        \frac{P\tran{\fA}_p P', \fA\not\in\calA}{P\hide{\calA}\tran{\fA}_p P'\hide{\calA}}
    \end{array}
  \end{displaymath}
  
  \caption{Operational semanitcs for the hiding operator}
  \label{fig:oper-hid}
\end{figure}



So we extend the set of operators with a new one: hiding a set of
features in a term.
\bdfn
  Let $\calA\subseteq \calF$ be a subset of features and
  $P\in\fodaPAp$ be a term,
  the $P\hide{\calA}$
  hiding operator of the features in $\calA$
  for the term $P$.
\edfn

Next we need to give the semantics of the new operator. First, the
operational semantics is given by the rules appearing in Figure~\ref{fig:oper-hid}.


\bdfn
  Dado el producto $pr\subseteq\calF$ y el conjunto de características ocultas $\calA\subseteq\calF$
  se define:
  
  \begin{displaymath}
    pr\hide{\calA} = \{\fA\ |\ \fA\in pr,\
    \fA\not\in\calA\}\cup
    \begin{cases}
      \{\bot\} & \mbox{si } pr\cap\calA\neq\emptyset\\
      \emptyset & \mbox{si } pr\cap\calA=\emptyset\\
    \end{cases}
  \end{displaymath}
  Dada la traza $s\in\calF^{*}$, $s\hide\calA$ representará la substitución
  de cada ocurrencia en la traza $s$ de cualquier característica en $\calA$ por $\bot$.
\edfn

\bdfn
  Dado el multiconjunto $M\in\calM$ y el conjunto de características ocultas $\calA\subseteq\calF$. Se define:
  \begin{displaymath}
    \semdenp{\cdot\hide{\calA}} = \accum\Bigl(\lbag(pr\hide{\calA},p)\
    |\ (pr,p)\in M\rbag \Bigr)
  \end{displaymath}
\edfn

\bprop
  Dados los términos $P,Q\in\fodaPAp$ y el conjunto de acciones ocultas $\calA\subseteq\calF$, entonces
  las siguientes proposiciones mantienen
  \begin{itemize}
  \item $\semdenp{\checkmark\hide\calA}=\semdenp{\checkmark}$
  \item $\semdenp{\nil\hide\calA}=\semdenp{\nil}$
  \item
      $\semdenp{(\fA;P)\hide\calA}=
      \begin{cases}
        \semdenp{\bot;(P\hide\calA)} & \mbox{si } A\in\calA\\
        \semdenp{\fA;(P\hide\calA)} & \mbox{si } A\not\in\calA\\
      \end{cases}$
  \item
      $\semdenp{(\ofA;_rP)\hide\calA}=
      \begin{cases}
        \semdenp{\overline{\bot};_r(P\hide\calA)} & \mbox{si } A\in\calA\\
        \semdenp{\ofA;_r(P\hide\calA)} & \mbox{si } A\not\in\calA\\
      \end{cases}$
  \item $\semdenp{(P\choice_P Q)\hide\calA}=\semdenp{(P\hide\calA)\choice_P (Q\hide\calA)}$
  \item $\semdenp{(P\paral Q)\hide\calA}=\semdenp{(P\hide\calA)\paral (Q\hide\calA)}$
  \item If $\fA,\fB\not\in\calA$ entonces 
    $\semdenp{(\require{A}{B}{P})\hide\calA}=\semdenp{\require{A}{B}{(P\hide\calA)}}$
  \item If $\fA,\fB\not\in\calA$ entonces 
    $\semdenp{(\exclude{B}{P})\hide\calA}=\semdenp{\exclude{A}{B}{(P\hide\calA)}}$
  \end{itemize}
  \bprf
    TO DO \lcomen{Espero que esto salga aplicando las definiciones. No
      creo que sea dificil, pero sí un conazo}
  \eprf
\eprop

\bprop
  $P\hide\calA\vtran{s}_r$ si y sólo si $r=\sum\lbag p \ |\ P\vtran{s'}_p Q,\
  s=s'\hide\calA\rbag$
  \textit{Demostración.}
    La demostración es realizada por inducción sobre la longitud
    de la traza $s$. Si la longitud es cero el resultado es trivial. Entonces
    se supone que
    $s=\fA\cdot s_1$. Si $\fA=bot$ entonces cualquier transición
    $P\hideA\tran{s}{p}Q\hideA$ puede ser dividida en transiciones,
    posiblemente mas de una, como por ejemplo.
    \begin{displaymath}
      P\hideA\tran{\bot}_{r_1}P_1\hideA\vtran{s_1}_{r2}Q
    \end{displaymath}
    Entonces se tiene
    \begin{displaymath}
      \begin{split}
        r = \sum\lbag p\ |\ P\hideA\vtran{s}_p Q\rbag = 
        \sum\lbag r_1\cdot r2\ |\
        P\hideA\tran{\bot}_{r_1}P_1\hideA\vtran{s_1}_{r2}Q\rbag =\\
        \sum\lbag r_1'\cdot r2\ |\
        P\hideA\tran{\fB}_{r_1'}P_1'\hideA\vtran{s_1}_{r2}Q,\ \fB\in\calA\rbag
      \end{split}
    \end{displaymath}
    Ahora por cada una de las $r_1'$, se puede aplicar el método inductivo a cada una de las
    transiciones $P_1'\hideA\vtran{s_1}_{r2}Q$ para obtener 
    $r_2=\sum\lbag r2'\ |\ P_1=\vtran{s_1'}Q,\
    s_1=s_1'\hideA\rbag$. Continuando la ecuación anterior :
    \begin{displaymath}
      \begin{split}
        \sum\lbag r_1'\cdot r2\ |\
        P\hideA\tran{\fB}_{r_1'}P_1'\hideA\vtran{s_1}_{r2}Q,\
        \fB\in\calA\rbag=\\
        \sum\lbag r_1'\cdot r2'\ |\ P\hideA\tran{\fB}_{r_1'}P_1'\vtran{s_1'}_{r2'}Q,\
        \fB\in\calA.\ s_1=s_1'\hideA\rbag = \\
        \sum\lbag r_1\cdot r2'\ |\ P\hideA\tran{\bot}_{r_1}P_1\vtran{s_1'}_{r2'}Q,\
        \fB\in\calA.\ s_1=s_1'\hideA\rbag =\\
        \sum\lbag r\ |\ P\vtran{s'}_{r}Q,\ s=s'\hideA\rbag
      \end{split}
    \end{displaymath}
    El caso cuando $\fA\not\in\calA$ es similar al anterior, solo necesita saltar el paso de $\fB$ a $\bot$.
  
\eprop

\bprop
  \begin{displaymath}
    \prodp(P\hideA)  = \semdenp{\prodp(P)\hideA}
  \end{displaymath}
  \textit{Demostración.}
    $(pr,p)\in \prodp(P\hideA)$ si y sólo si 
    \begin{displaymath}
      \begin{split}
        p = \sum\lbag r\ |\
          P\hideA\tran{s\checkmark}_rP'\hideA.\ pr=\product{s}\rbag =\\
        \sum\lbag r\ |\ P\tran{s'\checkmark}_rP',\ s=s'\hideA,\
          pr=\product{s}\rbag=\\
        \sum\lbag r\ |\ P\tran{s'\checkmark}_rP',\ s=pr\hideA\rbag =
        \\
        \sum\lbag r\ |\ (pr',r)\in\prodp(P),\ pr'=pr\hideA\rbag =
      \end{split}
    \end{displaymath}
    De tal manera 
    $(pr,p)\in\prodp(P\hideA)$ si y sólo si 
    $(pr,p)\in\semdenp{(\prodp(P))\hideA}$ 

\eprop
%%% Local Variables: 
%%% mode: latex
%%% TeX-master: "main"
%%% End: 
