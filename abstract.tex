\begin{abstract}
We introduce a probabilistic extension of our previous
work \fodaPA: a formal framework to specify and analyze software product lines.
We will use probabilistic information to identify those features that are used more frequently by computing the probability of having a feature in a specific software product line.
We redefine the syntax of \fodaPA\ to include probabilistic operators and define new operational and denotational semantics. We proof that the expected equivalence between these two semantic frameworks hold.
Our probabilistic framework is supported by a tool. We briefly comment on the characteristics of the tool and discuss the advantages of using probabilities to quantify the likelihood of having features in potential software product lines.
\end{abstract}
\textbf{Keywords;} Software Product Lines; Probabilistic Models; Formal Methods; Feature Models


%%% Local Variables:
%%% mode: latex
%%% TeX-master: "main"
%%% End:
