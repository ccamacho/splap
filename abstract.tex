\begin{abstract}
We introduce a probabilistic extension of our previous
work \fodaPA: a formal framework to specify and analyze software product lines.
We use probabilistic information to identify those features that are more frequently used. This is done by computing the probability of having a feature in a specific software product line, from now on \fodaPAp.
We redefine the syntax of \fodaPA\ to include probabilistic operators and define new operational and denotational semantics. We prove that the expected equivalence between these two semantic frameworks holds.
Our probabilistic framework is supported by a set of scripts
to show the model behavior. We briefly comment on the characteristics
of the scripts and discuss the advantages of using probabilities to quantify the likelihood of having features in potential software product lines.
\end{abstract}
\textbf{Keywords;} Software Product Lines; Probabilistic Models; Formal Methods; Feature Models


%%% Local Variables:
%%% mode: latex
%%% TeX-master: "main"
%%% End:
