\appendix
\section{Results for the proof of Proposition~\ref{prop:DenoOpe}}\label{app:proofs}

\begin{proofappendix}{Proposition~\ref{prop:nary}}.
  First of all, let us observe that $k_{n}$ is not defined above.
  In the rest of the prove let us define $k_{n}=1$.
  From the definition of $P$ we obtain that $P\tranp{\fA}{p} P'$ iff
  there is $1 \leq i\leq n$ and $q\in (0,1]$ such that
  $P_{i}\tranp{\fA}{q}P'$ and $p=q\cdot
  k_{i}\cdot\oriprod_{j=1}^{i-1}(1-k_{j})$.
  Then, it is enough to prove that $p_{i} =
  k_{i}\cdot\oriprod_{j=1}^{i-1}(1-k_{j})$. Or equivalently
  \begin{displaymath}
    k_{i} =
    \frac{p_{i}}{\oriprod_{j=1}^{i-1}(1-k_{j})}
  \end{displaymath}
  Then we need to prove
  \(\oriprod_{j=1}^{i-1}(1-k_{j})=1-\sum_{j=1}^{i-1}p_{j}\). Let us
  proceed by induction on $i$.
  \begin{description}
  \item[Base case.] If $i=1$ we obtain
    \(\oriprod_{j=1}^{0}(1-k_{j})=1\) and \(\sum_{j=1}^{0}p_{j}=0\).
  \item[Inductive case.] Let us assume $i>1$. Then
    \begin{displaymath}
      \oriprod_{j=1}^{i-1}(1-k_{j}) = (1-k_{i-1})\cdot \oriprod_{j=1}^{i-2}(1-k_{j})
    \end{displaymath}
    By induction hypothesis we obtain
    \(
      \oriprod_{j=1}^{i-2}(1-k_{j}) = 1-\sum_{j=1}^{i-2}p_{j}
    \).
    By definition
    \(
      k_{i-1}=\frac{p_{i-1}}{1-\sum_{j=1}^{i-2}p_{j}}
    \).
    Therefore
    \begin{displaymath}
      \begin{split}
        (1-k_{i-1})\cdot \oriprod_{j=1}^{i-2}(1-k_{j}) =
        \bigl(1-\frac{p_{i-1}}{1-\sum_{j=1}^{i-2}p_{j}}\bigr)\cdot
        \bigl(1-\sum_{j=1}^{i-2}p_{j}\bigr) = \\
        \bigl(\frac{1-\sum_{j=1}^{i-1}p_{j}}{1-\sum_{j=1}^{i-2}p_{j}}\bigr)\cdot
        \bigl(1-\sum_{j=1}^{i-2}p_{j}\bigr)=1-\sum_{j=1}^{i-1}p_{j}
      \end{split}
    \end{displaymath}
  \end{description}
\end{proofappendix}
\blem\label{lem:pref}
  Let $P\in\fodaPAp$ and $\feature{A}\in\calF$,
  then $(pr,p)\in\prodp(\feature{A};P)$ if and only if
  $$p=\sum\lbag r\ |\ (pr',r)\in\prodp(P) \y pr'\cup\{\feature{A}\} = pr\rbag$$
\begin{proof}
  The other transition of $\feature{A};P$ is
  $\feature{A};P\tranp{\feature{A}}{1}Q$.  Then
  $\feature{A};P\vtranp{s}{p}P$ if and only if
  \begin{displaymath}
    \feature{A};P\tranp{\feature{A}}{1}P\vtranp{s}{p}Q\quad\y\quad s=\feature{A}\cdot s'
  \end{displaymath}
  then
  \begin{equation*}
    \begin{split}
      p = \sum\lbag r \ |\ \feature{A};P\vtranp{s\checkmark}{p} \nil \y \product{s}=pr \ \rbag =\\
      \sum\lbag r\ |\ \feature{A};P\tranp{\feature{A}}{1}P\vtranp{s'\checkmark}{r}\nil\y \product{\feature{A}\cdot s'}=pr\rbag\\
      \sum\lbag r\ |\ P\vtranp{s'\checkmark}{r}\nil\y \{\feature{A}\}\cup \product{s'}=pr\rbag\\
      \sum\lbag r\ |\ (pr',r)\in\prodp(P)\y \{\feature{A}\}\cup
      pr'=pr\rbag
    \end{split}
  \end{equation*}
\end{proof}
\elem

\blem\label{lem:prefopt}
  Let $P\in\fodaPAp$, $\feature{A}\in\calF$ and $q\in (0,1)$,
  then $(pr,p)\in\prodp(\ofeature{A};_qP)$ if and only if $(pr,p)=(\emptyset,1-q)$ or
  $$p=q\cdot \sum\{r\ |\ (pr',r)\in\prodp(P) \y pr'\cup\{\feature{A}\} = pr\}$$
   \begin{proof}
    There exist two transitions to $\ofeature{A};_qP$: $\ofeature{A};_qP\tranp{\feature{A}}{q} P$ and
     $\ofeature{A};_qP\tranp{\checkmark}{1-q}\nil$. So forth if $\ofeature{A};_qP\vtranp{s}{r}Q$ then
     \begin{itemize}
     \item $s=\checkmark$ and $r=1-q$, or
     \item $s=\feature{A}\cdot s'$, $P\vtranp{s}{r'}Q$, and $r=q\cdot r'$.
     \end{itemize}
     So, if $pr=\product{\feature{A}\cdot s'}$ then
     $pr\neq\emptyset$. So then
     $(\emptyset,1-q)\in\prodp(\ofeature{A};_qP)$. Now suppose
     $pr\neq\emptyset$, then $(pr,p)\in\prodp(\ofeature{A};_qP)$ if and only if
     \begin{equation*}
       \begin{split}
         d = \sum\lbag r\ |\ \ofeature{A};_qP\vtran{s\checkmark}\nil\y \product{s}=pr\rbag=\\
         \sum\lbag r\ |\ \ofeature{A};_qP\tranp{\feature{A}}{q}P\vtranp{s'\checkmark}{r'}\nil\y \product{\feature{A}\cdot s'}=pr\y r=q\cdot r'\rbag=\\
         \sum\lbag r\ |\ P\vtranp{s'\checkmark}{r'}\nil\y \{\feature{A}\}\cup\product{s'}=pr\y r=q\cdot r'\rbag=\\
         \sum\lbag r\ |\ (pr',r')\in\prodp(P)\y \{\feature{A}\}\cup pr'=pr\y r=q\cdot r'\rbag=\\
         q\cdot\sum\lbag r'\ |\ (pr',r')\in\prodp(P)\y \{\feature{A}\}\cup pr'=pr\rbag
       \end{split}
     \end{equation*}
  \end{proof}
\elem


\blem\label{lem:choice1}
 Let $P,Q\in\fodaPAp$ and $q\in (0,1)$, then
 $P\choice_q Q\vtranp{s}{r} R$ if and only if
 \begin{itemize}
 \item $P\vtranp{s}{r'} R$ y $r=q\cdot r'$, o
 \item $Q\vtranp{s}{r'} R$ y $r=(1-q)\cdot r'$
 \end{itemize}
 \begin{proof}
   This lemma is a consequence of rules~\nombreRegla{cho1}
   and~\nombreRegla{cho2} from the operational semantics.
 \end{proof}
\elem

\blem\label{lem:choice}
  Let $P,Q\in\fodaPAp$ and $q\in (0,1)$, then
  $(pr,p)\in\prodp(P\choice_q Q)$ if and only if
  \begin{equation*}
    \begin{split}
      p = \left(q\cdot\sum\{r\ |\ (pr,r)\in\prodp(P) \}\right) +
      \left((1-q)\cdot\sum\{r\ |\ (pr,r)\in\prodp(Q)\}\right)
    \end{split}
  \end{equation*}
  \begin{proof}
    $(pr,p)\in\prodp(P\choice_q Q)$ if and only if
    \begin{equation*}
      \begin{split}
        p = \sum\lbag  r\ |\ P\choice_q Q\vtran{s\checkmark}_{r}\nil\rbag=\\
        \sum\lbag r\ |\ (P\vtran{s\checkmark}_{r'}\nil\y r=q\cdot
        r')\o (Q\vtran{s\checkmark}_{r'}\nil\y r=(1-q)\cdot r')
        \rbag=\\
        \sum\lbag r\ |\ P\vtran{s\checkmark}_{r'}\nil\y r=q\cdot
        r'\rbag +
        \sum\lbag  r\ |\ Q\vtran{s\checkmark}_{r'}\nil\y r=(1-q)\cdot r' \rbag=\\
        q\cdot\sum\lbag r\ |\ P\vtran{s\checkmark}_{r}\nil\rbag +
        (1-q)\cdot\sum\lbag  r\ |\ Q\vtran{s\checkmark}_{r}\nil\rbag=\\
        q\cdot\sum\lbag r\ |\ (pr,r)\in\prodp(P)\rbag +
        (1-q)\cdot\sum\lbag r\ |\ (pr,r)\in\prodp(Q)\rbag
      \end{split}
    \end{equation*}
  \end{proof}
\elem


\bdfn
Let \(P\in\fodaPAp\). We define the height of the syntactic tree of
\(P\), written \(\height(P)\) as follows:
\begin{displaymath}
  \begin{array}{l@{\hspace{1em} =\hspace{1em} }l}
    \height(\nil)& 0\\
    \height(\checkmark)& 1\\
    \begin{array}[c]{r}
    \height(\feature{A};P),\  \height(\ofeature{A};_{p} P),\\
      \height(\exclude{A}{B}{P}),\  \height(\forbid{A}{P}),\  \\
      \height(\require{A}{B}{P}),\ \height(\mandatory{A}{P})
    \end{array}
                 & 1 + \height(P)\\
    \height(P \choice_{p} Q),\ \height(P \paral Q) &
                                             1 = \max(\height(P), \height(Q))

  \end{array}
\end{displaymath}
\edfn

\blem
  Let \(P,P'\in\fodaPAp\), \(\fA\in\calF\), and \(p\in(0,1]\). If
  \(P\tran{\fA}_{p}P'\) then \(\height(P')<\height(P)\).
  \begin{proof}
    The proof is done easily by structural induction.
  \end{proof}
\elem

\blem\label{lem:paral}
Let $P,Q\in\fodaPAp$, $pr\subseteq\calF$ be a product, and
$p\in(0,1$), then $(pr,p)\in\prodp(P\paral Q)$ iff
\begin{displaymath}
  p = \sum\lbag r\ |\ \exists (pr_{1},p_{1})\in\prodp(Q),
  (pr_{2},p_{2})\in\prodp(Q):\ pr=pr_{1}\cup pr_{2}, r=p_{1}\cdot p_{2}\rbag
\end{displaymath}
\begin{proof}
  The proof is made by induction on \(\height(P)+\height(Q)\).
  First let us consider the base case \(\height(P)+\height(Q)=0\),
  that is \(P,Q\in\{\nil,\checkmark\}\).
  If $P=\nil$ (respectively \(Q=\nil\)) then
  \(P\paral Q\) has no products. If $P=Q\checkmark$ then
  \begin{displaymath}
    \prodp(P)=\prodp(Q)=\prodp(P\paral Q)=\{(0,1)\}
  \end{displaymath}
  from with the result
  are immediate from the definitions.

  So let assume
  the inductive case where $|pr|\geq 1$. In this case we obtain
  $(pr,p)\in\prodp(P\paral Q)$
  (by definition) iff
  \begin{equation}\label{eq:lempar1}
    p =
      \sum\lbag r\ |\ P\paral Q\vtran{s\checkmark}_{r}\nil,\ pr=\product{s}\rbag
    \end{equation}
    If $pr=\emptyset$, the only possible transition for $P\paral Q$ is
    the one derived from \nombreRegla{con3}. Then we obtain easily the result:
    \begin{displaymath}
      \begin{split}
        p = \sum\lbag r\ |\ P\paral Q\vtran{\checkmark}_{r}\nil\rbag =\\
        \sum\lbag r_{1}\cdot r_{2}\ |\ P\vtran{\checkmark}_{r_{1}}\nil,\
        Q\vtran{\checkmark}_{r_{2}}\nil\rbag = \\
        \sum\lbag r_{1}\cdot r_{2}\ |\ (\emptyset, r_{1})\in\prodp(P),
        \ (\emptyset, r_{2})\in\prodp(Q)\rbag
      \end{split}
    \end{displaymath}
    If \(pr\neq\emptyset\), we can split the previous sum according the
    first transition of \(P\paral Q\) according to rules
    \nombreRegla{con1}, \nombreRegla{con2}, \nombreRegla{con4}, and
    \nombreRegla{con5}.
    Since rules \nombreRegla{con1} and
    \nombreRegla{con4} are symmetric to \nombreRegla{con2} and
    \nombreRegla{con5}, we only show the corresponding transitions
    to the first two rules:
    \begin{equation}\label{eq:lempar2}
      \begin{split}
        \eqref{eq:lempar1} =
        \sum\lbag \frac{1}{2}\cdot r_{1}\cdot r_{2}\ |\
        P\tran{\fA}_{r_{1}}P',\\
        P'\paral
        Q\vtran{s'\checkmark}_{r_{2}}\nil,\
        pr=\{\fA\}\cup\product{s'}\rbag +\\
        \sum\lbag \frac{1}{2}\cdot r_{1}\cdot r_{2}\cdot r_{3}\ |\
        P\tran{\fA}_{r_{1}}P',\ Q\tran{\checkmark}_{r_{2}}\nil,\\ P'\vtran{s'\checkmark}_{r_{3}}\nil,\
        pr=\{\fA\}\cup\product{s'}\rbag+\\
        \mbox{\emph{term corresponding rule} \nombreRegla{con2}} +
        \mbox{\emph{term corresponding rule} \nombreRegla{con5}}
      \end{split}
    \end{equation}
    Applying the definitions and grouping traces giving the same
    product,
    the previous term can be transformed as follows
    \begin{equation}\label{eq:lempar3}
      \begin{split}
        \eqref{eq:lempar2} =
        \sum\lbag \frac{1}{2}\cdot r_{1}\cdot r_{2}\ |\
        P\tran{\fA}_{r_{1}}P',\\
        (pr', r_{2})\in\prodp(P'\paral Q),\
        pr=\{\fA\}\cup pr'\rbag +\\
        \sum\lbag \frac{1}{2}\cdot r_{1}\cdot r_{2}\cdot r_{3}\ |\
        P\tran{\fA}_{r_{1}}P',\
        (pr', r_{3})\in \prodp(P'),\\
        (\emptyset,r_{2})\in\prodp(Q),\
        pr=\{\fA\}\cup pr'\rbag+\\
        \mbox{term corresponding rule \nombreRegla{con2}} + \mbox{term corresponding rule \nombreRegla{con5}}
      \end{split}
    \end{equation}
    Now we can apply induction hypothesis to the first term of the
    previous sum (and the
    third that is symmetric).
    \begin{equation}\label{eq:lempar4}
      \begin{split}
        \eqref{eq:lempar3} =
        \sum\lbag\frac{1}{2}\cdot r_{1}\cdot r_{2}\cdot r_{3}\ |\
        P\tran{\fA}_{r_{1}}P',\
        (pr', r_{2})\in\prodp(P'), \\
        (pr'',r_{3})\in\prodp(Q),\
        pr=\{\fA\}\cup pr'\cup pr''\rbag +\\
        \sum\lbag \frac{1}{2}\cdot r_{1}\cdot r_{2}\cdot r_{3}\ |\
        P\tran{\fA}_{r_{1}}P',\ (\emptyset,r_{2})\in\prodp(Q),\\
        (pr', r_{3})\in \prodp(P'),\
        pr=\{\fA\}\cup pr'\rbag+\\
        \mbox{term corresponding rule \nombreRegla{con2}} + \mbox{term corresponding rule \nombreRegla{con5}}
      \end{split}
    \end{equation}
    Now let us consider the following set
    \begin{displaymath}
      \begin{split}
        \calQ = \{ (pr', r)\ |\ (pr', r)\in\prodp(Q),\ \exists P'\in\fodaPAp,\fA\in
        \calF,pr'\subseteq\calF,r_{1},r_{2}\in (0,1]:\\ P\tran{\fA}_{r_{1}} P',
        (pr'',\ r_{2})\in\prod(P'),\ pr=\{\fA\}\cup pr'\cup pr'' \}
      \end{split}
      \end{displaymath}
    All pairs in $\calQ$ appear in the first term of
    Equation~\eqref{eq:lempar4}. So we can apply the distributive
    property to reorganize that term obtaining
    \begin{equation}\label{eq:lempar5}
      \begin{split}
        \eqref{eq:lempar4} =
        \frac{1}{2}\sum_{(pr',r)\in\cal Q} r\cdot
        \sum\lbag r_{1}\cdot r_{2}\ |\
        P\tran{\fA}_{r_{1}}P',\
        (pr', r_{2})\in\prodp(P'),\\
        pr=\{\fA\}\cup pr'\cup pr''\rbag +\\
        \frac{1}{2}\cdot\sum\lbag  r_{1}\cdot r_{2}\cdot r_{3}\ |\
        P\tran{\fA}_{r_{1}}P',\ (\emptyset,r_{2})\in\prodp(Q),\\
        (pr', r_{3})\in \prodp(P'),\
        pr=\{\fA\}\cup pr'\rbag+\\
        \mbox{term corresponding rule \nombreRegla{con2}} + \mbox{term corresponding rule \nombreRegla{con5}}
      \end{split}
    \end{equation}
    If the empty product is not a product of $Q$ the second term of
    the previous sum may disappear. Otherwise there exists $r\in(0,1]$
    such that $(\emptyset, r)\in \prodp(Q)$. By definition,
    $(\emptyset, r)\in\calQ$, so we remove the empty set from the
    first term and we obtain:
    \begin{equation}\label{eq:lempar6}
      \begin{split}
        \eqref{eq:lempar5} =
        \frac{1}{2}\sum_{(pr',r)\in\calQ, pr'\neq\emptyset} r\cdot
        \sum\lbag r_{1}\cdot r_{2}\ |\
        P\tran{\fA}_{r_{1}}P',\\
        (pr'', r_{2})\in\prodp(P'),
        pr=\{\fA\}\cup pr'\cup pr''\rbag +\\
        \frac{1}{2}\cdot\sum\lbag  r_{1}\cdot r_{2}\cdot r_{3}\ |\
        P\tran{\fA}_{r_{1}}P',\ (\emptyset,r_{2})\in\prodp(Q),\\
        (pr', r_{3})\in \prodp(P'),\
        pr=\{\fA\}\cup pr'\rbag+\\
        \frac{1}{2}\cdot\sum\lbag  r_{1}\cdot r_{2}\cdot r_{3}\ |\
        P\tran{\fA}_{r_{1}}P',\ (\emptyset,r_{2})\in\prodp(Q),\\
        (pr', r_{3})\in \prodp(P'),\
        pr=\{\fA\}\cup pr'\rbag+\\
        \mbox{term corresponding rule \nombreRegla{con2}} + \mbox{term corresponding rule \nombreRegla{con5}}
      \end{split}
    \end{equation}
    Since the two last terms are identical can be added. Then,
    grouping the elements with the same product in the first, we obtain
    definition we obtain
    \begin{equation}\label{eq:lempar7}
      \begin{split}
        \eqref{eq:lempar6} =
        \frac{1}{2}\sum_{(pr,r)\in\calQ, pr'\neq\emptyset} r\cdot
        \sum\lbag r'\ |\
        (pr', r')\in\prodp(P), pr'\neq\emptyset,\\
        pr=pr\cup pr'\rbag +\\
        \sum\lbag  r_{1}\cdot r_{2}\ |\
        (pr,r_{1})\in\prod(P),\ pr\neq\emptyset,\
        (\emptyset, r_{2})\in \prodp(Q)\rbag+\\
        \mbox{term corresponding rule \nombreRegla{con2}} + \mbox{term corresponding rule \nombreRegla{con5}}
      \end{split}
    \end{equation}
    Rewriting, taking into account the definition of \(\calQ\) the
    previous equation we obtain
    \begin{equation}\label{eq:lempar8}
      \begin{split}
        \eqref{eq:lempar7} =
        \frac{1}{2}\sum\lbag r_{1}\cdot r_{2}\ |\ (pr_{1},r_{1})\in\prodp(Q),
        pr_{1}\neq\emptyset,\\ (pr_{2},p_{2})\in\prodp(P),\
        pr_{2}\neq\emptyset,pr = pr_{1}\cup pr_{2}\rbag +\\
        \sum\lbag  r_{1}\cdot r_{2}\ |\
        (pr,r_{1})\in\prodp(P),\ pr\neq\emptyset,\
        (\emptyset,r_{2})\in\prodp(Q)\rbag+\\
        \mbox{term corresponding rule \nombreRegla{con2}} + \mbox{term corresponding rule \nombreRegla{con5}}
      \end{split}
    \end{equation}
    Then adding the symmetrical terms, and having into account that
    $pr\neq\empty$, we obtain
    \begin{equation}\label{eq:lempar9}
      \begin{split}
        \eqref{eq:lempar8} =
        \sum\lbag r_{1}\cdot r_{2}\ |\ (pr_{1},r_{1})\in\prodp(Q),
        pr_{1}\neq\emptyset,\\ (pr_{2},p_{2})\in\prodp(P),
        pr_{2}\neq\emptyset, pr = pr_{1}\cup pr_{2}\rbag +\\
        \sum\lbag  r_{1}\cdot r_{2}\ |\
        (pr,r_{1})\in\prodp(P),\
        (\emptyset,r_{2})\in\prodp(Q)\rbag+\\
        \sum\lbag  r_{1}\cdot r_{2}\ |\
        (pr,r_{1})\in\prodp(Q),\
        (\emptyset,r_{2})\in\prodp(P)\rbag+\\
      \end{split}
    \end{equation}
    Finally we can include the two last terms into the first one
    having into account that $pr\neq\emptyset$.
    \begin{equation}\label{eq:lempar10}
      \begin{split}
        \eqref{eq:lempar9} =
        \sum\lbag r_{1}\cdot r_{2}\ |\ (pr_{1},r_{1})\in\prodp(Q),\\ (pr_{2},p_{2})\in\prodp(P),\
        pr = pr_{1}\cup pr_{2}\rbag
      \end{split}
    \end{equation}
  \end{proof}
\elem


% \bdfn
%   Let $s,s'\in\calF^*$ be traces, we denote with $\inter(s,s')$ the set of traces
%   obtained by alternating $s$ and $s'$.
% \edfn
    Since the two last terms are identical can be add    Since the two last terms are identical can be added and by
    definition we obtain
ed and by
    definition we obtain

% \blem\label{lem:paral}
%   Let $P,Q\in\fodaPAp$, $p,q\in (0,1)$ and
%   $s,s'\in\calF^*$
%   such $p=\sum\lbag p'\ |\ P\vtranp{s\checkmark}{p'}\nil\rbag$ and
%   $q=\sum\lbag q'\| |\ Q\vtranp{s'}{q'}\nil\rbag$
%   then
%   $$p\cdot q = \sum\lbag r\ |\ P\paral Q\vtranp{s''\checkmark}{r}\nil\y s''\in\inter(s,s)\rbag$$
%   \begin{proof}
%     By induction of $|s|+|s'|$.
%     \begin{description}
%     \item[$|s|+|s'|=0$] Since
%       $P\paral Q\tranp{\checkmark}{r}\nil$
%       if and only if $P\tranp{\checkmark}{r_1}\nil$,
%       $Q\tranp{\checkmark}{r_2}\nil$, and $r=r_1\cdot r_2$,
%       behold the following:
%       \begin{equation}
%         \label{eq:paral1}
%         \begin{split}
%           \sum \lbag r\ |\ P\paral Q\tranp{\checkmark}{r}\nil \rbag =\\
%           \sum \lbag r\ |\ P\tranp{\checkmark}{r_1}\nil\y Q\tranp{\checkmark}{r_2}\nil\y r=r_1\cdot r_2 \rbag=\\
%           \sum \lbag  r_1\cdot r_2 \ |\ P\tranp{\checkmark}{r_1}\nil\y Q\tranp{\checkmark}{r_2}\nil \rbag= \\
%           \sum \scaleleftright{\lbag}{r_1 \cdot \left(\sum \lbag r_2\ |\ Q\tranp{\checkmark}{r_2}\nil \rbag\right)\ \vstretch{2}{|}\  P\tranp{\checkmark}{r_1}\nil}{\rbag} = \\
%           \sum \lbag  r_1 \cdot q |\  P\tranp{\checkmark}{r_1}\nil\rbag = q\cdot \sum \lbag  r_1  |\  P\tranp{\checkmark}{r_1}\nil\rbag = q\cdot p
%         \end{split}
%       \end{equation}
%     \item[$|s|+|s'|>0$] Suppose that $|s|>0$ (the case $|s'|>0$
%       is symmetric). Let's consider that $s=\feature{A}\cdot s_1$.
%       Now consider the multiset
%       $\calP=\lbag P'\ |\ P\tranp{\f{A}}{r'} P'\vtranp{s_1\checkmark}{p'}\nil\rbag$ and given $r_1 = \sum\lbag r\ |\ P\tranp{\f{A}}{r} P'\y P'\in\calP\rbag$ and
%       $p_1=\sum\lbag r\ |\ P'\vtranp{s_1\checkmark}{r}\nil\y P'\in\calP\rbag$. It easy to verify that $p=r_1\cdot p_1$:\\
%       \scalebox{0.9}{\parbox{1.1\hsize}{\begin{equation}
%         \label{eq:paral2}
%         \begin{split}
%           p = \sum\lbag  r\ |\ P\vtranp{s\checkmark}{r}\nil \rbag =  \\
%           \sum\lbag  r'\cdot p'\ |\  P\tranp{\f{A}}{r'} P'\vtranp{s_1\checkmark}{p'}\nil \rbag =\\
%           \sum\scaleleftright{\lbag}{r'\cdot\underbrace{\left(\sum\lbag p'\ |\ P'\vtranp{s_1\checkmark}{p'}\nil\rbag\right)}_{p_1}\ \vstretch{2}{|}\  P\tranp{\f{A}}{r'} P'\y P'\in\calP}{\rbag} =\\
%           p_1\cdot\sum\lbag  r'\ |\  P\tranp{\f{A}}{r'} P'\y P'\in\calP \rbag = r_1\cdot p_1
%         \end{split}
%       \end{equation}}}\\
%       For any $P'\in\calP$, and then for the inductive hypothesis we obtain
%       $$
%       p'\cdot q =\sum\lbag r\ |\ P'\paral Q\vtranp{s''\checkmark}{r}\y s''\in\inter(s_1,s')\rbag
%       $$
%       So forth
%       \begin{equation}
%         \label{eq:paral3}
%         \begin{split}
%           \sum\lbag r\ |\ P'\in\calP \y P'\paral Q\vtranp{s''\checkmark}{r}\y s''\in\inter(s_1,s')\rbag = \\
%           \sum\lbag p'\cdot q\ |\  P'\in\calP\y P'\vtranp{s_1\checkmark}{p'}\nil\y s''\in\inter(s_1,s')\rbag\\
%           q\cdot \sum\lbag r\ |\ P'\vtranp{s_1\checkmark}\nil\y P'\in\calP\rbag=p_1\cdot q
%         \end{split}
%       \end{equation}
%       then for any $P'\in\calP$ we obtain $P\paral Q\vtranp{s''\checkmark}{r}\nil$
%       where $P\tranp{\f{A}}{r'}P'$, $r=\frac{r'}{2}\cdot p'\cdot q$ and $s''$
%       is the alternation of $s$ and $s'$. So forth\\
%       \scalebox{0.8}{\parbox{1.3\hsize}{\begin{equation}
%         \label{eq:paral4}
%         \begin{split}
%           \sum\lbag  r\ |\ P\paral Q\tranp{\f{A}}{\frac{r_1}{2}}P'\paral Q\vtranp{s''\checkmark}{r_2}\nil\y P'\tranp{\f{A}}{r_1}P' \y \\ r=\frac{r_1}{2}\cdot r_1\y s''\in\inter(s_1,s') \rbag = \\
%           \sum\scaleleftright{\lbag}{\frac{r'}{2}\cdot r'' \  \vstretch{2}{|}\ P\tranp{\f{A}}{r'}P'\y P'\paral Q\vtranp{s''\checkmark}{r''}\nil\y s''\in\inter(s_1,s') }{\rbag} = \\
%           \sum\scaleleftright{\lbag}{  \frac{r'}{2}\cdot\left(\underbrace{\sum \scaleleftright{\lbag}{ r''\ \vstretch{4}{|}\
%                   \begin{split}
%                     P'\paral Q\vtranp{s''\checkmark}{r''}\nil\y \\s''\in\inter(s_1,s')
%                   \end{split}}{\rbag}}_{\mathrm{induction\ hypothesis:}\ p_1\cdot q} \right) \ \vstretch{4}{|}\
%             P\tranp{\f{A}}{r'}P'\y P'\in\calP  }{\rbag} = \\
%           \sum\scaleleftright{\lbag}{  \frac{r'}{2}\cdot (p_1\cdot q) \  \vstretch{2}{|}\ P\tranp{\f{A}}{r'}P' \y P'\in\calP   }{\rbag} = \\
%           \frac{(p_1\cdot q)}{2}\cdot  \sum\lbag  r'\ |\ P\tranp{\f{A}}{r'}P' \y P'\in\calP  \rbag = \\
%           \frac{1}{2}(p_1\cdot q\cdot r_1)=\frac{1}{2}(p\cdot q)
%         \end{split}
%       \end{equation}}}

%       Equation~\ref{eq:paral4} groups all probabilities
%       of alternating  $s$ and $s'$ where
%       the first action of $s$ happens in the first place.
%       Then we have to group the probabilistic
%       information when the first action of $s'$ happens in the first place.

%       There are two possibilities, depending on whether these actions exist or not.
%       Depending on $s'$ we obtain the following options.
%       \begin{description}
%       \item[$|s'|=0$]
%         In this case we have $q=\sum\lbag q'\ |\ Q\tranp{\checkmark}{q'}\nil \rbag$,
%         If $Q\tranp{\checkmark}{q'}\nil$, when applying
%         the rule $\nombreRegla{con4}$ and then obtain
%         $P\paral Q\tranp{\f{A}}{\frac{r'\cdot q'}{2}} P_1$.
%         In this case we obtain\\
%         \scalebox{0.9}{\parbox{1.1\hsize}{\begin{equation}
%           \label{eq:paral5}
%           \begin{split}
%             \sum\lbag  r\ |\  P\paral Q\tranp{\f{A}}{\frac{q'\cdot r'}{2}} P'\vtranp{s_1\checkmark}{p'}\nil\y r=p'\cdot \frac{q'\cdot r'}{2}\rbag=\\
%             \sum\lbag \frac{1}{2}\cdot(p'\cdot r'\cdot q' \ |\  P\paral Q\tranp{\f{A}}{\frac{q'\cdot r'}{2}} P'\vtranp{s_1\checkmark}{p'}\nil \rbag=\\
%             \sum\lbag \frac{1}{2}\cdot(p'\cdot r'\cdot (\underbrace{\sum\lbag q'\ |\ Q\tranp{\checkmark}{q'}\nil \rbag}_{q}) \ |\  P\tranp{\f{A}}{r'}P'\y P'\vtranp{s_1\checkmark}{p'}\nil \rbag=\\
%             \frac{q}{2}\cdot \sum\lbag r'\cdot (\underbrace{\sum\lbag p'\ |\ P'\vtranp{s_1\checkmark}{p'}\nil \rbag}_{p_1}) \ |\  P\tranp{\f{A}}{r'}P'\y P'\in\calP \rbag=\\
%             \frac{1}{2}(q\cdot p_1)\cdot(\underbrace{\sum\lbag r'\ |\ P\tranp{\f{A}}{r'}P'\y P'\in\calP \rbag}_{r_1})=\frac{1}{2}\cdot(q\cdot p_1\cdot r_1)=\frac{1}{2}(q\cdot p)
%           \end{split}
%         \end{equation}}}\\
%         The only transition of $P\paral Q$ that allows
%         to alternate $s$ and $s'$ has the form
%         \begin{enumerate}
%         \item $P\paral Q\tranp{\f{A}}{\frac{r'}{2}} P'\paral Q\vtranp{s''\checkmark}{p'}\nil $ as is indicated above
%         , or
%         \item $P\paral Q\tranp{\f{A}}{\frac{q'\cdot r'}{2}} P'\vtranp{s''\checkmark}{p'}\nil $.
%         \end{enumerate}
%         So forth
%         \begin{equation}
%           \begin{split}
%             \label{eq:paral5}
%             \sum\lbag  r\ |\  P\paral Q\vtran{s\checkmark}\nil\rbag =\\
%             \sum\lbag  r\ |\ P\paral Q\tranp{\f{A}}{\frac{r_1}{2}}P'\paral Q\vtranp{s''\checkmark}{r_2}\nil\y P'\tranp{\f{A}}{r_1}P' \y \\ r=\frac{r_1}{2}\cdot r_1\y s''\in\inter(s_1,s') \rbag +\\
%             \sum\lbag  r\ |\  P\paral Q\tranp{\f{A}}{\frac{q'\cdot r'}{2}} P'\vtranp{s_1\checkmark}{p'}\nil\y r=p'\cdot \frac{q'\cdot r'}{2}\rbag=\\
%             \frac{1}{2}(p\cdot q) + \frac{1}{2}(p\cdot q)= p\cdot q
%           \end{split}
%         \end{equation}
%        \item[$|s'|>0$] Let consider $s'=\f{B}\cdot s_2$.
%         In this case we consider the alternations where
%         $Q$ produces the feature $\f{B}$ in the first place, that is,
%         first we apply the rule~\nombreRegla{con2}. Considering
%         multiset $\calQ=\lbag Q'\ |\ Q\tranp{\f{B}}{r}Q'\vtranp{s_2\checkmark}{q'}\nil\rbag$, and given
%         $r_2=\sum\lbag r\ |\ Q\tranp{\f{B}}{r}Q'\y Q'\in\calQ\rbag$ and $q_1=\sum\lbag r\ |\ Q'\vtranp{s_2\checkmark}{r}\nil\y Q'\in\calQ\rbag$,
%         with a similar reasoning as for the equations~\eqref{eq:paral1} and~\eqref{eq:paral2} we obtain that $q=r_2\cdot q_1$ and
%         \begin{equation}
%           \label{eq:paral6}
%           \begin{split}
%             \sum\lbag  r\ |\ P\paral Q\tranp{\f{B}}{\frac{r_1}{2}}P\paral Q'\vtranp{s''\checkmark}{r_2}\nil\y\\ r=\frac{r_1}{2}\cdot r_1\y s''\in\inter(s,s_2) \rbag = \frac{1}{2}(p\cdot q)
%           \end{split}
%         \end{equation}
%         The only transition of $P\paral Q$ that implies alternating
%         $s$ and $s'$ has the form:
%         \begin{enumerate}
%         \item $P\paral Q\tranp{\f{A}}{p'\cdot\frac{r}{2}} P'\paral Q\vtranp{s''\checkmark}{p'}\nil $ where $s''$ is an alternation of
%         $s_1$ and $s'$, or
%         \item $P\paral Q\tranp{\f{B}}{q'\cdot\frac{r}{2}} P\paral Q'\vtranp{s''\checkmark}{p'}\nil $ where $s''$ is an alternation of
%         $s$ and $s_2$,
%         \end{enumerate}
%         So forth,
%         \begin{equation}
%           \label{eq:paral7}
%           \begin{split}
%             \sum\lbag  r\ |\  P\paral Q\vtran{s\checkmark}\nil\rbag =\\
%             \sum\lbag  r\ |\ P\paral Q\tranp{\f{A}}{\frac{r_1}{2}}P'\paral Q\vtranp{s''\checkmark}{r_2}\nil\y P'\tranp{\f{A}}{r_1}P' \y\\ r=\frac{r_1}{2}\cdot r_1\y s''\in\inter(s_1,s') \rbag + \\
%             \sum\lbag  r\ |\ P\paral Q\tranp{\f{B}}{\frac{r_1}{2}}P\paral Q'\vtranp{s''\checkmark}{r_2}\nil\y Q'\tranp{\f{B}}{r_1}Q' \y\\ r=\frac{r_1}{2}\cdot r_1\y s''\in\inter(s,s_2) \rbag = \\
%             \frac{1}{2}(p\cdot q) + \frac{1}{2}(p\cdot q)= p\cdot q
%           \end{split}
%         \end{equation}
%       \end{description}
%     \end{description}
% \end{proof}
% \elem

\blem\label{lem:mand}
  Let $P\in\fodaPAp$, $\feature{A}\in\calF$ and $P\vtranp{s\checkmark}{p}\nil$.
  \begin{enumerate}
  \item $\f{A}\in s$ if and only if $\mandatory{A}{P}\vtranp{s\checkmark}{p}\nil$.
  \item $\f{A}\not\in s$ if and only if $\mandatory{A}{P}\vtranp{s\f{A}\checkmark}{p}\nil$.
  \end{enumerate}
  \begin{proof}
    In both cases the proof is made by induction of the length
    of $s$.
  \end{proof}
\elem


\blem\label{lem:forb}
  Let $P\in\fodaPAp$, $\feature{A}\in\calF$, $s\in\calF^*$ and
  $p\in(0,1)$. $P\vtranp{s\checkmark}{p}\nil$, if and only if
  $\forbid{P}{A}\vtranp{s\checkmark}{p}\nil$ and $\f{A}\not\in s$.
  \begin{proof}
    The proof is simply by induction on the length of $s$.
  \end{proof}
\elem


\blem\label{lem:req}
  Let $P\in\fodaPAp$, $\feature{A},\feature{B}\in\calF$, $s\in\calF^*$
  and $p\in(0,1)$. Then $P\vtranp{s\checkmark}{p}\nil$ if and only if
  $\require{A}{B}{P}\vtranp{s'\checkmark}p\nil$ and
  $s'$ is in the form:
  $\f{A}\not\in s$ and $s'=s$, $\f{B}\in s$ and $s'=s$, or
  $\f{A}\in s$, $\f{B}\not\in s$ and $s'=s\cdot\f{B}$.
  \begin{proof}
    By induction of the length of $s$.
    \begin{description}
    \item[$|s|=0$] In this case $P\tranp{\checkmark}{p}\nil$.
    We obtain the result applying the rule~\nombreRegla{req3}.
    \item[$|s|>0$] Now we can distinguish three cases depending on
    the first feature of $s$:
      \begin{description}
      \item[$s=\f{A}s_1$.] In this case there exist $p_1,q\in(0,1)$ such that
        $P\tranp{\f{A}}{p_1}P_1\vtranp{s_1\checkmark}{q}\nil$.  When
        applying the rule~\nombreRegla{req2} we obtain
        $\require{A}{B}{P}\tranp{\f{A}}{p_1}\mandatory{B}{P_1}$.
        We obtain the result by applying the lemma~\ref{lem:mand}.
      \item[$s=\f{B}s_1$.] In this case there exist $p_1,q\in(0,1)$ such that
        $P\tranp{\f{A}}{p_1}P_1\vtranp{s_1\checkmark}{q}\nil$.
        When applying the rule~\nombreRegla{req2} we obtain
        $\require{A}{B}{P}\tranp{\f{B}}{p_1}\mandatory{A}{P_1}$.
        We obtain the result by applying the lemma~\ref{lem:mand}.
      \item[$s=\f{C}s_1$ with $\f{C}\neq\f{A}$ and $\f{C}\neq\f{A}$.]
        In this case there exist $p_1,q\in(0,1)$ such that
        $P\tranp{\f{C}}{p_1}P_1\vtranp{s_1\checkmark}{q}\nil$.
        When applying the rule~\nombreRegla{req1}, we obtain
        $ \require{A}{B}{P}\tranp{\f{C}}{p_1}\require{A}{B}{P_1} $, and
        then the result by applying the inductive hypothesis over
        $s_1$.
      \end{description}
    \end{description}
  \end{proof}
\elem


\blem\label{lem:excl}
  Let $P\in\fodaPAp$, $\feature{A},\feature{B}\in\calF$, $s\in\calF^*$ and
  $p\in(0,1)$. Then $P\vtranp{s\checkmark}{p}\nil$ if and only if
  $\exclude{A}{B}{P}\vtranp{s\checkmark}{p}\nil$, $\f{A}\not\in s$ and $\f{B}\not\in s$.
  \begin{proof}
    By the induction on the length of $s$.
    \begin{description}
    \item[$|s|=0$] In this case $P\tranp{\checkmark}{p}\nil$.
      We obtain the result by applying the rule~\nombreRegla{excl4}.
    \item[$|s|>0$] Now it is possible to distinguish three cases depending
    on the first feature of $s$:
      \begin{description}
      \item[$s=\f{A}s_1$.] In this case there exist $p_1,q\in(0,1)$
        such that $P\tranp{\f{A}}{p_1}P_1\vtranp{s_1\checkmark}{q}\nil$.
        When applying rule~\nombreRegla{req2}
        we obtain $\require{A}{B}{P}\tranp{\f{A}}{p_1}\forbid{B}{P_1}$.
        Now based on Lemma~\ref{lem:mand},
        \begin{itemize}
        \item $\f{B}\in s_1$ if and only if $\mandatory{B}{P_1}\vtranp{s_1\cdot \checkmark}{q}\nil$.
        \item $\f{B}\not\in s_1$ if and only if  $\mandatory{B}{P_1}\vtranp{s_1\f{B}\checkmark}{q}\nil$.
        \end{itemize}
      \item[$s=\f{C}s_1$ with $\f{C}\neq\f{A}$.]
        In this case there exist $p_1,q\in(0,1)$
        such that $P\tranp{\f{C}}{p_1}P_1\vtranp{s_1\checkmark}{q}\nil$.
        When applying rule~\nombreRegla{req1},  we obtain
        $
        \require{A}{B}{P}\tranp{\f{C}}{p_1}\require{A}{B}{P_1}
        $, and then the result is obtained by applying the inductive
        hypothesis over $s_1$.
      \end{description}
    \end{description}
      \end{proof}
\elem

\section{Proof of Proposition~\ref{prop:hid}}\label{appendixB}

\bprop\label{prop:appendixB}
  $P\hide\calA\vtran{s}_rQ\hide\calA$ if and only if $r=\sum\lbag p \ |\ P\vtran{s'}_p Q,\
  s=s'\hide\calA\rbag$
  \begin{proof}
    The proof is achieved by induction over the length of the trace
    $s$. If the length is zero the result is trivial. Then we suppose that
    $s=\fA\cdot s_1$. If $\fA=\bot$ then any transition
    $P\hideA\vtranp{s}{p}Q\hideA$ can be divided in transitions,
    possibly more than one, for example.
    \begin{displaymath}
      P\hideA\tranp{\bot}{r_1}P_1\hideA\vtran{s_1}_{r2}Q
    \end{displaymath}
    then we have
    \begin{displaymath}
      \begin{split}
        r = \sum\lbag p\ |\ P\hideA\vtran{s}_p Q\rbag =
        \sum\lbag r_1\cdot r2\ |\
        P\hideA\tran{\bot}_{r_1}P_1\hideA\vtran{s_1}_{r2}Q\rbag =\\
        \sum\lbag r_1'\cdot r2\ |\
        P\hideA\tran{\fB}_{r_1'}P_1'\hideA\vtran{s_1}_{r2}Q,\ \fB\in\calA\rbag
      \end{split}
    \end{displaymath}
    Now for each $r_1'$, we can apply the induction
    hypothesis to each of the transitions
    $P_1'\hideA\vtran{s_1}_{r2}Q$ to obtain
    $r_2=\sum\lbag r2'\ |\ P_1\vtran{s_1'}Q,\
    s_1=s_1'\hideA\rbag$. Continuing the last equation:
    \begin{displaymath}
      \begin{split}
        \sum\lbag r_1'\cdot r2\ |\
        P\hideA\tran{\fB}_{r_1'}P_1'\hideA\vtran{s_1}_{r2}Q,\
        \fB\in\calA\rbag=\\
        \sum\lbag r_1'\cdot r2'\ |\ P\hideA\tran{\fB}_{r_1'}P_1'\vtran{s_1'}_{r2'}Q,\
        \fB\in\calA.\ s_1=s_1'\hideA\rbag = \\
        \sum\lbag r_1\cdot r2'\ |\ P\hideA\tran{\bot}_{r_1}P_1\vtran{s_1'}_{r2'}Q,\
        \fB\in\calA.\ s_1=s_1'\hideA\rbag =\\
        \sum\lbag r\ |\ P\vtran{s'}_{r}Q,\ s=s'\hideA\rbag
      \end{split}
    \end{displaymath}
    The case $\fA\not\in\calA$ is similar to the last one:
    we just skip the step from $\fB$ to $\bot$.
  \end{proof}
\eprop


\paragraph{Proof of Proposition~\ref{prop:hid}}\label{prof:prop:hid}
     $(pr,p)\in \prodp(P\hideA)$ if and only if
    \begin{displaymath}
      \begin{split}
        p = \sum\lbag r\ |\
        P\hideA\vtran{s\checkmark}_rP'\hideA.\ pr=\product{s}\rbag =\\
        \sum\lbag r\ |\ P\vtran{s'\checkmark}_rP',\ s=s'\hideA,\
        pr=\product{s}\rbag=\\
        \sum\lbag r\ |\ P\vtran{s'\checkmark}_rP',\ s=pr\hideA\rbag =
        \\
        \sum\lbag r\ |\ (pr',r)\in\prodp(P),\ pr'=pr\hideA\rbag =
      \end{split}
    \end{displaymath}
    So, $(pr,p)\in\prodp(P\hideA)$ if and only if
    $(pr,p)\in\semdenp{(\prodp(P))\hideA}$



%%% Local Variables:
%%% mode: latex
%%% TeX-master: "main"
%%% End:
