\appendix
\section{Proofs of Setion~\ref{sec:equivalence}}\label{app:proofs}

\blem\label{prop:pref}
  Dado el término $P\in\fodaPAp$ y la característica $\feature{A}\in\calF$, 
  entonces $(pr,p)\in\prodp(\feature{A};P)$ si y sólo si
  $$p=\sum\lbag r\ |\ (pr',r)\in\prodp(P) \y pr'\cup\{\feature{A}\} = pr\rbag$$
\begin{proof}
  La otra transición de $\feature{A};P$ es
  $\feature{A};P\tranp{\feature{A}}{1}Q$.  Entonces
  $\feature{A};P\vtranp{s}{p}P$ si y sólo si
  \begin{displaymath}
    \feature{A};P\tranp{\feature{A}}{1}P\vtranp{s}{p}Q\quad\y\quad s=\feature{A}\cdot s'
  \end{displaymath}
  Entonces
  \begin{equation*}
    \begin{split}
      p = \sum\lbag r \ |\ \feature{A};P\vtranp{s\checkmark}{p} \nil \y \product{s}=pr \ \rbag =\\
      \sum\lbag r\ |\ \feature{A};P\tranp{\feature{A}}{1}P\vtranp{s'\checkmark}{r}\nil\y \product{\feature{A}\cdot s'}=pr\rbag\\
      \sum\lbag r\ |\ P\vtranp{s'\checkmark}{r}\nil\y \{\feature{A}\}\cup \product{s'}=pr\rbag\\
      \sum\lbag r\ |\ (pr',r)\in\prodp(P)\y \{\feature{A}\}\cup
      pr'=pr\rbag
    \end{split}
  \end{equation*}
\end{proof}
\elem

\blem\label{lem:prefopt}
  Dado el término $P\in\fodaPAp$, la característica $\feature{A}\in\calF$, y la probabilidad $q\in (0,1)$, 
  entonces $(pr,p)\in\prodp(\ofeature{A};_qP)$ si y sólo si $(pr,p)=(\emptyset,1-q)$ o
  $$p=q\cdot \sum\{r\ |\ (pr',r)\in\prodp(P) \y pr'\cup\{\feature{A}\} = pr\}$$
  \bprf
    Existen dos transiciones para $\ofeature{A};_qP$: $\ofeature{A};_qP\tranp{\feature{A}}{q} P$ y
     $\ofeature{A};_qP\tranp{\checkmark}{1-q}\nil$. Por lo tanto si $\ofeature{A};_qP\vtranp{s}{r}Q$ entonces
     \begin{itemize}
     \item $s=\checkmark$ y $r=1-q$, o
     \item $s=\feature{A}\cdot s'$, $P\vtranp{s}{r'}Q$, y $r=q\cdot r'$.
     \end{itemize}
     Además, si $pr=\product{\feature{A}\cdot s'}$ entonces
     $pr\neq\emptyset$. Por lo tanto 
     $(\emptyset,1-q)\in\prodp(\ofeature{A};_qP)$. Ahora suponemos 
     $pr\neq\emptyset$, entonces $(pr,p)\in\prodp(\ofeature{A};_qP)$ si y sólo si
     \begin{equation*}
       \begin{split}
         d = \sum\lbag r\ |\ \ofeature{A};_qP\vtran{s\checkmark}\nil\y \product{s}=pr\rbag=\\
         \sum\lbag r\ |\ \ofeature{A};_qP\tranp{\feature{A}}{q}P\vtranp{s'\checkmark}{r'}\nil\y \product{\feature{A}\cdot s'}=pr\y r=q\cdot r'\rbag=\\
         \sum\lbag r\ |\ P\vtranp{s'\checkmark}{r'}\nil\y \{\feature{A}\}\cup\product{s'}=pr\y r=q\cdot r'\rbag=\\
         \sum\lbag r\ |\ (pr',r')\in\prodp(P)\y \{\feature{A}\}\cup pr'=pr\y r=q\cdot r'\rbag=\\
         q\cdot\sum\lbag r'\ |\ (pr',r')\in\prodp(P)\y \{\feature{A}\}\cup pr'=pr\rbag
       \end{split}
     \end{equation*}
  \eprf
\elem


\blem\label{lem:choice1}
 Dados los términos $P,Q\in\fodaPAp$, y la probabilidad $q\in (0,1)$, entonces 
 $P\choice_q Q\vtranp{s}{r} R$ si y sólo si
 \begin{itemize}
 \item $P\vtranp{s}{r'} R$ y $r=q\cdot r'$, o
 \item $Q\vtranp{s}{r'} R$ y $r=(1-q)\cdot r'$
 \end{itemize}
 \bprf
   Este lema es consecuencia de las reglas~\nombreRegla{cho1} y~\nombreRegla{cho2} de la semántica operacional.
 \eprf
\elem

\blem\label{lem:choice}
  Dados los términos $P,Q\in\fodaPAp$, y la probabilidad $q\in (0,1)$, entonces
  $(pr,p)\in\prodp(P\choice_q Q)$ si y sólo si
  \begin{equation*}
    \begin{split}
      p = \left(q\cdot\sum\{r\ |\ (pr,r)\in\prodp(P) \}\right) +
      \left((1-q)\cdot\sum\{r\ |\ (pr,r)\in\prodp(Q)\}\right)
    \end{split}
  \end{equation*}  
  \begin{proof}
    $(pr,p)\in\prodp(P\choice_q Q)$ si y sólo si
    \begin{equation*}
      \begin{split}
        p = \sum\lbag  r\ |\ P\choice_q Q\vtran{s\checkmark}_{r}\nil\rbag=\\
        \sum\lbag r\ |\ (P\vtran{s\checkmark}_{r'}\nil\y r=q\cdot
        r')\o (Q\vtran{s\checkmark}_{r'}\nil\y r=(1-q)\cdot r')
        \rbag=\\
        \sum\lbag r\ |\ P\vtran{s\checkmark}_{r'}\nil\y r=q\cdot
        r'\rbag +
        \sum\lbag  r\ |\ Q\vtran{s\checkmark}_{r'}\nil\y r=(1-q)\cdot r' \rbag=\\
        q\cdot\sum\lbag r\ |\ P\vtran{s\checkmark}_{r}\nil\rbag +
        (1-q)\cdot\sum\lbag  r\ |\ Q\vtran{s\checkmark}_{r}\nil\rbag=\\
        q\cdot\sum\lbag r\ |\ (pr,r)\in\prodp(P)\rbag +
        (1-q)\cdot\sum\lbag r\ |\ (pr,r)\in\prodp(Q)\rbag
      \end{split}
    \end{equation*}
  \end{proof}
\elem


\bdfn
  Dadas las trazas $s,s'\in\calF^*$, se denota el conjunto de trazas obtenidas de
  alternar $s$ y $s'$ por $\inter(s,s')$.
\edfn

\blem\label{lem:paral1}
  Dados los términos $P,Q\in\fodaPAp$, las probabilidades $p,q\in (0,1)$y las trazas $s,s'\in\calF^*$ 
  tal que $p=\sum\lbag p'\ |\ P\vtranp{s\checkmark}{p'}\nil\rbag$ y 
  $q=\sum\lbag q'\| |\ Q\vtranp{s'}{q'}\nil\rbag$
  Entonces
  $$p\cdot q = \sum\lbag r\ |\ P\paral Q\vtranp{s''\checkmark}{r}\nil\y s''\in\inter(s,s)\rbag$$
  \textit{Demostración.}
    Por inducción de $|s|+|s'|$.
    \begin{description}
    \item[$|s|+|s'|=0$] Desde 
      $P\paral Q\tranp{\checkmark}{r}\nil$
      si y sólo si $P\tranp{\checkmark}{r_1}\nil$,
      $Q\tranp{\checkmark}{r_2}\nil$, y $r=r_1\cdot r_2$,
      se sostiene lo siguiente:
      \begin{equation}
        \label{eq:paral1}
        \begin{split}
          \sum \lbag r\ |\ P\paral Q\tranp{\checkmark}{r}\nil \rbag =\\
          \sum \lbag r\ |\ P\tranp{\checkmark}{r_1}\nil\y Q\tranp{\checkmark}{r_2}\nil\y r=r_1\cdot r_2 \rbag=\\
          \sum \lbag  r_1\cdot r_2 \ |\ P\tranp{\checkmark}{r_1}\nil\y Q\tranp{\checkmark}{r_2}\nil \rbag= \\
          \sum \scaleleftright{\lbag}{r_1 \cdot \left(\sum \lbag r_2\ |\ Q\tranp{\checkmark}{r_2}\nil \rbag\right)\ \vstretch{2}{|}\  P\tranp{\checkmark}{r_1}\nil}{\rbag} = \\
          \sum \lbag  r_1 \cdot q |\  P\tranp{\checkmark}{r_1}\nil\rbag = q\cdot \sum \lbag  r_1  |\  P\tranp{\checkmark}{r_1}\nil\rbag = q\cdot p
        \end{split}
      \end{equation}
    \item[$|s|+|s'|>0$] Supóngase que  $|s|>0$ (para el caso cuando $|s'|>0$
      es simétrico). Consideremos que $s=\feature{A}\cdot s_1$.
      Ahora consideremos el multiconjunto 
      $\calP=\lbag P'\ |\ P\tranp{\f{A}}{r'} P'\vtranp{s_1\checkmark}{p'}\nil\rbag$ y dado $r_1 = \sum\lbag r\ |\ P\tranp{\f{A}}{r} P'\y P'\in\calP\rbag$ y
      $p_1=\sum\lbag r\ |\ P'\vtranp{s_1\checkmark}{r}\nil\y P'\in\calP\rbag$. Es facil verificar que $p=r_1\cdot p_1$:
      \begin{equation}
        \label{eq:paral2}
        \begin{split}
          p = \sum\lbag  r\ |\ P\vtranp{s\checkmark}{r}\nil \rbag =  \\
          \sum\lbag  r'\cdot p'\ |\  P\tranp{\f{A}}{r'} P'\vtranp{s_1\checkmark}{p'}\nil \rbag =\\
          \sum\scaleleftright{\lbag}{r'\cdot\underbrace{\left(\sum\lbag p'\ |\ P'\vtranp{s_1\checkmark}{p'}\nil\rbag\right)}_{p_1}\ \vstretch{2}{|}\  P\tranp{\f{A}}{r'} P'\y P'\in\calP}{\rbag} =\\
          p_1\cdot\sum\lbag  r'\ |\  P\tranp{\f{A}}{r'} P'\y P'\in\calP \rbag = r_1\cdot p_1
        \end{split}
      \end{equation}
      Para cualquier $P'\in\calP$, y luego por la hipótesis de inducción obtenemos
      $$
      p'\cdot q =\sum\lbag r\ |\ P'\paral Q\vtranp{s''\checkmark}{r}\y s''\in\inter(s_1,s')\rbag
      $$
      Por lo tanto
      \begin{equation}
        \label{eq:paral3}
        \begin{split}
          \sum\lbag r\ |\ P'\in\calP \y P'\paral Q\vtranp{s''\checkmark}{r}\y s''\in\inter(s_1,s')\rbag = \\
          \sum\lbag p'\cdot q\ |\  P'\in\calP\y P'\vtranp{s_1\checkmark}{p'}\nil\y s''\in\inter(s_1,s')\rbag\\
          q\cdot \sum\lbag r\ |\ P'\vtranp{s_1\checkmark}\nil\y P'\in\calP\rbag=p_1\cdot q
        \end{split}
      \end{equation}
      Entonces para cualquier $P'\in\calP$ se obtiene $P\paral Q\vtranp{s''\checkmark}{r}\nil$
      donde $P\tranp{\f{A}}{r'}P'$, $r=\frac{r'}{2}\cdot p'\cdot q$ y $s''$
      es la alternación de $s$ y $s'$. Por lo tanto
      \begin{equation}
        \label{eq:paral4}
        \begin{split}
          \sum\lbag  r\ |\ P\paral Q\tranp{\f{A}}{\frac{r_1}{2}}P'\paral Q\vtranp{s''\checkmark}{r_2}\nil\y P'\tranp{\f{A}}{r_1}P' \y \\ r=\frac{r_1}{2}\cdot r_1\y s''\in\inter(s_1,s') \rbag = \\
          \sum\scaleleftright{\lbag}{\frac{r'}{2}\cdot r'' \  \vstretch{2}{|}\ P\tranp{\f{A}}{r'}P'\y P'\paral Q\vtranp{s''\checkmark}{r''}\nil\y s''\in\inter(s_1,s') }{\rbag} = \\
          \sum\scaleleftright{\lbag}{  \frac{r'}{2}\cdot\left(\underbrace{\sum \scaleleftright{\lbag}{ r''\ \vstretch{4}{|}\
                  \begin{split}
                    P'\paral Q\vtranp{s''\checkmark}{r''}\nil\y \\s''\in\inter(s_1,s')
                  \end{split}}{\rbag}}_{\mathrm{hip\acute{o}tesis\ de\ inducci\acute{o}n:}\ p_1\cdot q} \right) \ \vstretch{4}{|}\ 
            P\tranp{\f{A}}{r'}P'\y P'\in\calP  }{\rbag} = \\
          \sum\scaleleftright{\lbag}{  \frac{r'}{2}\cdot (p_1\cdot q) \  \vstretch{2}{|}\ P\tranp{\f{A}}{r'}P' \y P'\in\calP   }{\rbag} = \\
          \frac{(p_1\cdot q)}{2}\cdot  \sum\lbag  r'\ |\ P\tranp{\f{A}}{r'}P' \y P'\in\calP  \rbag = \\
          \frac{1}{2}(p_1\cdot q\cdot r_1)=\frac{1}{2}(p\cdot q)
        \end{split}
      \end{equation}
      La ecuación \ref{eq:paral4} agrupa todas las probabilidades de alternar $s$ y $s'$ donde
      la primera acción de $s$ se dá en primer lugar. Luego se tiene que agrupar la información probabilística
      cuando la primera acción de $s'$ ocurre en primer lugar.
       
      Existen dos posibilidades, si esas acciones existen o no.
      Dependiendo de $s'$ se obtienen las siguientes posibilidades.
      \begin{description}
      \item[$|s'|=0$] 
        En este case tenemos $q=\sum\lbag q'\ |\ Q\tranp{\checkmark}{q'}\nil \rbag$,
        Si $Q\tranp{\checkmark}{q'}\nil$, al aplicar
        la regla $\nombreRegla{con4}$ y así obtener
        $P\paral Q\tranp{\f{A}}{\frac{r'\cdot q'}{2}} P_1$. 
        En este caso se obtiene
        \begin{equation}
          \label{eq:paral5}
          \begin{split}
            \sum\lbag  r\ |\  P\paral Q\tranp{\f{A}}{\frac{q'\cdot r'}{2}} P'\vtranp{s_1\checkmark}{p'}\nil\y r=p'\cdot \frac{q'\cdot r'}{2}\rbag=\\
            \sum\lbag \frac{1}{2}(p'\cdot r'\cdot q' \ |\  P\paral Q\tranp{\f{A}}{\frac{q'\cdot r'}{2}} P'\vtranp{s_1\checkmark}{p'}\nil \rbag=\\
            \sum\lbag \frac{1}{2}(p'\cdot r'\cdot (\underbrace{\sum\lbag q'\ |\ Q\tranp{\checkmark}{q'}\nil \rbag}_{q}) \ |\  P\tranp{\f{A}}{r'}P'\y P'\vtranp{s_1\checkmark}{p'}\nil \rbag=\\
            \frac{q}{2}\cdot \sum\lbag r'\cdot (\underbrace{\sum\lbag p'\ |\ P'\vtranp{s_1\checkmark}{p'}\nil \rbag}_{p_1}) \ |\  P\tranp{\f{A}}{r'}P'\y P'\in\calP \rbag=\\
            \frac{1}{2}(q\cdot p_1)\cdot(\underbrace{\sum\lbag r'\ |\ P\tranp{\f{A}}{r'}P'\y P'\in\calP \rbag}_{r_1})=\frac{1}{2}(q\cdot p_1\cdot r_1)=\frac{1}{2}(q\cdot p)
          \end{split}
        \end{equation}
        La única transición de $P\paral Q$ que permite alternar $s$ y $s'$ son de la forma
        \begin{enumerate}
        \item $P\paral Q\tranp{\f{A}}{\frac{r'}{2}} P'\paral Q\vtranp{s''\checkmark}{p'}\nil $ como se indica arriba, o
        \item $P\paral Q\tranp{\f{A}}{\frac{q'\cdot r'}{2}} P'\vtranp{s''\checkmark}{p'}\nil $.
        \end{enumerate}
        Por lo tanto
        \begin{equation}
          \begin{split}
            \label{eq:paral5}
            \sum\lbag  r\ |\  P\paral Q\vtran{s\checkmark}\nil\rbag =\\
            \sum\lbag  r\ |\ P\paral Q\tranp{\f{A}}{\frac{r_1}{2}}P'\paral Q\vtranp{s''\checkmark}{r_2}\nil\y P'\tranp{\f{A}}{r_1}P' \y \\ r=\frac{r_1}{2}\cdot r_1\y s''\in\inter(s_1,s') \rbag +\\
            \sum\lbag  r\ |\  P\paral Q\tranp{\f{A}}{\frac{q'\cdot r'}{2}} P'\vtranp{s_1\checkmark}{p'}\nil\y r=p'\cdot \frac{q'\cdot r'}{2}\rbag=\\
            \frac{1}{2}(p\cdot q) + \frac{1}{2}(p\cdot q)= p\cdot q
          \end{split}
        \end{equation}
       \item[$|s'|>0$] Consideremos $s'=\f{B}\cdot s_2$. 
        En este caso se consideran las alternaciones donde
        $Q$ produce la característica $\f{B}$ en primer lugar, lo que es,
        que primero se aplica la regla~\nombreRegla{con2}. Considerando el
        multiconjunto $\calQ=\lbag Q'\ |\ Q\tranp{\f{B}}{r}Q'\vtranp{s_2\checkmark}{q'}\nil]$, y dado
        $r_2=\sum\lbag r\ |\ Q\tranp{\f{B}}{r}Q'\y Q'\in\calQ\rbag$ y $q_1=\sum\lbag r\ |\ Q'\vtranp{s_2\checkmark}{r}\nil\y Q'\in\calQ\rbag$.
        Con un razonamiento similar al de las ecuaciones~\eqref{eq:paral1} y~\eqref{eq:paral2} se obtiene que $q=r_2\cdot q_1$ y
        \begin{equation}
          \label{eq:paral6}
          \begin{split}
            \sum\lbag  r\ |\ P\paral Q\tranp{\f{B}}{\frac{r_1}{2}}P\paral Q'\vtranp{s''\checkmark}{r_2}\nil\y\\ r=\frac{r_1}{2}\cdot r_1\y s''\in\inter(s,s_2) \rbag = \frac{1}{2}(p\cdot q)
          \end{split}
        \end{equation}
        La única transición de $P\paral Q$ que implique alternar $s$ y $s'$ son de la forma:
        \begin{enumerate}
        \item $P\paral Q\tranp{\f{A}}{p'}{\frac{r}{2}} P'\paral Q\vtranp{s''\checkmark}{p'}\nil $ donde $s''$ es una alternación de $s_1$ y $s'$, o
        \item $P\paral Q\tranp{\f{B}}{q'}{\frac{r}{2}} P\paral Q'\vtranp{s''\checkmark}{p'}\nil $ donde $s''$ es una alternación de $s$ y $s_2$, 
        \end{enumerate}        
        Por lo tanto,
        \begin{equation}
          \label{eq:paral7}
          \begin{split}
            \sum\lbag  r\ |\  P\paral Q\vtran{s\checkmark}\nil\rbag =\\
            \sum\lbag  r\ |\ P\paral Q\tranp{\f{A}}{\frac{r_1}{2}}P'\paral Q\vtranp{s''\checkmark}{r_2}\nil\y P'\tranp{\f{A}}{r_1}P' \y\\ r=\frac{r_1}{2}\cdot r_1\y s''\in\inter(s_1,s') \rbag + \\
            \sum\lbag  r\ |\ P\paral Q\tranp{\f{B}}{\frac{r_1}{2}}P\paral Q'\vtranp{s''\checkmark}{r_2}\nil\y Q'\tranp{\f{B}}{r_1}Q' \y\\ r=\frac{r_1}{2}\cdot r_1\y s''\in\inter(s,s_2) \rbag = \\
            \frac{1}{2}(p\cdot q) + \frac{1}{2}(p\cdot q)= p\cdot q
          \end{split}
        \end{equation}
      \end{description}
    \end{description}
\elem

\blem\label{prop:mand}
  Dado el término  $P\in\fodaPAp$ y la característica $\feature{A}\in\calF$, $P\vtranp{s\checkmark}{p}\nil$.
  \begin{enumerate}
  \item $\f{A}\in s$ si y sólo si $\mandatory{A}{P}\vtranp{s\checkmark}{p}\nil$.
  \item $\f{A}\not\in s$ si y sólo si $\mandatory{A}{P}\vtranp{s\f{A}\checkmark}{p}\nil$.
  \end{enumerate}
  \textit{Demostración.}
    En ambos casos la demostración se realiza por inducción de la longitud de $|s|$.
    \lcomen{I think it is trivial, is it necessary to write it in detail.}
\elem


\blem\label{lem:forb}
  Dado el término  $P\in\fodaPAp$, la característica $\feature{A}\in\calF$, la traza $s\in\calF^*$ y la probabilidad
  $p\in(0,1)$. $P\vtranp{s\checkmark}{p}\nil$, si y sólo si
  $\forbid{P}{A}\vtranp{s\checkmark}{p}\nil$ y $\f{A}\not\in s$.
  \begin{proof}
    La demostración es simple por inducción de la longitud de $s$.
  \end{proof}
\elem


\blem\label{lem:req}
  Dado el término  $P\in\fodaPAp$, las características $\feature{A},\feature{B}\in\calF$, la traza $s\in\calF^*$
  y la probabilidad $p\in(0,1)$. Entonces $P\vtranp{s\checkmark}{p}\nil$ si y sólo si
  $\require{A}{B}{P}\vtranp{s'\checkmark}p\nil$ y 
  $s'$ y tiene alguna de las siguientes formas:
  $\f{A}\not\in s$ y $s'=s$, $\f{B}\in s$ y $s'=s$, o
  $\f{A}\in s$, $\f{B}\not\in s$ y $s'=s\cdot\f{B}$.
  \begin{proof}
    Se realiza por inducción de la longitud de $s$.
    \begin{description}
    \item[$|s|=0$] En este caso $P\tranp{\checkmark}{p}\nil$.  Se
      obtiene el resultado aplicando la regla~\nombreRegla{req3}.
    \item[$|s|>0$] Ahora se pueden distinguir tres casos dependiendo
      de la primera característica de $s$:
      \begin{description}
      \item[$s=\f{A}s_1$.] En este caso existe $p_1,q\in(0,1)$ tal que
        $P\tranp{\f{A}}{p_1}P_1\vtranp{s_1\checkmark}{q}\nil$.  Al
        aplicar la regla~\nombreRegla{req2} se obtiene
        $\require{A}{B}{P}\tranp{\f{A}}{p_1}\mandatory{B}{P_1}$.
        Obtenemos el resultado al aplicar la
        proposición~\ref{prop:mand}.
      \item[$s=\f{B}s_1$.] En este caso existe $p_1,q\in(0,1)$ tal que
        $P\tranp{\f{A}}{p_1}P_1\vtranp{s_1\checkmark}{q}\nil$.  Al
        aplicar la regla~\nombreRegla{req2} se obtiene
        $\require{A}{B}{P}\tran{\f{B}}{p_1}\mandatory{A}{P_1}$.  Se
        obtiene el resultado al aplicar la
        proposición~\ref{prop:mand}.
      \item[$s=\f{C}s_1$ con $\f{C}\neq\f{A}$ y $\f{C}\neq\f{A}$.]  En
        este caso existe $p_1,q\in(0,1)$ tal que
        $P\tranp{\f{C}}{p_1}P_1\vtranp{s_1\checkmark}{q}\nil$.  Al
        aplicar la regla~\nombreRegla{req1}, se obtiene
        $ \require{A}{B}{P}\tranp{\f{C}}{p_1}\require{A}{B}{P_1} $, y
        luego el resultado al aplicar la hipótesis de inducción sobre
        $s_1$.
      \end{description}
    \end{description}
  \end{proof}
\elem


\blem\label{lem:excl}
  Dado el término $P\in\fodaPAp$, las características $\feature{A},\feature{B}\in\calF$, la traza $s\in\calF^*$ y
  la probabilidad $p\in(0,1)$. Entonces $P\vtran{s\checkmark}{p}\nil$ si y sólo si
  $\exclude{A}{B}{P}\vtranp{s\checkmark}{p}\nil$, $\f{A}\not\in s$ y $\f{B}\not\in s$.
  \textit{Demostración.}
    Se realiza la demostración por la inducción de la longitud de $s$.
    \begin{description}
    \item[$|s|=0$] En este caso $P\tranp{\checkmark}{p}\nil$. 
      Se obtiene el resultado al aplicar la regla~\nombreRegla{excl4}.
    \item[$|s|>0$] Ahora pueden distinguirse tres casos dependiendo de la
    primera característica de $s$:
      \begin{description}
      \item[$s=\f{A}s_1$.] En este caso existe $p_1,q\in(0,1)$
        tal que $P\tranp{\f{A}}{p_1}P_1\vtranp{s_1\checkmark}{q}\nil$. 
        Al aplicar la regla~\nombreRegla{req2}
        se obtiene $\require{A}{B}{P}\tranp{\f{A}}{p_1}\forbid{B}{P_1}$.
        Ahora por la proposición~\ref{prop:mand1}, 
        \begin{itemize}
        \item $\f{B}\in s_1$ si y sólo si $\mandatory{B}{P_1}\vtranp{s_1\cdot \checkmark}{q}\nil$.
        \item $\f{B}\not\in s_1$ si y sólo si  $\mandatory{B}{P_1}\vtranp{s_1\f{B}\checkmark}{q}\nil$.
        \end{itemize}
      \item[$s=\f{C}s_1$ con $\f{C}\neq\f{A}$.]
        En este caso es $p_1,q\in(0,1)$
        tal que $P\tranp{\f{C}}{p_1}P_1\vtranp{s_1\checkmark}{q}\nil$.
        Al aplicar la regla~\nombreRegla{req1},  se obtiene
        $
        \require{A}{B}{P}\tranp{\f{C}}{p_1}\require{A}{B}{P_1}
        $, y luego el resultado se obtiene al aplicar la  hipótesis de inducción sobre $s_1$.
      \end{description}
    \end{description}
\elem


%%% Local Variables:
%%% mode: latex
%%% TeX-master: "main"
%%% End:

