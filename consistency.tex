%\subsection{Consistence of the probabilistic model}
%\label{sec:stat:consis}

\mncomment{Comprobad que lo que digo en este parrafo es lo que
  queriais decir porque dais pocas pistas!}
\lcomen{es correcto}

\mncomment{No veo la utilidad de esto pero a lo mejor lo haceis porque es tipico en SPL... ni idea... pero yo lo quitaba sin piedad...}
\lcomen{Nos lo dijeron bastante en el primer artículo}

Finally we prove an important property of our probabilistic language: its consistency. We say that a non-probabilistic SPL model is \emph{consistent} if it has products~\cite{acl13}.  In our case, we can define consistency by having $\ham(P)>0$. We will prove that a translation from our probabilistic framework into the non-probabilistic one keeps consistency in the expected way.

\bdfn
  We define the translation function $\np:\fodaPAp\mapsto \fodaPA$ as follows:
  \begin{displaymath}
     \np(P)=
     \begin{cases}
       \checkmark & \mathrm{if}\ P=\checkmark\\
       \feature{A};\np(P) & \mathrm{if}\ P=\feature{A};P\\
       \ofeature{A};\np(P) & \mathrm{if}\ P=\feature{A};_pP\\
       \np(P) \choice \np(Q) & \mathrm{if}\ P \choice_{p} Q \\
       \np(P) \paral \np(Q) & \mathrm{if}\ P \paral Q \\
       \np(P) \paral \np(Q) & \mathrm{if}\ P \paral Q \\
       \require{A}{B}{\np(P)}  & \mathrm{if}\ \require{A}{B}{P}\\
       \exclude{A}{B}{\np(P)} & \mathrm{if}\ \exclude{A}{B}{P}\\
       \mandatory{A}{\np(P)} \paral & \mathrm{if}\ \mandatory{A}{P}\\
       \forbid{A}{\np(P)} \paral & \mathrm{if}\ \forbid{A}{P}\\
       P & \mathrm{otherwise}
     \end{cases}
  \end{displaymath}
\edfn

The proof of the following result is straightforward by taking into account that our operational semantics rules are the same, if we discard probabilities, as in~\cite{acl13}. Therefore, any sequence of transitions
derived in the probabilistic model can be also derived in the non probabilistic model~\ref{lem:pr:sum1}. In addition, by Lemma~\ref{lem:pr:sum2} \mncomment{Que lema es este?} we know that any derived trace in the probabilistic model has a positive probability.

\bthm\label{thm:relnonprob}
  Let $P,Q\in\fodaPAp$. We have 
 $P\vtranp{s}{p}Q$ if and only if $\np(P)\vtran{s}\np(Q)$.
  Moreover, we have $pr\in\prod(\np(P))$ if and only if there exists
    $p>0$ such that $(pr,p)\in\prodp(P)$.
  % \item\label{prop:relnonprob-c} $pr\in\semden{\np(P)}$ si y sólo si existe $p>0$ tal que $(pr,p)\in\semdenp{P}$.
  % \item\label{prop:relnonprob-d} Existe $p\in(0,1)$ tal que $(pr,p)\in\prodp(P)$ si y sólo si existe $q\in(0,1)$ tal que $(pr,q)\in\semdenp{P}$.
 \ethm


%%% Local Variables:
%%% mode: latex
%%% TeX-master: "main"
%%% ispell-local-dictionary: "english"
%%% End:
