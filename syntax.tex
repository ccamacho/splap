% $Id$

\section{Sintaxis}
\label{sec:stat:sintax}

Para el modelo probabilístico, el conjunto de características será representado como $\calF$,
las características aisladas se mostrarán como $\feature{A}, \feature{B}, \feature{C}\dots$,
los términos del modelo probabilístico serán descritos como $P,Q$ y
las probabilidades $p,q$ son expresadas dentro del rango $(0,1)$.

\bdfn
\label{sec:stat:sintax:dfn}
Una \emph{línea de productos probabilística} es un término generado por la siguiente
expresion EBNF:
$$
\begin{array}{ll}
P::=& \checkmark \barra \nil \barra \feature{A};P \barra
\ofeature{A};_{p} P \barra P \choice_{p} Q \barra P \paral Q \barra
\\ 
& \exclude{A}{B}{P}\barra  \require{A}{B}{P}\barra  \forbid{A}{P}\barra  \mandatory{A}{P}\\ 
\end{array}     
$$
\noindent 
donde $\feature{A} \in \calF$ y $p\in(0,1)$. Los términos del álgebra
serán representados dentro del modelo $\fodaPAp$.
\edfn

Como puede observarse, la descripción de la sintaxis
para la extensión probabilística (ver def. \ref{sec:stat:sintax:dfn})
parte de la definición original de \fodaPA\ (ver def. \ref{dfn:syntax}).
Sin embargo los elementos sintácticos deben ser modificados tal que
permitan almacenar información relativa a las probabilidades.
%
Los operadores de la sintaxis que han sido extendidos para soportar
probabilidades son $P \choice_{p} Q$ y $\ofeature{A};_{p} P$
por dos razones.
%
La primera razón es que estos dos operadores son los operadores mínimos
a los que un término \fodaPA\ puede reducirse (formas normales y pre-normales) y
cualquier otro término podría ser representado utilizando estos dos operadores. 
La segunda razón es porque estos operadores son los operadores que representan
la selección exclusiva o de adición opcional de características dentro de
los productos en construcción, afectando
así a la cardinalidad de las características dentro de los productos finales.



%\todo{Hay que explicar los operadores probabilísticos $P \choice_{p}
%	Q$ y $\ofeature{A};_{p} P$ y la razón por la que el resto no
%	necesita probabilidades.}

%%% Local Variables: 
%%% mode: latex
%%% TeX-master: "../../main"
%%% End: 
