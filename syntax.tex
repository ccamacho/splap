\section{Sintaxis}
\label{sec:stat:sintax}
As in previous previous works~\cite{acl13,cln16} we are considering a
set of features denoted by $\calF$.  $\feature{A}, \feature{B},
\feature{C}\dots\in\calF$. We have an special feature
$\checkmark\not\in\calF$ to mark the end  of a product. The syntax we consider is similar to 
\fodaPA. The probabilities are introduce in the choice operator $P \choice_{p} Q $ and in
the optional feature operator $\ofeature{A};_{p} P$. We do not allow
degenerated probabilities in the syntax: $p\neq 0$ and $p\neq 1$.

\bdfn
\label{sec:stat:sintax:dfn}
A \emph{probabilitistc línea SPL} is a term generated by the following
BNF expression:
$$
\begin{array}{ll}
P::=& \checkmark \barra \nil \barra \feature{A};P \barra
\ofeature{A};_{p} P \barra P \choice_{p} Q \barra P \paral Q \barra
\\ 
& \exclude{A}{B}{P}\barra  \require{A}{B}{P}\barra  \forbid{A}{P}\barra  \mandatory{A}{P}\\ 
\end{array}     
$$
\noindent 
where $\feature{A} \in \calF$ and $p\in(0,1)$. The terms of the
algebra
will be denotated by  $\fodaPAp$.
\edfn




%\todo{Hay que explicar los operadores probabilísticos $P \choice_{p}
%       Q$ y $\ofeature{A};_{p} P$ y la razón por la que el resto no
%       necesita probabilidades.}

%%% Local Variables: 
%%% mode: latex
%%% TeX-master: "main"
%%% End: 
