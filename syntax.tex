\section{$\fodaPAp$: Syntax and Semantics}
\label{sec:stat:sintax}
In this section we introduce our language. In addition to provide its syntax, we define an operational semantics and a denotational semantics. In the next section we will show the equivalence between these two semantic frameworks.

\subsection{Syntax and operational semantics}
\label{sec:stat:sintax}
Following our previous work~\cite{acl13,cln16}, we will consider a
set of features. We denote this set by $\calF$ and consider that $\feature{A}, \feature{B},
\feature{C}$ range over $\calF$. We have a special feature
$\checkmark\not\in\calF$ to mark the end  of a product. The syntax we consider is similar to
\fodaPA. The probabilities are introduced in the choice operator $P \choice_{p} Q $ and in
the optional feature operator $\ofeature{A};_{p} P$. We do not allow
\emph{degenerated} probabilities, that is, for all probability $p$ we have $0< p<1$.

\bdfn
\label{sec:stat:sintax:dfn}
A \emph{probabilitistc SPL} is a term generated by the following
BNF expression:
$$
\begin{array}{ll}
P::=& \checkmark \barra \nil \barra \feature{A};P \barra
\ofeature{A};_{p} P \barra P \choice_{p} Q \barra P \paral Q \barra
\\
& \exclude{A}{B}{P}\barra  \require{A}{B}{P}\barra  \forbid{A}{P}\barra  \mandatory{A}{P}\\
\end{array}
$$
\noindent
where $\feature{A} \in \calF$ and $p\in(0,1)$. The set of terms of the
algebra will be denoted by  $\fodaPAp$.
\edfn
\mncomment{Por que no se cuantifica el $P \paral Q$? El  resultado es
  que hay que poner un cutre $\frac12$ en la semantica operacional
  que, ademas, se justifica regular. Charlote anyadido, incluyendo
  cita a D'Argenio que esta en el PC.}
\lcomen{No lo cuantifico porque no se darle un significado
  ``intuitivo'' a esa probabilidad}

\mncomment{Creo que $P \paral \nil$ no es equivalente a $P$. De hecho, $P \paral \nil$ es equivalente a $\nil$, no? Esto es lo que quereis que sea? Hay un problema similar con $P \choice_{p} \nil$.}
\lcomen{Es cierto que $P\paral\ni$ es equivalente a $\nil$. $P\choice
  \nil$ es equivalente a $P$. Algo parecido con probabilidades, pero
  hay en este caso hay un $\waste$.}
%\todo{Hay que explicar los operadores probabilísticos $P \choice_{p}
%       Q$ y $\ofeature{A};_{p} P$ y la razón por la que el resto no
%       necesita probabilidades.}

%%% Local Variables:
%%% mode: latex
%%% TeX-master: "main"
%%% End:
