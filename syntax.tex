% $Id$

\section{Sintaxis}

Para el modelo probabilístico, el conjunto de características será representado como $\calF$,
las características aisladas se mostrarán como $\feature{A}, \feature{B}, \feature{C}\dots$,
los términos del modelo probabilístico serán descritos como $P,Q$ y
las probabilidades $p,q$ son expresadas dentro del rango $(0,1)$.
\todo{Hay que explicar los operadores probabilísticos $P \choice_{p}
  Q$ y $\ofeature{A};_{p} P$ y la razón por la que el resto no
  necesita probabilidades.}
\bdfn
Una \emph{línea de productos probabilística} es un término generado por la siguiente
expresion EBNF:
$$
\begin{array}{ll}
  P::=& \checkmark \barra \nil \barra \feature{A};P \barra
    \ofeature{A};_{p} P \barra P \choice_{p} Q \barra P \paral Q \barra
  \\ 
  & \exclude{A}{B}{P}\barra  \require{A}{B}{P}\barra  \forbid{A}{P}\barra  \mandatory{A}{P}\\ 
\end{array}     
$$
\noindent 
donde $\feature{A} \in \calF$ y $p\in(0,1)$. Los términos del álgebra
serán representados dentro del modelo $\fodaPAp$.
\edfn


%%% Local Variables: 
%%% mode: latex
%%% TeX-master: "../../main"
%%% End: 
