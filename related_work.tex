% $Id: introduction.tex,v 1.8 2013/12/03 09:17:27 ccamacho Exp $
\section{Related work}
\label{ref:related_work}

The study of probabilistic extensions of formal methods can be dated
back to the end of the 1980s. This is already a well established area,
with many extensive contributions to include probabilistic information
in classical formalisms (I/O Automata, Finite State Machines,
(co-)algebraic approaches, among
others)~\cite{hm09,aprs11,sok11,hn12,dghm14,agl16,Dubslaff2015}.
%others)~\cite{ls91,rgs95,cdsy99,nun03,lnr06,csv07,dp07,dghm08,hm09,aprs11,sok11,hn12,dghm14,agl16,Dubslaff2015}.
%
%
%Figures like the displayed trees describe how probabilities are assigned in the system behavior (see Figure~\ref{fig:displayedTrees}). These models allow to assign a real value between 0 and 1 to each transition in order to execute an event. It is important to notice that the probability of execute a transition \textit{t} is not necessarily equal to the other probabilities from the same level.
%\begin{figure}[h]
%       \hspace{50px}
%       \begin{minipage}{0.4\hsize}
%
%               \begin{minipage}{0.2\hsize}
%                       \scalebox{0.6}{
%                               \begin{tikzpicture}[>=stealth',shorten >=1pt,auto,node distance=2cm, semithick]
%
%                               \node[circle,draw] (A)   {};
%                               \node[circle,draw] (B) [below left of=A]   {};
%                               \node[regular polygon,regular polygon sides=3,draw,scale=0.8] (F) [below of=B]   {T};
%                               \node[circle,draw] (C) [below right of=A]   {};
%                               \node[circle,draw] (D) [below left of=C]   {};
%                               \node[circle,draw] (E) [below right of=C]   {};
%                               \node[regular polygon,regular polygon sides=3,draw,scale=0.8] (G) [below  of=D]   {T};
%                               \node[regular polygon,regular polygon sides=3,draw,scale=0.8] (H) [below  of=E]   {T};
%
%                               \path (A) edge [] node [pos=0.5, above left, draw=white, opacity=0, text opacity=1]{1/3} (B);
%                               \path (B) edge [] node [pos=0.5, left, draw=white, opacity=0, text opacity=1]{a} (F);
%                               \path (A) edge [] node [pos=0.5, above right, draw=white, opacity=0, text opacity=1]{1/2} (C);
%                               \path (C) edge [] node [pos=0.5, above left, draw=white, opacity=0, text opacity=1]{1/2} (D);
%                               \path (C) edge [] node [pos=0.5, above right, draw=white, opacity=0, text opacity=1]{1/2} (E);
%                               \path (D) edge [] node [pos=0.5, right, draw=white, opacity=0, text opacity=1]{b} (G);
%                               \path (E) edge [] node [pos=0.5, left, draw=white, opacity=0, text opacity=1]{c} (H);
%                               \end{tikzpicture}
%                       }
%               \end{minipage}
%       \end{minipage}
%       \begin{minipage}{0.4\hsize}
%
%               \begin{minipage}{0.2\hsize}
%                       \scalebox{0.6}{
%                               \begin{tikzpicture}[>=stealth',shorten >=1pt,auto,node distance=4cm, semithick]
%                               \node[circle,draw] (A)   {};
%                               \node[regular polygon,regular polygon sides=3,draw,scale=0.8] (B) [below left of=A]   {T};
%                               \node[regular polygon,regular polygon sides=3,draw,scale=0.8] (C) [below  of=A]   {T};
%                               \node[regular polygon,regular polygon sides=3,draw,scale=0.8] (D) [below right of=A]   {T};
%
%                               \path (A) edge [] node [pos=0.5, above left, draw=white, opacity=0, text opacity=1]{1/3} (B);
%                               \path (A) edge [] node [pos=0.5, left, draw=white, opacity=0, text opacity=1]{1/3} (C);
%                               \path (A) edge [] node [pos=0.5, above right, draw=white, opacity=0, text opacity=1]{1/3} (D);
%
%                               \end{tikzpicture}}
%               \end{minipage}
%
%       \end{minipage}
%
%       \caption{Displayed trees representing probabilities.}
%       \label{fig:displayedTrees}
%
%\end{figure}
%
%\acomen{He puesto título y referencia a esta figura (Fig. 2). Please, check!}
%
%\acomen{Carlos, he cambiado el siguiente texto con referencias a papers de probabilidades. Please, check para ver que no digo nada cantoso)}
%%
Although the addition of probabilistic information to model \SPLs\ is relatively new, different proposals can be found in the current literature~\cite{chssgl13,tllv15,tlll15,dpcslsh17}. In particular, a very recent
work shows that statistic analysis allows users to determine relevant characteristics, like the certainty of finding valid products among complex models~\cite{dpcslsh17}.
%
Another approach focuses on testing properties of \SPLs, like reliability, by defining three verification techniques: a probabilistic model checker on each product, on a model range, and testing the behavior relations with other models~\cite{chssgl13}.
%
Some of these approaches describe models to run statistical analysis over \SPLs, where pre-defined syntactic elements are computed by applying a specific set of operational rules~\cite{tllv15,tlll15}. These models demonstrate their ability to be integrated into standard tools, like QFLan~\cite{tlll15}, Microsoft's SMT Z3~\cite{ln08} and MultiVeStA~\cite{sa13}.
%
%Some of these works describe models to run statistical analysis to represent \SPLs\ by defining an specific set of operational rules to compute their defined syntax elements~\cite{tllv15,tlll15}. These models demonstrate their ability by being integrated to standard tools like QFLan, Microsoft's SMT Z3 and MultiVeStA.
%

Other works focus on describing use cases for analyzing the probability of finding features inside valid products~\cite{dpcslsh17}.
It is true that variability models computing can create combinatory problems
depending on how the models are computed and how the models are represented, which is directly
correlated to the information to be generated~\cite{dpcslsh17}.
This analysis makes the process of studying product lines a complex computational  task.

An interesting aspect of \fodaPAp is that
any of the research articles in the literature manage
to describe in their work the use of multisets.
Also, they do not explicitly work on
the translation of \FODA\ to represent probabilities and they do
not introduce the notion of hiding those not needed features to
calculate the probability of a specific feature.
%
%
%
%
%


There are proposals that allow the introduction of probabilities in
feature models. For instance \cite{Dubslaff2015} uses Markow Decision
processes to represent the behavior of products. Whereas in
\cite{vk13}, the behavior of the system.
The variability on those formalisms is modeled existing tools like FODA or
just propositional logic to describe their products.
On the contrary, our
contributions focuses on defining a probabilistic language to describe
the products. We use the probabilities to quantify how relevant is a
product of a feature within the product line.

% In particular, there exist mathematical verification models that allow
% the representation of the products - of a product line - using models,
% like discrete time Markov chain families, for representing the
% probabilistic behavior of all the products in the product
% line~\cite{Dubslaff2015} and~\cite{vk13}, in this particular case
% representing them using operational semantics.
% %
% The main differences between both~\cite{Dubslaff2015} and~\cite{vk13}
% with respect to the current probabilistic extension is
% that~\cite{Dubslaff2015} and~\cite{vk13} are not basing their
% definitions of implementations in providing a literal translation
% between the existing variability models, like FODA, and, therefore,
% these do not provide implementations over their practical uses.

% However, these approaches present analysis techniques that should have
% a significant impact on the effort to compute variability models. In
% specific for~\cite{vk13} when comparing the way probabilities are
% computed, for each Markov process, the sum of the probabilities of the
% transitions when doing a transition from an state to another state
% must always sum 1.
% %
% In the case of the probabilistic extension presented in this research
% article, in particular when the operational semantics of the
% probabilistic extension is defined, we need to take into account all
% the traces when we produce a valid product.  When all the traces are
% computed, if those traces represent the same set of features, the sum
% of all the computed probabilities for those traces can be up to 1 ,
% also~\cite{vk13} models probabilistic systems and not probabilistic
% software product lines.

% Also, the formal representation of~\cite{Dubslaff2015} is given in
% terms of Markov decision processes for representing the probabilistic
% information, which is not used particularly in this research paper.
% Instead of having sub-states with the probabilistic information we add
% the probabilistic data itself to the operational semantics definition.
% %
% The compositional framework defined in~\cite{Dubslaff2015} do not have
% a direct translation from well known graphical representations of
% feature models like \FODA\ making them to skip the representation of
% other relationships commonly represented, like the cross tree
% constraints for the inclusion or exclusion of specific features.
% %
% In the contrary they represent the parallel composition of the
% features and a the fact that each feature can have a different
% cardinality in the model, allowing that the same feature can be
% multiple times in the same product.



In previous years, the studies focusing the analysis of variability models  - and
their practical applications - with realistic use cases
have demonstrated that those uses cases do not describe such complex models~\cite{jhf11, jhfes12}. Thus, these
can be processed in the practice without much algorithmic sophistication or
complex analysis.
In particular, the study of expending machines
has been widely used across the whole
literature to show practical and real usages of products
line modeling~\cite{jhfes12}. Moreover, it is described that those models for defining products
lines does not always apply to the
formal definition and description of software
product lines, as they are not directly related~\cite{cds06, fub06, nnz14}.
Recent implementations, like ProFeat~\cite{Chrszon2018}, allow to
help in the verification of requirements for families of probabilistic systems. These implementations,
together with PRISM~\cite{mgd12}, use their own language and are based on Markov decision processes.





% 3) Thank you for explicitly devoting a section to related work.
% For estimating the contribution of this paper, I still miss a comparison
% to the approaches mentioned, especially to the ones that use operational
% semantics. What is the difference of your operational semantics to the ones
% in [25] (considering the Markov chain fragment) and [32]? The authors might
% furthermore explain the new sentence "state that state it is possible to
% describe a formal framework that translates the current feature models to
% probabilistic methods." as I could not grasp its meaning. Which
% "probabilistic methods" are meant and how in principle models can
% be translated to methods?


%Comparar nuestro approach con:

%C. Dubsla, C. Baier, and S. Kluppelholz, Probabilistic Model Checking for Feature-
%17 Oriented Systems, pp. 180{220. Springer Berlin Heidelberg, 2015

%y

%M. Varshosaz and R. Khosravi, \Discrete time markov chain families: Modeling
%42 and verication of probabilistic software product lines," pp. 34{41, 08 2013.




%%% Local Variables:
%%% mode: latex
%%% TeX-master: "main"
%%% End:
