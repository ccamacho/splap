% $Id: introduction.tex,v 1.8 2013/12/03 09:17:27 ccamacho Exp $
\section{Related work}
\label{ref:related_work}



The study of probabilistic extensions of formal methods can be dated
back to the end of the 1980s. This is already a well established area,
with many extensive contributions to include probabilistic information
in classical formalisms (I/O Automata, Finite State Machines,
(co-)algebraic approaches, among others)~\cite{ls91,rgs95,cdsy99,nun03,lnr06,csv07,dp07,dghm08,hm09,aprs11,sok11,hn12,dghm14,agl16,Dubslaff2015}.
%
%
%Figures like the displayed trees describe how probabilities are assigned in the system behavior (see Figure~\ref{fig:displayedTrees}). These models allow to assign a real value between 0 and 1 to each transition in order to execute an event. It is important to notice that the probability of execute a transition \textit{t} is not necessarily equal to the other probabilities from the same level.
%\begin{figure}[h]
%       \hspace{50px}
%       \begin{minipage}{0.4\hsize}
%
%               \begin{minipage}{0.2\hsize}
%                       \scalebox{0.6}{
%                               \begin{tikzpicture}[>=stealth',shorten >=1pt,auto,node distance=2cm, semithick]
%
%                               \node[circle,draw] (A)   {};
%                               \node[circle,draw] (B) [below left of=A]   {};
%                               \node[regular polygon,regular polygon sides=3,draw,scale=0.8] (F) [below of=B]   {T};
%                               \node[circle,draw] (C) [below right of=A]   {};
%                               \node[circle,draw] (D) [below left of=C]   {};
%                               \node[circle,draw] (E) [below right of=C]   {};
%                               \node[regular polygon,regular polygon sides=3,draw,scale=0.8] (G) [below  of=D]   {T};
%                               \node[regular polygon,regular polygon sides=3,draw,scale=0.8] (H) [below  of=E]   {T};
%
%                               \path (A) edge [] node [pos=0.5, above left, draw=white, opacity=0, text opacity=1]{1/3} (B);
%                               \path (B) edge [] node [pos=0.5, left, draw=white, opacity=0, text opacity=1]{a} (F);
%                               \path (A) edge [] node [pos=0.5, above right, draw=white, opacity=0, text opacity=1]{1/2} (C);
%                               \path (C) edge [] node [pos=0.5, above left, draw=white, opacity=0, text opacity=1]{1/2} (D);
%                               \path (C) edge [] node [pos=0.5, above right, draw=white, opacity=0, text opacity=1]{1/2} (E);
%                               \path (D) edge [] node [pos=0.5, right, draw=white, opacity=0, text opacity=1]{b} (G);
%                               \path (E) edge [] node [pos=0.5, left, draw=white, opacity=0, text opacity=1]{c} (H);
%                               \end{tikzpicture}
%                       }
%               \end{minipage}
%       \end{minipage}
%       \begin{minipage}{0.4\hsize}
%
%               \begin{minipage}{0.2\hsize}
%                       \scalebox{0.6}{
%                               \begin{tikzpicture}[>=stealth',shorten >=1pt,auto,node distance=4cm, semithick]
%                               \node[circle,draw] (A)   {};
%                               \node[regular polygon,regular polygon sides=3,draw,scale=0.8] (B) [below left of=A]   {T};
%                               \node[regular polygon,regular polygon sides=3,draw,scale=0.8] (C) [below  of=A]   {T};
%                               \node[regular polygon,regular polygon sides=3,draw,scale=0.8] (D) [below right of=A]   {T};
%
%                               \path (A) edge [] node [pos=0.5, above left, draw=white, opacity=0, text opacity=1]{1/3} (B);
%                               \path (A) edge [] node [pos=0.5, left, draw=white, opacity=0, text opacity=1]{1/3} (C);
%                               \path (A) edge [] node [pos=0.5, above right, draw=white, opacity=0, text opacity=1]{1/3} (D);
%
%                               \end{tikzpicture}}
%               \end{minipage}
%
%       \end{minipage}
%
%       \caption{Displayed trees representing probabilities.}
%       \label{fig:displayedTrees}
%
%\end{figure}
%
%\acomen{He puesto título y referencia a esta figura (Fig. 2). Please, check!}
%
%\acomen{Carlos, he cambiado el siguiente texto con referencias a papers de probabilidades. Please, check para ver que no digo nada cantoso)}
%%
Although the addition of probabilistic information to model \SPLs\ is relatively new, there are already several proposals in the literature~\cite{chssgl13,tllv15,tlll15,dpcslsh17}. In particular, very recent
work~\cite{dpcslsh17} shows that statistic analysis allows users to determine relevant characteristics, like the certainty of finding valid products among complex models.
%
Another approach~\cite{chssgl13} focuses on testing properties of \SPLs, like reliability, by defining three verification techniques: a probabilistic model checker on each product, on a model range, and testing the behavior relations with other models.
%
Some of these approaches describe models to run statistical analysis over \SPLs, where pre-defined syntactic elements are computed by applying a specific set of operational rules~\cite{tllv15,tlll15}. These models demonstrate their ability to be integrated into standard tools like QFLan~\cite{tlll15}, Microsoft's SMT Z3~\cite{ln08} and MultiVeStA~\cite{sa13}.
%
%Some of these works describe models to run statistical analysis to represent \SPLs\ by defining an specific set of operational rules to compute their defined syntax elements~\cite{tllv15,tlll15}. These models demonstrate their ability by being integrated to standard tools like QFLan, Microsoft's SMT Z3 and MultiVeStA.
%
%Other works focus on describing use cases for analyzing the probability of finding features inside valid products~\cite{dpcslsh17}. %An interesting feature is that any of the referenced research articles describe their approaches using multisets.

Es cierto que el procfesamiento de los modelos de
variabilidad puede crear problemas combinatorios~\cite{dpcslsh17}
y hacer que el analisis de los mismos se combierta en una tarea computacionalmente compleja.
En particular existen modelos de verificacion matematica~\cite{vk13}
que permiten representar el comportamiento de los productos en lineas de productos
haciendo uso de modelos como el Discrete Time Markov Chain Family, que permite

represents the probabilistic behavior of all
the products in the product line pero cuya implementacion
no define una traduccion literal entre modelos de variabilidad conocidos
como FODA y tampoco proveen de implementaciones sobre su
uso practico, sin embargo, muestran tecnicas de analisis que una vez implementadas
podrian tener un impacto drastico en el esfuerzo requerido para procesar los modelos de
variabilidad.


Sin embargo se muestra en años anteriores~\cite{jhf11, jhfes12} la relacion
entre el estudio y analisis de modelos de variabilidad y su realacion
sobre modelos de variabilidad realisticos y que esta a su vez no es compleja
ya que se muestra que la relacion combinatoria con modelos reales es
procesable en la practica. En particular el estudio de maquinas expendedoras 
de artículos~\cite{jhfes12} es un caso de uso real sobre lineas de productos
ampliamente descrito en la literatura. Pero su relacion con el estudio de las lineas de productos
de software no siempre esta relacionada directamente~\cite{cds06, fub06, nnz14}.

Recent implementations like ProFeat~cite{Chrszon2018} allows to 
help in the verification of requirements on families of probabilistic systems using 
it own language and based on Markov decision processes together with PRISM~\cite{mgd12}.










%%% Local Variables:
%%% mode: latex
%%% TeX-master: "main"
%%% End:
