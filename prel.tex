\section{Preliminares}



 \bdfn
   Una \emph{M�quina probabil�stica} (PM) es una tupla
   $N=(\calS,s_0,\Sigma,F,\calT)$ donde
    $\calS$ es el conjunto finito de estados y $s_0\in \calS$ representa el estado inicial.
     $\Sigma$ representa el alfabeto.
    $F=\{f_s | s\in \calS\}$ es un conjunto funciones probabil�sticas.
    Para todos los estados $s\in \calS$, $f_s$ es una funci�n
    $f_s:\Sigma \mapsto [0,1]$, y
    $\calT\subseteq \calS\times \Sigma \times \calS$ representa el conjunto 
    de transiciones.

    Si $s$ es un estado, entonces la suma de las probabilidades
    de las acciones realizadas por $s$, representadas como $\suma{N,s}$,
    ser�n descritas por la expresi�n.
   $$
   \suma{N,s}=\sum_{a\in \Sigma}f_s(a)
   $$

   Se muestra mediante $s_{1}\tranp{a}{p}s_{2}$ que $(s_{1},a,s_{2})\in \calT$ y 
       $p=f_{s_{1}}(a)$.
 \edfn


 \bdfn
   Dado el conjunto de acciones $\Sigma$.
   Una traza probabil�stica es una secuencia
   $\langle (a_1,p_1), (a_2,p_2),\ldots,(a_n,p_n)\rangle$
   donde $n\geq 0$, y para todos $0\leq k\leq n$ se tiene $a_k\in \Sigma$
   y $p_k\in[0,1]$.

   Dada una m�quina probabil�stica $N=(\calS,s_0,\Sigma,F,\calT)$ y la traza
   probabil�stica $\sigma= \langle (a_1,p_1),
   (a_2,p_2),$ $\ldots,(a_n,p_n)\rangle$. 
   Se dice que
   $\sigma$ es una traza de $N$, denotada por $\sigma\in\ptraces{N}$, 
   si existe una secuencia de transiciones de $N$ tal que:
   $
     (s_0,a_1,s_1),$ $(s_1,a_2, s_{2}),$ $\ldots, (s_{n-1},a_n, s_{n}) \in \calT
   $
   y para todos $1\leq k\leq n$ se tiene $p_k=f_{s_{k-1}}(a_k)$.
  
  En este caso ser� utilizada la siguiente notaci�n:
   \begin{itemize}
   \item Se dice que todas las acciones $a_k$, para $1\leq k\leq n$, 
     \emph{pertenecen a}
     $\sigma$, y es descrito por $a_k\in\sigma$.
   \item Se puede escribir esta traza probabil�stica como $s_0\vtran{\sigma}s_n$.
   \end{itemize}

 Se establece que $\sigma$ es una \emph{traza completa} si
 existe un estado $s_1\in \calS$ tal que
 $s_0\vtran{\sigma} s_1$ y $s_1$ no tiene transiciones.
 Es descrito por $\ctraces{N}$ el conjunto de todas las trazas completas de $N$.
 \edfn

%%% Local Variables: 
%%% mode: latex
%%% TeX-master: "main"
%%% End: 
