% $Id: fodaa.tex,v 1.32 2013/02/21 00:36:43 luis Exp $


\section{$\fodaPA$: syntax and semantics}
\label{sec:foda:sintaxMain}

In this section we will describe the 
terms, syntactical and semantic operators
from \fodaPA\ presented in~\cite{acl13}
in order to introduce the
reader to the probabilistic extension presented in this article.


\subsection{$\fodaPA$ syntax}

\fodaPA\ terms are denoted as $\calF$,
which will be an infinite set of features.
Also, \feature{A}, \feature{B}, \feature{C}\dots\ will
represent isolated features, belonging to $\calF$.
%
Now we need to define the semantic operators in order to establish the
language semantics.

\bdfn\label{dfn:syntax}
A \fodaPA\ \emph{software product line} is a term
with the following syntax::
$$
\begin{array}{ll}
  P::=&   \checkmark \barra \nil \barra \feature{A};P \barra \ofeature{A};P \barra\\
        &  P \choice Q \barra P\paral Q  \barra   \exclude{A}{B}{P}\barra\\
        & \require{A}{B}{P}\barra  \forbid{A}{P}\barra  \mandatory{A}{P}
\end{array}     
$$
\noindent 
where $\feature{A},\feature{B} \in \calF$.
To the set of terms belonging to the algebra
will be known as $\fodaPA$.
\edfn

Now we will proceed to describe each operator meaning and show
with examples how the operators should be computed based on the
semantic rules from ~\cite{acl13}.

We propose two terminal symbols, $\nil$ and $\checkmark$,
which are required to define the language
semantics and used as a termination mark when processing the
algebra terms.
If there is a case in which the $\checkmark$ symbol can be computed
this means that it is possible to generate a
valid product from the term, thus
will be computed immediately the $\nil$ symbol to
show that is not possible to continue the execution on that
branch.
%
The operators $\feature{A};P$ and $\ofeature{A};P$
allow to add the feature $\feature{A}$ to any product obtained
from $P$, where the $\feature{A};P$ operator means that
$\feature{A}$ is mandatory,
instead of the operator $\ofeature{A};P$ which means that
$\feature{A}$ is optional.
%
For example, the term $\feature{A};\ofeature{B};\checkmark$ represents
an \SPL\ with two valid products. In this case we obtain
a product with the feature \feature{A} and another with the features
\feature{A} y \feature{B}, as \feature{B} is optional.

In the initial definition for $\fodaPA$ there are two binary operators,
$P \choice Q$ and $P\paral Q$, where 
the first denotes the choose-one operator and the second one
the conjunction or parallel selection.

For example, the term
$\feature{A};\checkmark\choice (\feature{B};\checkmark\choice \feature{C};\checkmark)$ 
generated three valid products, each of them with one single feature, a 
product with the feature \feature{A}, another product with the feature \feature{B} and 
the last product with the feature \feature{C}.
%
The term $\feature{A};(\feature{B};\checkmark\paral \feature{C};\checkmark)$
describes a valid term including the paralel operator, this means
to allow the selection of all the features in the product, as in this case
the features are mandatory.
This will generate a single product with all the
features combined as ($\{\fA\fB\fC\}$) o
products combining the order in which each feature is computed like 
($\{[\fA\fB\fC],[\fA\fC\fB]\}$).

The operator $\require{A}{B}{P}$
represents the \emph{requirement} restriction, and the
$\exclude{A}{B}{P}$ operator shows the \emph{exclusion} restriction.
%
For example, the term $\require{A}{B}{\feature{A};\checkmark}$
generates one product with the features \feature{A} y \feature{B}.
The term
$P=\feature{A};(\feature{B};\checkmark\choice\feature{C};\checkmark)$,
generates two valid products.
The first with the features $\feature{A}$ and $\feature{B}$,
and the second with the
features  $\feature{A}$ and $\feature{C}$.
If we add a restriction like
$\exclude{A}{B}{P}$, this new term will generate
one single product with the features
$\feature{A}$ and $\feature{C}$. 
%
The operator $\mandatory{A}{P}$ is required to 
represent the requires relationship $\require{A}{B}{P}$.
When generating the products from the term
$\require{A}{B}{P}$, it must be taken into account 
if the feature \feature{A} was computed or not.
If it was computed, then \feature{B} must be marked to be processed
later.
The operator $\mandatory{B}{P}$ is used with this purpose.
Same behavior applies with the operator $\forbid{B}{P}$.
If the feature \feature{A} is computed, then, 
\feature{B} must not be included.
In this way the operator $\forbid{B}{P}$
indicates that \feature{B} is hidden.


% $Id: operational.tex,v 1.19 2013/02/21 00:36:43 luis Exp $
%---------------------------------------------------------------------------------------------------------------------------------------------------------------
\subsection{$\fodaPA$ operational semantics}
\label{sec:foda:operational}

\begin{figure*}[h]
	\linefigure
	\vspace{0.5cm}
	\centering\scalebox{1}{%
		$
		\begin{array}{*{3}{l@{}c@{\hspace{4em}}}}
		\nombreRegla{tick} & \checkmark\tran{\checkmark}\nil &
		\nombreRegla{feat} & \feature{A};P\tran{\feature{A}}P \\
		\nombreRegla{ofeat1} & \ofeature{A};P\tran{\feature{A}}P &
		\nombreRegla{ofeat2} & \ofeature{A};P\tran{\checkmark}\nil\\
		\nombreRegla{cho1} & \displaystyle \frac{P\tran{\feature{A}} P_1}{P\choice Q\tran{\feature{A}}P_1}&  
		\nombreRegla{cho2} & \displaystyle\frac{Q\tran{\feature{A}} Q_1}{P\choice Q\tran{\feature{A}}Q_1}\\
		\nombreRegla{con1} &  \displaystyle\frac{P\tran{\feature{A}} P_1}{P\paral Q\tran{\feature{A}}P_1\paral Q} &
		\nombreRegla{con2} & \displaystyle\frac{Q\tran{\feature{A}} Q_1}{P\paral Q\tran{\feature{A}}P\paral Q_1}\\
		% \nombreRegla{con3} &  \displaystyle\frac{P\tran{\feature{A}} P_1, Q\tran{\checkmark}\nil}{P\paral Q\tran{\feature{A}}P_1\paral Q} &
		% \nombreRegla{con4} & \displaystyle\frac{P\tran{\feature{A}} P_1, Q\tran{\checkmark}\nil}{Q\paral P\tran{\feature{A}}Q\paral P_1} \\ 
		\nombreRegla{con3} &  \displaystyle\frac{P\tran{\checkmark}\nil, Q\tran{\checkmark}\nil}{P\paral Q\tran{\checkmark}\nil} &
		\nombreRegla{req1} & \displaystyle
		\frac{P \tran{\feature{C}} P_1,\ \feature{C}\neq\feature{A}}{\require{A}{B}{P}
			\tran{\feature{C}} \require{A}{B}{P_1}} \\
		\nombreRegla{req2} &  \displaystyle
		\frac{P \tran{\feature{A}} P_1}{\require{A}{B}{P}\tran{\feature{A}} \mandatory{B}{P_1}} & 
		\nombreRegla{req3} &  \displaystyle
		\frac{P \tran{\feature{\checkmark}} \nil}{\require{A}{B}{P}
			\tran{\feature{\checkmark}} \nil}\\
		
		\nombreRegla{excl1} &    \displaystyle
		\frac{P \tran{\feature{C}} P_1,\ \feature{C}\neq\feature{A} \land \feature{C}\neq\feature{B}}{\exclude{A}{B}{P}
			\tran{\feature{C}} \exclude{A}{B}{P_1}} &
		\nombreRegla{excl2} & \displaystyle
		\frac{P \tran{\feature{A}} P_1}{\exclude{A}{B}{P}
			\tran{\feature{A}}\forbid{B}{P_1}} \\
		\nombreRegla{excl3} & \displaystyle
		\frac{P \tran{\feature{B}} P_1}{\exclude{A}{B}{P}\tran{\feature{B}}\forbid{A}{P_1}}
		&
		\nombreRegla{excl4} & \displaystyle
		\frac{P \tran{\checkmark} \nil}{\exclude{A}{B}{P}\tran{\checkmark}\nil}\\
		
		\nombreRegla{forb1} & \displaystyle
		\frac{P \tran{\feature{B}} P_1,\ \feature{B}\neq\feature{A}}{\forbid{A}{P}
			\tran{\feature{B}} \forbid{A}{P_1}} &
		\nombreRegla{forb2} & \displaystyle
		\frac{P \tran{\checkmark} \nil}{\forbid{A}{P}
			\tran{\checkmark} \nil}  \\
		%      \nombreRegla{forb3} & \displaystyle
		%     \frac{P\tran{\feature{A}}P_2 \tran{\feature{F}} P_1}{\forbid{A}{P}
		%       \tran{\feature{B}} \forbid{A}{P_1}},\ \feature{B}\neq\feature{A} &
		%      \nombreRegla{forb4} & \displaystyle
		%     \frac{P\tran{\feature{A}}P_2 \tran{\feature{\checkmark}} P_1}{\forbid{A}{P}
		%       \tran{\checkmark} \nil}\\ 
		%     
		% 
		% 
		\nombreRegla{mand1} &  \displaystyle\frac{P\tran{\checkmark} \nil}{\mandatory{A}{P}
			\tran{\f A} \checkmark}  &
		\nombreRegla{mand2} &  \displaystyle\frac{P\tran{\feature{A}} P_1}{\mandatory{A}{P}\tran{\feature{A}} P_1} \\
		\nombreRegla{mand3} &  \displaystyle\frac{P\tran{\feature{B}}
			P_1,\ \feature{A}\neq\feature{B}}{\mandatory{A}{P}\tran{\feature{B}} \mandatory{A}{P_1}}
		\\ \\
		\multicolumn{4}{c}{\feature{A},\feature{B},\feature{C}\in\calF,\ a\in\calF\cup\{\checkmark\}}
		\end{array}  
		$}
	\vspace{0.5cm}
	\linefigure
	
	\caption{Operational semantic rules of \fodaPA.  \label{fig:spla:sos-rules}}
	
	
	
\end{figure*}

\bdfn\label{def:insof:trantions}
Given the terms $P,Q\in\fodaPA$ and the feature
 $\feature{A}\in\calF\cup\{\checkmark\}$,
there is a transition from
$P$ to $Q$ labeled as $\feature{A}$, denoted by
$P\tran{\feature{A}}Q$, if it can be inferred from the semantic rules
describe in the figure~\ref{fig:spla:sos-rules}.  
\edfn

The intuitive meaning of the rule \nombreRegla{tick}
is that there are no more features to compute, so
a valid product can be generated, after this, the symbol
$\nil$ is computed to stop the rules processing.

Rules $\nombreRegla{feat}$, $\nombreRegla{ofeat1}$,
and $\nombreRegla{ofeat2}$ are related directly with the features
processing.
Rule $\nombreRegla{ofeat2}$ show the transition when the
feature is not computed, then a $\checkmark$
can be processed creating a valid product.
The feature \feature{A} is optional in $P$ because
$P\tran{\feature{A}}P_1$ and $P\tran{\checkmark}\nil$.


Rules $\nombreRegla{cho1}$ and $\nombreRegla{cho2}$ 
are related with the \emph{choose-one} operator, so forth
when processing 
$P\choice Q$, it must be chosen among the features in 
$P$ or the features in $Q$.

The rules  $\nombreRegla{con1}$,  $\nombreRegla{con2}$
and  $\nombreRegla{con3}$ 
represents the 
\emph{parallel} operator.
The rule $\nombreRegla{con1}$ and  $\nombreRegla{con2}$ show that the parallel operator is symmetrical, so forth,
any product of  $P\paral Q$
have to contain the features from  $P$ and $Q$. 
%
The rule$\nombreRegla{con3}$ shows that in order to generate a valid product
both parts of the operator need to be able to execute \nombreRegla{tick}.

The rules~\nombreRegla{req1}, \nombreRegla{req2},
and \nombreRegla{req3} are related to the requirement
restriction, which means that once
a feature is processed another must be also 
included in the product.
The rule~\nombreRegla{req1} shows that
$\require{A}{B}{P}$ behaves like 
$P$ in the mean time \feature{A} is not computed.
The rule \nombreRegla{req2} shows that \feature{B}
is mandatory once \feature{A} is processed.
Finally, the rule \nombreRegla{req3} is necessary 
because if any requires restriction was computed, 
and it can be computed \nombreRegla{tick},
then we can generate a valid product.


The rules related with the exclusion operator, \nombreRegla{excl1}, \nombreRegla{excl2},\nombreRegla{excl3} and \nombreRegla{excl4},
behaves in a similar way than the requirement rules.
The rule \nombreRegla{excl1} shows that $\exclude{A}{B}{P}$
behaves like $P$, iff $P$ can not produce the  feature \feature{A}
or the feature~\feature{B}.
The rule~\nombreRegla{excl2} shows that once $P$
produced \feature{A}, then \feature{B} must be excluded or hidden.
The rule~\nombreRegla{excl3} describes the same behavior with the
difference that is taken the opposite feature in the rule. 
The last exclusion rule \nombreRegla{excl4} shows the system behavior when
there was no exclusion rule computed. So forth, ${P}$ can compute
\nombreRegla{tick} and generate a valid product.

The rules~\nombreRegla{forb1} and \nombreRegla{forb2}
refer to the hiding operators introduced by the rules
\nombreRegla{excl2} and \nombreRegla{excl3}, 
where~\nombreRegla{forb1} establishes that in the mean time
it's not computed the hidden feature, the term can continue with
the processing, and the rule \nombreRegla{forb2} indicates that
once processed all the features, if there is no feature hidden, then
it is possible to generate a valid product.

The rules
\nombreRegla{mand1}, \nombreRegla{mand2} and~\nombreRegla{mand3} are
related with the auxiliary operator $\mandatory{A}{P}$.
The rule~\nombreRegla{mand1} indicates that~\feature{A}
must be computed before generating a valid product.
The rule~\nombreRegla{mand2} refers to the step in which once the
$\mandatory{A}{P}$ relationship is computed, then
the operator should be removed from the remaining term.
In the case of the rule \nombreRegla{mand3}, if the the feature in the relationship
$\mandatory{A}{P}$ is not processed,  then the operator should remain in the term
until is computed.
In the contrary case, if the term is fully computed, then \nombreRegla{mand1}
will generate a valid product.


\begin{figure}[h]
	
	\centering 
	
	\linefigure
	\vspace{0.5cm}
	\begin{minipage}{0.3\hsize}\centering
		\EX{a}
		\vskip 0.5em
		
		\scalebox{0.8}{%
			\begin{tikzpicture}[->,>=stealth',node distance=1.8cm]
			\node[syntax] (A)   {$\feature{A};\ofeature{B};\checkmark$};
			\node[syntax] (B) [below of=A]   {$\ofeature{B};\checkmark$};
			\node[syntax] (C) [below left of=B]   {\nil};
			\node[syntax] (D) [below right of=B]   {\checkmark};
			\node[syntax] (F) [below of=D]   {\nil};
			
			\path (A) edge[->] node[right, opacity=0, text opacity=1] {\feature{A}~$\nombreRegla{feat}$} (B)
			(B) edge[->] node[above left, opacity=0, text opacity=1] {$\nombreRegla{ofeat2}$~\checkmark} (C)
			(B) edge[->] node[above right, opacity=0, text opacity=1] {\feature{B}~$\nombreRegla{ofeat1}$} (D)
			%(C) edge[->] node[left] {$\nombreRegla{nil}$~\checkmark} (E)
			(D) edge[->] node[left, opacity=0, text opacity=1] {$\nombreRegla{nil}$~\checkmark} (F);
			\end{tikzpicture}
		}
		%
	\end{minipage}
	\begin{minipage}{0.3\hsize}\centering
		\EX{b}
		\vskip 0.5em
		\scalebox{0.8}{%
			\begin{tikzpicture}[->,>=stealth',node distance=1.8cm]
			\node[syntax] (A)   {$\feature{A};\feature{B};\checkmark$};
			\node[syntax] (B) [below of=A]   {$\feature{B};\checkmark$};
			\node[syntax] (D) [below of=B]   {\checkmark};
			\node[syntax] (E) [below of=D]   {\nil};
			
			\path (A) edge[->] node[right, opacity=0, text opacity=1] {\feature{A}~$\nombreRegla{feat}$} (B)
			(B) edge[->] node[right, opacity=0, text opacity=1] {\feature{B}~$\nombreRegla{feat}$} (D)
			(D) edge[->] node[left, opacity=0, text opacity=1] {$\nombreRegla{nil}$~\checkmark} (E);
			\end{tikzpicture}
		}
	\end{minipage}
	\begin{minipage}{0.3\hsize}\centering
		\EX{c}
		\vskip 0.5em
		\scalebox{0.8}{%
			\begin{tikzpicture}[->,>=stealth',auto,node distance=1.8cm]
			\node[syntax] (A)   {$\feature{A};(\feature{B};\checkmark\choice \feature{C};\checkmark)$};
			\node[syntax] (B) [below of=A]   {$\feature{B};\checkmark\choice \feature{C};\checkmark$};
			\node[syntax] (D) [below left of=B,node distance=2.5cm]   {\checkmark};
			\node[syntax] (E) [below right of=B,node distance=2.5cm]   {\checkmark};
			\node[syntax] (F) [below  of=E]   {\nil};
			\node[syntax] (G) [below  of=D]   {\nil};
			
			\path (A) edge[->] node [opacity=0, text opacity=1]{\feature{A}~$\nombreRegla{feat}$} (B)
			(B) edge[->] node[above left, opacity=0, text opacity=1] {\feature{B}~$\nombreRegla{cho1}$} (D)
			(B) edge[->] node[above right, opacity=0, text opacity=1] {\feature{C}~$\nombreRegla{cho2}$} (E)
			(D) edge[->] node[left, opacity=0, text opacity=1] {\checkmark~$\nombreRegla{nil}$} (G)
			(E) edge[->] node[right, opacity=0, text opacity=1] {\checkmark~$\nombreRegla{nil}$} (F);
			
			;
			\end{tikzpicture}
		}
	\end{minipage}
	
	
	\vspace{0.5cm}
	
	$\begin{array}{ll}
	\EX{a} &
	\products{\feature{A};\ofeature{B};\checkmark}=
	\left\{
	\begin{array}{l}
	[\feature{A}],
	[\feature{A}\feature{B}]%
	\end{array}
	\right\}
	\\ 
	\EX{b}&
	\products{\feature{A};\feature{B};\checkmark}=
	\left\{
	\begin{array}{l}
	[\feature{A}\feature{B}]
	\end{array}
	\right\}
	\\ 
	\EX{c}&
	\products{\feature{A};(\feature{B};\checkmark\choice \feature{C};\checkmark)}=
	\left\{
	\begin{array}{l}
	\left[\feature{A}\feature{B}\right],
	\left[\feature{A}\feature{C}\right]\\
	\end{array}
	\right\}
	\end{array}$
	
	\vspace{0.5cm}
	\linefigure
	\caption{Operational semantics application for the operators 
		$\nombreRegla{feat}$, $\nombreRegla{ofeat}$ and $\nombreRegla{cho}$.\label{figure:semantics:optcomp}}
\end{figure}

\begin{figure}[h]
	\linefigure
	\vspace{0.5cm}
	\centering
	\begin{minipage}[t]{.5\textwidth}
		\centering
		\EX{d} \\ \vspace{0.5cm}
		\scalebox{0.7}{
			\begin{tikzpicture}[->,>=stealth',auto,node distance=2cm]
			
			\node[syntax] (A){$\feature{A};(\feature{B};\checkmark\paral \feature{C};\checkmark)$};
			\node[syntax] (B) [below of=A]   {$\feature{B};\checkmark\paral \feature{C};\checkmark$};
			\node[syntax] (D) [below left of=B,node distance=2.5cm]   {$\checkmark\paral \feature{C};\checkmark$};
			\node[syntax] (E) [below right of=B,node distance=2.5cm]   {$\feature{B};\checkmark\paral \checkmark$};
			\node[syntax] (F) [below  of=E]   {$\checkmark\paral \checkmark$};
			\node[syntax] (G) [below  of=D]   {$\checkmark\paral \checkmark$};
			\node[syntax] (H) [below  of=F]   {\nil};
			\node[syntax] (I) [below  of=G]   {\nil};
			
			
			\path (A) edge[->] node [opacity=0, text opacity=1]{\feature{A}~$  \nombreRegla{feat}$} (B)
			(B) edge[->] node[left, opacity=0, text opacity=1] {\feature{B}~$\nombreRegla{con1}$} (D)
			(B) edge[->] node[right, opacity=0, text opacity=1] {\feature{C}~$\nombreRegla{con2}$} (E)
			(D) edge[->] node[left, opacity=0, text opacity=1] {\feature{C}~$\nombreRegla{con2}$} (G)
			(E) edge[->] node[right, opacity=0, text opacity=1] {$\feature{B}~\nombreRegla{con1}$} (F)
			(F) edge[->] node[right, opacity=0, text opacity=1] {$\checkmark~\nombreRegla{con3}$} (H)
			(G) edge[->] node[left, opacity=0, text opacity=1] {$\checkmark~\nombreRegla{con3}$} (I)          
			;
			\end{tikzpicture}}
	\end{minipage}% <---------------- Note the use of "%"
	\begin{minipage}[t]{.5\textwidth}
		\centering
		\EX{e} \\ \vspace{0.5cm}
		\scalebox{0.7}{%
			
			\begin{tikzpicture}[->,>=stealth',auto,node distance=2cm]
			
			\node[syntax] (A){$\feature{A};(\ofeature{B};\checkmark\paral \feature{C};\checkmark)$};
			\node[syntax] (B) [below of=A]   {$\ofeature{B};\checkmark\paral \feature{C};\checkmark$};
			\node[syntax] (C) [below left of=B,node distance=2.5cm]   {$\checkmark\paral \feature{C};\checkmark$};
			\node[syntax] (D) [below right of=B,node distance=2.5cm]   {$\ofeature{B}\checkmark\paral \checkmark$};
			\node[syntax] (E) [below of=C,node distance=2.5cm]   {$\checkmark\paral \checkmark$};
			\node[syntax] (F) [below left of=D,node distance=2.5cm]   {$\checkmark\paral \checkmark$};
			\node[syntax] (G) [below right of=D,node distance=2.5cm]   {$\nil$};
			\node[syntax] (H) [below of=E,node distance=2.5cm]   {$\nil$};
			\node[syntax] (I) [below of=F,node distance=2.5cm]   {$\nil$};			
			
			
			
			
			
			\path (A) edge[->] node [opacity=0, text opacity=1]{\feature{A}~$\nombreRegla{feat}$} (B)
			
			(B) edge[->] node[left, opacity=0, text opacity=1] {\feature{B}~$\nombreRegla{con1}$} (C)
			(B) edge[->] node[right, opacity=0, text opacity=1] {\feature{C}~$\nombreRegla{con2}$} (D)
			
			(C) edge[->] node[left, opacity=0, text opacity=1] {\feature{C}~$\nombreRegla{con2}$} (E)
			
			(D) edge[->] node[left, opacity=0, text opacity=1] {\feature{B}~$\nombreRegla{con1}$} (F)	
			(D) edge[->] node[right, opacity=0, text opacity=1] {$\checkmark$~$\nombreRegla{con3}$~$\nombreRegla{ofeat2}$} (G)	
			
			(E) edge[->] node[left, opacity=0, text opacity=1] {$\checkmark$~$\nombreRegla{con3}$} (H)	
			(F) edge[->] node[right, opacity=0, text opacity=1] {$\checkmark$~$\nombreRegla{con3}$} (I)					
			
			
			;		
			
			
			\end{tikzpicture}% 
		}
	\end{minipage}
	\vspace{0.5cm}
	
	$\begin{array}{ll}
	\EX{d}&
	\products{\feature{A};(\feature{B};\checkmark\paral \feature{C};\checkmark)}=
	\left\{
	\begin{array}{l}
	\left[\feature{A}\feature{B}\feature{C}\right]%, \\
	% \left[\feature{A}\feature{C}\feature{B}\right]\\ [ABC] y [ACB] son el mismo producto
	\end{array}
	\right\}
	\\
	\EX{e} &
	\products{\feature{A};(\ofeature{B};\checkmark\paral \feature{C};\checkmark)}=
	\left\{
	\begin{array}{l}
	\left[\feature{A}\feature{B}\feature{C}\right], 
	\left[\feature{A}\feature{C}\right]\\ 
	\end{array}
	\right\}
	\end{array}$	
	\vspace{0.5cm}
	
	\linefigure
	\caption{Operational semantics application for the operator $\nombreRegla{con}$.\label{figure:semantics:optcomp2}}
	
\end{figure}

In the examples~\EX{a} and~\EX{b} from figure
\ref{figure:semantics:optcomp}
can be described the rules behavior when computing optional or
mandatory feature.
The example~\EX{a}, 
shows that the feature~\feature{B} is optional, in the mean time, in \EX{b}
is mandatory.
This condition is described in the example
\EX{b},
because there is a term corresponding with the transition
$\ofeature{B};\checkmark\tran{\checkmark}\nil$,
which is not present in the example~\EX{a}.


\begin{figure}[h]
	\centering
	\linefigure
	\vspace{0.5cm}
	\centering
	\begin{minipage}[t]{.4\textwidth}
		\centering
		\EX{f} \\ \vspace{0.5cm}
		
		\scalebox{0.6}{
			\begin{tikzpicture}[->,>=stealth',auto,node distance=2cm]
			
			\node[syntax] (A){$\exclude{B}{C}{ \feature{A};(\ofeature{B};\checkmark\paral \ofeature{C};\checkmark)}$};
			\node[syntax] (B) [below of=A]   {$\exclude{B}{C}{(\ofeature{B};\checkmark\paral \ofeature{C};\checkmark)}$};
			\node[syntax] (C) [below left of=B,node distance=4cm]   {$\forbid{C}{(\checkmark\paral\ofeature{C};\checkmark)}$};
			\node[syntax] (D) [below of=B,node distance=3cm]   {$\nil$};
			\node[syntax] (E) [below right=1.5cm and 2cm of B]   {$\forbid{B}{(\ofeature{B};\checkmark\paral\checkmark)}$};
			\node[syntax] (F) [below  of=C]   {$\nil$};
			\node[syntax] (G) [below  of=E]   {$\nil$};
			
			
			\path (A) edge[->] node [opacity=0, text opacity=1]{\feature{A}~$\nombreRegla{feat}$~$\nombreRegla{excl1}$} (B)
			
			(B) edge[->] node[left, opacity=0, text opacity=1] {\feature{B}~$\nombreRegla{excl2}$} (C)
			(B) edge[->] node[below right, opacity=0, text opacity=1] {\checkmark~$\nombreRegla{con3}$~$\nombreRegla{ofeat2}$} (D)
			(B) edge[->] node[right, opacity=0, text opacity=1] {\feature{C}~$\nombreRegla{excl3}$} (E)
			
			(C) edge[->] node[right, opacity=0, text opacity=1] {$\checkmark$~$\nombreRegla{ofeat2}$~$\nombreRegla{forb2}$} (F)
			(E) edge[->] node[right, opacity=0, text opacity=1] {$\checkmark$~$\nombreRegla{ofeat2}$~$\nombreRegla{forb2}$} (G)
			
			;
			\end{tikzpicture}}
		
	\end{minipage}% <---------------- Note the use of "%"
	\begin{minipage}[t]{.6\textwidth}
		\centering
		\EX{g} \\ \vspace{0.5cm}
		\scalebox{0.6}{
			\begin{tikzpicture}[->,>=stealth',auto,node distance=2cm]
			
			
			\node[syntax] (A){$\require{B}{C}{ \feature{A};(\ofeature{B};\checkmark\paral \ofeature{C};\checkmark)}$};
			\node[syntax] (B) [below of=A]   {$\require{B}{C}{(\ofeature{B};\checkmark\paral \ofeature{C};\checkmark)}$};
			\node[syntax] (C) [below left=1cm and 0.5cm of B]   {$\mandatory{C}{(\checkmark\paral\ofeature{C};\checkmark)}$};
			\node[syntax] (D) [below of=B]   {$\nil$};
			\node[syntax] (E) [below right=1cm and 0.5cm of B]   {$\require{B}{C}{(\ofeature{B};\checkmark\paral\checkmark)}$};
			\node[syntax] (F) [below  of=C]   {$\checkmark$};
			\node[syntax] (G) [below  left=2.5cm and 0.5cm of E]   {$\nil$};
			\node[syntax] (H) [below  of =E]   {$\mandatory{C}{(\checkmark\paral\checkmark)}$};			
			\node[syntax] (I) [below  of=H]   {$\checkmark$};			
			\node[syntax] (J) [below  of=I]   {$\nil$};	
			\node[syntax] (K) [below  of=F]   {$\nil$};	
			
			
			\path (A) edge[->] node [opacity=0, text opacity=1]{\textbf{\feature{A}~$\nombreRegla{feat}$~$\nombreRegla{excl1}$}} (B)
			
			(B) edge[->] node[left, opacity=0, text opacity=1] {\feature{B}~$\nombreRegla{req2}$} (C)
			(B) edge[->] node[right, opacity=0, text opacity=1] {\checkmark~$\nombreRegla{con3}$} (D)
			(B) edge[->] node[right, opacity=0, text opacity=1] {\feature{C}~$\nombreRegla{con2}$} (E)
			
			(C) edge[->] node[right, opacity=0, text opacity=1] {$\feature{C}$~$\nombreRegla{mand1}$~$\nombreRegla{mand2}$} (F)
			(E) edge[->] node[below left=0cm and 0.5cm, opacity=0, text opacity=1] {$\checkmark$~$\nombreRegla{ofeat2}$~$\nombreRegla{con3}$} (G)
			(E) edge[->] node[right, opacity=0, text opacity=1] {$\feature{B}$~$\nombreRegla{ofeat2}$~$\nombreRegla{con3}$} (H)	
			(F) edge[->] node[right, opacity=0, text opacity=1] {\checkmark~$\nombreRegla{nil}$} (K)	
			(H) edge[->] node[right, opacity=0, text opacity=1] {$\feature{C}$~$\nombreRegla{mand1}$} (I)
			(I) edge[->] node[right, opacity=0, text opacity=1] {\checkmark~$\nombreRegla{nil}$} (J)
			;
			
			\end{tikzpicture}}
	\end{minipage}
	\vspace{0.5cm}
	
	$\begin{array}{ll}
	\EX{f} &
	\products{\begin{array}{l}\exclude{B}{C}{ \feature{A};\\(\ofeature{B};\checkmark\paral \ofeature{C};\checkmark)}\end{array}}=
	\left\{
	\begin{array}{l}
	[\feature{A}],
	\left[\feature{A}\feature{B}\right],
	\left[\feature{A}\feature{C}\right]\\
	\end{array}
	\right\}
	\\ \\
	\EX{g}&
	\products{\begin{array}{l}\require{B}{C}{ \feature{A};\\(\ofeature{B};\checkmark\paral \ofeature{C};\checkmark)}\end{array}}=
	\left\{
	\begin{array}{l}
	
	\left[\feature{A}\feature{B}\feature{C}\right],
	[\feature{A}],\left[\feature{A}\feature{C}\right]\\
	\end{array}
	\right\}
	\end{array}$	
	\vspace{0.5cm}
	
	\linefigure	
	
	\caption{Operational semantics application for the operators
		$\nombreRegla{excl}$ and $\nombreRegla{mand}$.\label{figure:semantics:optcomp3}}
	
\end{figure}

The example \EX{c} from the figure~\ref{figure:semantics:optcomp}
and the example~\EX{d} from the figure \ref{figure:semantics:optcomp2}
show the difference between the choose-one and parallel operators respectively.
In the \emph{choose-one}, operator the non-necessary part from the term which is not
needed for processing the next features is removed, 
in the mean time in the \emph{parallel} operator we keep all the term.
From the example~\EX{c}, we obtain the traces
$\feature{A}\feature{B}$ and $\feature{A}\feature{C}$,
which mean two different products
$[\feature{A}\feature{B}]$ and $[\feature{A}\feature{C}]$.
In the contrary, the traces from the example~\EX{d} are $\feature{A}\feature{B}\feature{C}$ and $\feature{A}\feature{C}\feature{B}$,
generating a unique valid product,
$[\feature{A}\feature{B}\feature{C}]$.
In the example \EX{e} it is displayed the same term used in the
example \EX{d} adding \feature{B} as optional, generating
$[\feature{A}\feature{B}\feature{C}]$ and
$[\feature{A}\feature{C}]$.

In the figure~\ref{figure:semantics:optcomp3} with the examples
\EX{f} and \EX{g}, we can show the \emph{exclusion} and \emph{requirement}
operators behavior, showing that once those rules are applied will be
processed the auxiliary operators of \emph{hiding} and \emph{mandatory}
respectively.

\subsection{$\fodaPA$ denotational semantics}
\label{sec:foda:denotational}

The \emph{denotational}
semantics rules do not rely
on computational steps for transitioning between
states. A set of functions will be defined
to model the system behavior, based on describing a
semantic operator for each syntactical operator in $\fodaPA$.


Given the needed mathematical domain, $\calP(\calP(\calF))$,
if $\calF$ is a set,
$\calP(\calF)$ will be the superset of $\calF$.

This is achieved using the following definition.

\bdfn\label{def:semmatic:operators}
Given the products sets $P, Q\in\calP(\calP(\calF))$
and the features $\feature{A},\feature{B}\in\calF$,
the following operators are defined:

\begin{itemize}
	\item $\semden{\nil}=\emptyset$
	\item $\semden{\checkmark}=\{\emptyset\}$
	\item 
	$\semden{\feature{A};\cdot}:\calP(\calP(\calF))\mapsto\calP(\calP(\calF))$
	as
	$$\semden{\feature{A};\cdot}(P)=\{\{\feature{A}\}\cup p\ |\ p\in
	P\}$$
	
	\item 
	$\semden{\ofeature{A};\cdot}:\calP(\calP(\calF))\mapsto\calP(\calP(\calF))$
	as
	$$\semden{\ofeature{A};\cdot}(P)=\{\emptyset\}\cup\{\{\feature{A}\}\cup p\ |\ p\in
	P\}$$
	
	\item 
	$\semden{\cdot\choice\cdot}:\calP(\calP(\calF))\times\calP(\calP(\calF))\mapsto\calP(\calP(\calF))$
	as 
	$$\semden{\cdot\choice\cdot}(P,Q)=P\cup Q$$
	\item 
	$\semden{\cdot\paral\cdot}:\calP(\calP(\calF))\times\calP(\calP(\calF))\mapsto\calP(\calP(\calF))$
	as 
	$$\semden{\cdot\paral\cdot}(P,Q)=\{ p\cup q\ |\ p\in P,\ q\in Q\}$$
	\item  
	$\semden{\require{A}{B}{\cdot}}:\calP(\calP(\calF))\mapsto\calP(\calP(\calF))$
	as 
	$$\semden{\require{A}{B}{\cdot}}(P)=\begin{array}[t]{l}
	\{p\ |\ p\in P, \feature{A}\not\in p\}\cup\\
	\{p\cup\{\feature{B}\}\ |\ p\in P, \feature{A}\in p\} 
	\end{array}$$
	\item  
	$\semden{\exclude{A}{B}{\cdot}}:\calP(\calP(\calF))\mapsto\calP(\calP(\calF))$
	as 
	$$\semden{\exclude{A}{B}{\cdot}}(P)=\begin{array}[t]{l}
	\{p\ |\ p\in P, \feature{A}\not\in p\}\cup\\
	\{p\ |\ p\in P, \feature{B}\not\in p\}
	\end{array}$$
	\item $\semden{\mandatory{A}{\cdot}}:\calP(\calP(\calF))\mapsto\calP(\calP(\calF))$
	as 
	$$\semden{\mandatory{A}{\cdot}}(P) = \{p\cup\{\feature{A}\}\ |\ p\in P\}$$
	\item   $\semden{\forbid{A}{\cdot}}:\calP(\calP(\calF))\mapsto\calP(\calP(\calF))$
	as 
	$$\semden{\forbid{A}{\cdot}}(P) = \{p\ |\ p\in P, \feature{A}\not\in p\}$$
\end{itemize}
\edfn


\begin{figure}[h] 
	\linefigure	
	\vspace{0.5cm}
	\noindent\begin{minipage}[t]{.52\linewidth}
		\centering
		\scalebox{0.7}{
			$
			\begin{array}{l@{}l@{}l}
			\multicolumn{3}{c}{\EX{a}} \\
			\semden{\checkmark} &=& \{\emptyset\}\\
			\semden{\ofeature{B};\checkmark}\ &=& 
			\semden{\ofeature{B};\cdot}(\semden{\checkmark})\ = 
			\semden{\ofeature{B};\cdot}(\{\emptyset\})\ =  \{\emptyset, \{\feature{B}\}\}\\
			\semden{\feature{A};\ofeature{B};\checkmark}\ &=&
			\semden{\feature{A};\cdot}(\semden{\ofeature{B};\checkmark})\ =\semden{\feature{A};\cdot}(\{\emptyset\}\cup \{\feature{B}\})\ =\\
			&&  \{\{\feature{A}\},\{\feature{A},\feature{B}\}\}\\
			
			& & \\
			\multicolumn{3}{c}{\EX{b}} \\
			\semden{\feature{B};\checkmark}\ &=&
			\semden{\feature{B};\cdot}(\semden{\checkmark})\ = \
			\semden{\feature{B};\cdot}(\{\emptyset\})\ =\
			\{\{\feature{B}\}\}\\
			\semden{\feature{A};\feature{B};\checkmark}\ &=&
			\semden{\feature{A};\cdot}(\semden{\feature{B};\checkmark})\ =\
			\semden{\feature{A};\cdot}(\{\feature{B}\})\ = \{\{\feature{A},\feature{B}\}\}\\
			
			& & \\
			\multicolumn{3}{c}{\EX{c}} \\
			\semden{\feature{B};\checkmark}\ &=& \{\{\feature{B}\}\}\\
			\semden{\feature{C};\checkmark}\ &=& \{\{\feature{C}\}\}\\
			\semden{\feature{B};\checkmark\choice \feature{C};\checkmark}\ &=&
			\semden{\cdot\choice\cdot}(\semden{\feature{B};\checkmark},\semden{\feature{C};\checkmark})\ =\\
			&& \semden{\feature{B};\checkmark}\cup\semden{\feature{C};\checkmark}\ =\
			\{\{\feature{B}\},\{\feature{C}\}\}\\
			\semden{\feature{A};(\feature{B};\checkmark\choice \feature{C};\checkmark)}\ &=&
			\semden{\feature{A};\cdot}(\semden{\feature{B};\checkmark\choice \feature{C};\checkmark})\ =\\
			&& \semden{\feature{A};\cdot}(\{\{\feature{B}\},\{\feature{C}\}\})\ =\
			\{\{\feature{A},\feature{B}\},\{\feature{A},\feature{C}\}\}\\
			
			
			& & \\
			\multicolumn{3}{c}{\EX{d}} \\
			\semden{\feature{B};\checkmark\paral \feature{C};\checkmark}\ &=&
			\semden{\cdot\paral\cdot}(\semden{\feature{B};\checkmark},\semden{\feature{C};\checkmark})\ =\\
			\semden{\cdot\paral\cdot}(\{\{\feature{B}\}\}, \{\{\feature{C}\}\})\ &=&
			\{\{\feature{B},\feature{C}\}\}\\
			\semden{\feature{A};(\feature{B};\checkmark\paral \feature{C};\checkmark)}\ &=&
			\semden{\feature{A};\cdot}(\{\{\feature{B},\feature{C}\}\})\ =\\
			&&
			\{\{\feature{A},\feature{B},\feature{C}\}\}\\
			\end{array}
			$}
	\end{minipage}%%
	\begin{minipage}[t]{.4\linewidth}
		\centering
		\scalebox{0.7}{
			$
			\begin{array}{l@{}l@{}l}
			\multicolumn{3}{c}{\EX{e}} \\
			\semden{\ofeature{B};\checkmark\paral \feature{C};\checkmark)}\ &=&
			\semden{\cdot\paral\cdot}(\{\emptyset,\{\feature{B}\}\},\{\{\feature{C}\}\})\ =\\
			&&   \{\{\feature{C}\},\{\feature{B},\feature{C}\}\}\\
			\semden{\feature{A};(\ofeature{B};\checkmark\paral \feature{C};\checkmark)}\ &=&
			\semden{\feature{A};\cdot}(\semden{\ofeature{B};\checkmark\paral \feature{C};\checkmark)})\ =\\
			&&\semden{\feature{A};\cdot}(\{\{\feature{C}\},\{\feature{B},\feature{C}\}\})\ =\\
			&& \{\{\feature{A},\feature{C}\},\{\feature{A},\feature{B},\feature{C}\}\}\\
			
			
			& & \\
			\multicolumn{3}{c}{\EX{f}} \\
			\semden{\ofeature{B};\checkmark\paral \ofeature{C};\checkmark}\ &=&
			\semden{\cdot\paral\cdot}(\semden{\ofeature{B};\checkmark},\semden{\ofeature{C};\checkmark})\ =\\
			&&   \semden{\cdot\paral\cdot}(\{\emptyset,\{\feature{B}\}\},\{\emptyset,\{\feature{C}\}\})\ = \\
			&&   \{\emptyset,\{\feature{B}\},\{\feature{C}\},\{\feature{B},\feature{C}\}\}\\
			\semden{\feature{A};
				(\ofeature{B};\checkmark\paral \ofeature{C};\checkmark)}\ &=& 
			\semden{\feature{A};\cdot}(\semden{\ofeature{B};\checkmark\paral \ofeature{C};\checkmark})\ =\\
			&& \semden{\feature{A};\cdot}(\{\emptyset,\{\feature{B}\},\{\feature{C}\},\{\feature{B},\feature{C}\}\})\ =\\
			&& \{\{\feature{A}\},\{\feature{A},\feature{B}\},\{\feature{A},\feature{C}\},\{\feature{A},\feature{B},\feature{C}\}\}\\
			\Semden{
				\begin{array}{l}\exclude{B}{C}{\feature{A};}\\
				(\ofeature{B};\checkmark\paral \ofeature{C};\checkmark)\end{array}}\ &=&
			\semden{\exclude{B}{C}{\cdot}}\left(\begin{array}{l}\{\{\feature{A}\},\{\feature{A}, \feature{B}\},\\ \{\feature{A},\feature{C}\},\{\feature{A},\feature{B},\feature{C}\}\}\end{array}\right)\ =\\
			&& \{\{\feature{A}\},\{\feature{A},\feature{C}\},\{\feature{A},\feature{B}\}\}\\ 
			
			
			& & \\
			\multicolumn{3}{c}{\EX{g}} \\
			\Semden{
				\begin{array}{l}\require{B}{C}{\feature{A};}\\
				(\ofeature{B};\checkmark\paral \ofeature{C};\checkmark)\end{array}}\ &=&
			\semden{\require{B}{C}{\cdot}}\left(\begin{array}{l}\{\{\feature{A}\},\{\feature{A}, \feature{B}\},\\ \{\feature{A},\feature{C}\},\{\feature{A},\feature{B},\feature{C}\}\}\end{array}\right)\ =\\
			&& \{\{\feature{A}\},\{\feature{A},\feature{B},\feature{C}\},\{\feature{A},\feature{C}\}\}\\ 
			\end{array}
			$}		
		
	\end{minipage} 
	\vspace{0.5cm}
	\linefigure
	
	\caption{Denotational semantics rules application.\label{figure:denotational:optcomp}}
\end{figure}

To explain the denotational semantics behavior the same examples
\EX{a}
\EX{b}
\EX{c}
\EX{d}
\EX{e}
\EX{f}
\EX{g}
from the figures
\ref{figure:semantics:optcomp},
\ref{figure:semantics:optcomp2} and
\ref{figure:semantics:optcomp3}
respectively
are processed used the rules presented in the denotational semantics definition
in the figure~\ref{figure:denotational:optcomp}.

Given the proper introduction to the operational and denotational semantics
from the previous work, we continue in the next section defining the syntax and semantics
of the probabilistic extension.


