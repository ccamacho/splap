% $Id$


\section{Semántica Operacional}


%TODO: 
%\begin{itemize}
%\item to define a notation for multisets: $[q\ |\ ..... ]$ is a multiset.
%\item why do we need a multiset?
%\item the product corresponding to a trace $s$: $\product{s}$
%\item what a product is
%\end{itemize}

La semántica operacional para aquellos términos  $\fodaPAp$ está
basada en las reglas descritas en la figura~\ref{fig:sos-rules}.
Sin embargo, estas reglas deberán  ser modificadas para permitir
representar probabilidades dentro del modelo.

En esta sección serán descritos aquellos cambios necesarios para poder
representar probabilidades en el modelo.

\todo{Hay que explicar las reglas}

\begin{figure*}[h]
        \linefigure
        
        \centering\scalebox{0.95}{%
        $
        \begin{array}{*{3}{l@{}c@{\hspace{4em}}}}
        \nombreRegla{tick} & \checkmark\tranp{\checkmark}{1}\nil &
        \nombreRegla{feat} & \feature{A};P\tranp{\feature{A}}{1}P \\
        \nombreRegla{ofeat1} & \ofeature{A};_{p}P\tranp{\feature{A}}{p}P &
        \nombreRegla{ofeat2} & \ofeature{A};_{p}P\tranp{\checkmark}{(1-p)}\nil\\
        \nombreRegla{cho1} & \displaystyle \frac{P\tranp{a}{q} P_1}{P\choice_{p} Q\tranp{a}{p\cdot q} P_1}&  
        \nombreRegla{cho2} & \displaystyle\frac{Q\tranp{a}{q} Q_1}{P\choice_{p} Q\tranp{a}{(1-p)\cdot q}Q_1}\\
        \nombreRegla{con1} & \displaystyle\frac{P\tranp{\feature{A}}{p} P_1}{P\paral Q\tranp{\feature{A}}{\frac{p}{2}}P_1\paral Q} &
        \nombreRegla{con2} & \displaystyle\frac{Q\tranp{\feature{A}}{q} Q_1}{P\paral Q\tranp{\feature{A}}{\frac{q}{2}}P\paral Q_1}\\
        \nombreRegla{con3} & \displaystyle\frac{P\tranp{\checkmark}{q}\nil, Q\tranp{\checkmark}{p}\nil}{P\paral Q\tranp{\checkmark}{p\cdot q}\nil} &\\
        \nombreRegla{con4} & \displaystyle\frac{P\tranp{\feature{A}}{p} P_1, Q\tranp{\checkmark}{q}\nil}{P\paral Q\tranp{\feature{A}}{\frac{p\cdot q}{2}} P_1} &
        \nombreRegla{con5} & \displaystyle\frac{Q\tranp{\feature{A}}{q} Q_1, P\tranp{\checkmark}{p}\nil}{P\paral Q\tranp{\feature{A}}{\frac{p\cdot q}{2}} Q_1} \\  
        %
        %
        \nombreRegla{req1} & \displaystyle
        \frac{P \tranp{\feature{C}}{p} P_1,\ \feature{C}\neq\feature{A},\ \feature{C}\neq\feature{B},}{\require{A}{B}{P}
                \tranp{\feature{C}}{p} \require{A}{B}{P_1}} &
        \nombreRegla{req2} &  \displaystyle
        \frac{P \tranp{\feature{A}}{p} P_1}{\require{A}{B}{P}\tranp{\feature{A}}{p} \mandatory{B}{P_1}} & \\
        % \nombreRegla{req3} & \displaystyle
        % \frac{P \tranp{\feature{B}}{p} P_1}{\require{A}{B}{P}
        %   \tranp{\feature{B}}{p} P_1} & 
        \nombreRegla{req4} &  \displaystyle  
        \frac{P \tranp{\feature{\checkmark}}{p} \nil}{\require{A}{B}{P}
                \tranp{\feature{\checkmark}}{p} \nil}\\
        \nombreRegla{excl1} &    \displaystyle
        \frac{P \tranp{\feature{C}}{p} P_1,\ \feature{C}\neq\feature{A},\  \feature{C}\neq\feature{B}}{
                \exclude{A}{B}{P}\tranp{\feature{C}}{p} \exclude{A}{B}{P_1}} &
        \nombreRegla{excl2} & \displaystyle
        \frac{P \tranp{\feature{A}}{p} P_1}{\exclude{A}{B}{P}
                \tranp{\feature{A}}{p}\forbid{B}{P_1}} \\
        \nombreRegla{excl3} & \displaystyle
        \frac{P \tranp{\feature{B}}{p} P_1}{\exclude{A}{B}{P}\tranp{\feature{B}}{p}\forbid{A}{P_1}}
        &
        \nombreRegla{excl4} & \displaystyle
        \frac{P \tranp{\checkmark}{p} \nil}{\exclude{A}{B}{P}\tranp{\checkmark}{p}\nil}\\
        
        \nombreRegla{forb1} & \displaystyle
        \frac{P \tranp{\feature{B}}{p} P_1,\ \feature{B}\neq\feature{A}}{\forbid{A}{P}
                \tranp{\feature{B}}{p} \forbid{A}{P_1}} &
        \nombreRegla{forb2} & \displaystyle
        \frac{P \tranp{\checkmark}{p} \nil}{\forbid{A}{P}
                \tranp{\checkmark}{p} \nil}  \\
        %      \nombreRegla{forb3} & \displaystyle
        %     \frac{P\tran{\feature{A}}P_2 \tran{\feature{F}} P_1}{\forbid{A}{P}
        %       \tran{\feature{B}} \forbid{A}{P_1}},\ \feature{B}\neq\feature{A} &
        %      \nombreRegla{forb4} & \displaystyle
        %     \frac{P\tran{\feature{A}}P_2 \tran{\feature{\checkmark}} P_1}{\forbid{A}{P}
        %       \tran{\checkmark} \nil}\\ 
        %     
        % 
        % 
        \nombreRegla{mand1} &  \displaystyle\frac{P\tranp{\checkmark}{p} \nil}{\mandatory{A}{P}
                \tranp{\feature{A}}{p} \checkmark}  &
        \nombreRegla{mand2} &  \displaystyle\frac{P\tranp{\feature{A}}p P_1}{\mandatory{A}{P}\tranp{\feature{A}}p P_1} \\
        \nombreRegla{mand3} &  \displaystyle\frac{P\tranp{\feature{B}}p
                P_1,\ \feature{A}\neq\feature{B}}{\mandatory{A}{P}\tranp{\feature{B}}p \mandatory{A}{P_1}}
        \\
        \multicolumn{4}{c}{\feature{A},\feature{B},\feature{C}\in\calF,\ a\in\calF\cup\{\checkmark\}}
        
        \end{array}  
        $}
        
        \noindent
        
        \linefigure
        
        \caption{Reglas para definir la semántica operacional de \fodaPA.  \label{fig:sos-rules-p}}
        
\end{figure*}


\bdfn\label{def:trantions}
  Dados los términos $P,Q\in\fodaPA$ y la característica $a\in\calF\cup\{\checkmark\}$.
  Existe una transición de $P$ to $Q$ etiquetada con el símbolo $a$ y la probabilidad $p$,
  representada por $P\tranp{a}{p}Q$,
  si puede ser deducida de las reglas descritas en la figura~\ref{fig:sos-rules-p}.  
\edfn

\todo{Hay que explicar despacio las cosas}





\todo{explicar los lemas y poner más texto}
\blem\label{lem:prob}
  Dados los términos $P,\ Q\in\fodaPA$, 
  \begin{itemize}
  \item Dada la característica  $a\in\calF\cup\{\checkmark\}$, si
    $P\tranp{a}{p}Q$ entonces $p\in(0,1]$.
  \item Dada la característica   $s\in\calF^*$, si $P\vtranp{s}{p}Q$ entonces $p\in(0,1]$.
  \end{itemize}
\elem

\blem
  Para cualquier término $P\in\fodaPA$ se cumple
  \begin{displaymath}
    \sum \{p\, |\ \exists a\in\calF,\ Q\in\fodaPA:\ P\tranp{a}{p}Q \}\in[0,1]
  \end{displaymath}
\elem

\blem\label{lem:check}
  Dados los términos $P,Q\in\fodaPA$ y la probabilidad $p\in(0,1]$, $P\tranp{\checkmark}{p}{Q}$ si y sólo si $Q=\nil$.
\elem

\bdfn
  $P\vtranp{s}{p}Q$ si y sólo si $s=a_1a_2\cdots a_n$ y $p=p_1\cdot p_2\cdot \cdots p_{n}$ y
  \begin{displaymath}
    P=P_0\tranp{a_1}{p_1}P_1\tranp{a_2}{p_2}P_2\cdots P_{n-1}\tranp{a_n}{p_n} P_n=Q
  \end{displaymath}

  $(pr,p)\in\prodp(P)$ si y sólo si $p>0$ y $p=\sum[q\ |\ P\vtranp{s\checkmark}{q} Q \y \product{s}=pr ]$
\edfn







\begin{figure}[h]



\linefigure


        
\vspace*{0.5em}

\begin{minipage}{0.3\hsize}
\centering 

\texttt{$P_{1}$}
\end{minipage}
\vspace*{1em}
\begin{minipage}{0.75\hsize}
\centering

\texttt{$P_{2}$}
\end{minipage}



\begin{minipage}{0.3\hsize}
\centering 




%$\feature{A};\ofeature{B};\checkmark \leadsto$
% \begin{minipage}{0.35\hsize}
%         \centering
\scalebox{0.7}
{
        \begin{tikzpicture}[->,>=stealth',node distance=1.5cm]
                \node[syntax] (A)   {$\feature{A};\ofeature{B};_{p}\checkmark$};
                \node[syntax] (B) [below of=A]   {$\ofeature{B};\checkmark$};
                \node[syntax] (C) [below left of=B]   {\nil};
                \node[syntax] (D) [below right of=B]   {\checkmark};
                 \node[syntax] (F) [below of=D]   {\nil};

                \path (A) edge[->] node[right, opacity=0, text opacity=1] {$(\feature{A},1)$~$\nombreRegla{feat}$} (B)
                 (B) edge[->] node[above left, opacity=0, text opacity=1] {$\begin{array}{c}\nombreRegla{ofeat2}\\(\checkmark,(1-p))\end{array}$} (C)
                 (B) edge[->] node[above right, opacity=0, text opacity=1] {$(\feature{B},p)$~$\nombreRegla{ofeat1}$} (D)
                %(C) edge[->] node[left] {$\nombreRegla{nil}$~\checkmark} (E)
                (D) edge[->] node[left, opacity=0, text opacity=1] {$\nombreRegla{tick}$~$(\checkmark,1)$} (F);
        \end{tikzpicture}
}



\end{minipage}
\begin{minipage}{0.75\hsize}
\centering





%$\feature{A};\ofeature{B};\checkmark \leadsto$
% \begin{minipage}{0.35\hsize}
%         \centering

\scalebox{0.7}{
        \begin{tikzpicture}[->,>=stealth',node distance=1.5cm]
                \node[syntax] (A)   {$\feature{A};(\ofeature{B};_{p}\checkmark \choice_{q} \ofeature{B};_{r}\checkmark)$};
                \node[syntax] (AA) [below of=A]   {$(\ofeature{B};_{p}\checkmark \choice_{q} \ofeature{B};_{r}\checkmark)$};
                \node[state,color=white] (phantom) [below of=AA]   {}; 
                %left part 
               \node[state,color=white] (B) [left of=phantom,node distance=4cm]   {};
                \node[syntax] (C) [left of=B]   {\nil};
                \node[syntax] (D) [below right of=B]   {\checkmark};
                 \node[syntax] (F) [below of=D]   {\nil};
                % right part
                \node[state, color=white] (BB) [right of=phantom,node distance=4cm]   {};
                \node[syntax] (CB) [below of=AA, node distance=4cm]   {\nil};
                % right part
                \node[syntax] (DB) [below right of=BB]   {\checkmark};
                 \node[syntax] (FB) [below of=DB]   {\nil};


                \path (A) edge[->] node[right, opacity=0, text opacity=1] {$(\feature{A},1)$~$\nombreRegla{feat}$} (AA)
                 (AA) edge[->,bend right] node[above left, opacity=0, text opacity=1] {\begin{tabular}{l}$\nombreRegla{ofeat2}$~$\nombreRegla{cho1}$\\$(\checkmark,q\cdot(1-p))$\end{tabular}} (C)
                 (AA) edge[->,bend left] node[above left, opacity=0, text opacity=1] {$(\feature{B},q\cdot p)$~$\nombreRegla{ofeat1}$~$\nombreRegla{cho1}$} (D)
                (D) edge[->] node[left, opacity=0, text opacity=1] {$\nombreRegla{tick}$~$(\checkmark,1)$} (F)
                %right part
                 (AA) edge[->] node[right, opacity=0, text opacity=1] {\begin{tabular}{l}$\nombreRegla{ofeat2}$~$\nombreRegla{cho2}$\\ $(\checkmark,(1-q)\cdot(1-r))$\end{tabular}} (CB)
                 (AA) edge[->,bend left] node[above right, opacity=0, text opacity=1] {\begin{tabular}{l}$(\feature{B},(1-q)\cdot r)$\\$\nombreRegla{ofeat1}$~$\nombreRegla{cho2}$\end{tabular}} (DB)
 
                (DB) edge[->] node[left, opacity=0, text opacity=1] {$\nombreRegla{tick}$~$(\checkmark,1)$} (FB);

        \end{tikzpicture}
}

\end{minipage}


\vspace*{0.5em}

\begin{minipage}{0.30\hsize}
\centering 

\texttt{$P_{3}$}
\end{minipage}
\begin{minipage}{0.27\hsize}
\centering

\texttt{$P_{4}$}
\end{minipage}
\begin{minipage}{0.30\hsize}
\centering

\texttt{$P_{5}$}
\end{minipage}
\vspace*{0.5em}


\begin{minipage}{0.30\hsize}
\centering 



\scalebox{0.6}{
        \begin{tikzpicture}[->,>=stealth',node distance=1.5cm]
                \node[syntax] (A)   {$\feature{A};(\feature{A};\checkmark\choice_{q} \feature{B};\checkmark)$};
                \node[syntax] (B) [below of=A]   {$(\feature{A};\checkmark\choice_{q} \feature{B};\checkmark)$};
                \node[syntax] (C) [below left of=B,node distance=3cm]   {$\checkmark$};
                \node[syntax] (D) [below of=B,node distance=3cm]   {$\checkmark$};
                 \node[syntax] (F) [below of=C]   {$\nil$};
                 \node[syntax] (H) [below of=D]   {$\nil$};


                \path (A) edge[->] node[right, opacity=0, text opacity=1] {$(\feature{A},1)$~$\nombreRegla{feat}$} (B)

                (B) edge[->] node[left, opacity=0, text opacity=1] {$\nombreRegla{cho1}$~$(\feature{A},q)$} (C)
                (B) edge[->] node[right, opacity=0, text opacity=1] {$\nombreRegla{cho2}$~$(\feature{B},1-q)$} (D)

                (C) edge[->] node[left, opacity=0, text opacity=1] {$\nombreRegla{tick}$~$(\checkmark,1)$} (F)
                (D) edge[->] node[right, opacity=0, text opacity=1] {$\nombreRegla{tick}$~$(\checkmark,1)$} (H);
        \end{tikzpicture}

}


\end{minipage}
\begin{minipage}{0.27\hsize}
\centering



\scalebox{0.6}{
        \begin{tikzpicture}[->,>=stealth',node distance=1.5cm]
                \node[syntax] (A)   {$(\feature{A};\checkmark \choice_{p} \feature{B};\nil)$};
                \node[syntax] (B) [below left of=A,node distance=3cm]   {$\checkmark$};
                \node[syntax] (C) [below right of=A,node distance=3cm]   {$\nil$};
                \node[syntax] (D) [below of=B]   {$(\nil$};

                \path (A) edge[->] node[above left, opacity=0, text opacity=1] {$\begin{array}{c}(\feature{A},p)\\ \nombreRegla{cho1}~\nombreRegla{feat}\end{array}$} (B)
                (A) edge[->] node[right, opacity=0, text opacity=1] {$\begin{array}{c}(\feature{B},1-p)\\ \nombreRegla{cho2}~\nombreRegla{feat}\end{array}$} (C)
                (B) edge[->] node[right, opacity=0, text opacity=1] {$(\checkmark,1)$~$\nombreRegla{tick}$} (D)

                ;

        \end{tikzpicture}
}

\end{minipage}
\begin{minipage}{0.30\hsize}
\centering



\scalebox{0.6}{
        \begin{tikzpicture}[->,>=stealth',node distance=1.5cm]
                \node[syntax] (A)   {$(\feature{A};\checkmark \paral \feature{B};\checkmark)$};

                \node[syntax] (B) [below left of=A,node distance=3cm]  {$(\checkmark \paral \feature{B};\checkmark)$};
                \node[syntax] (C) [below of=B]  {$(\checkmark \paral \checkmark)$};
                \node[syntax] (D) [below of=C]  {$(\nil)$};
                \node[syntax] (E) [below of=A,node distance=3cm]  {$(\feature{A}; \checkmark \paral \checkmark)$};
                \node[syntax] (F) [below of=E]  {$(\checkmark \paral \checkmark)$};
                \node[syntax] (G) [below of=F]  {$(\nil)$};


                \path (A) edge[->] node[left, opacity=0, text opacity=1] {$(\feature{A},1)$~$\nombreRegla{con1}$~$\nombreRegla{feat}$} (B)
                (A) edge[->] node[right, opacity=0, text opacity=1] {$(\feature{B},1)$~$\nombreRegla{con2}$~$\nombreRegla{feat}$} (E)
                (B) edge[->] node[left, opacity=0, text opacity=1] {$(\feature{B},1)$~$\nombreRegla{con5}$~$\nombreRegla{feat}$} (C)
                (E) edge[->] node[right, opacity=0, text opacity=1] {$(\feature{A},1)$~$\nombreRegla{con4}$~$\nombreRegla{feat}$} (F)
                (C) edge[->] node[left, opacity=0, text opacity=1] {$(\checkmark,0.5)$~$\nombreRegla{con3}$~$\nombreRegla{tick}$} (D)
                (F) edge[->] node[right, opacity=0, text opacity=1] {$(\checkmark,0.5)$~$\nombreRegla{con3}$~$\nombreRegla{tick}$} (G)

                ;

        \end{tikzpicture}
}
\review{Verificar $P_5$}
\end{minipage}
\linefigure

\caption{Ejemplos de la ejecución de la reglas de la semántica operacional del modelo probabilístico.\label{example:op}}
\end{figure}

\todo{la notación de este ejemplo est'a mal}
Los ejemplos descritos en la figura~\ref{example:op} muestran el
cálculo de la información probabilística al procesar los términos $P_1$, $P_2$, $P_3$, $P_4$ y $P_5$.
Consideremos $P_1=\feature{A};\ofeature{B};_{p}\checkmark$,
para este término se tiene que el conjunto de $\straces{N_{P_1}}$ es: $\{\langle (\feature{A},1),(\checkmark,(1-p))\rangle,\langle(\feature{A},1),(\feature{B},p)(\checkmark,1)\rangle\}$
y $\products{P_1}=\{[\feature{A}],[\feature{A}\feature{B}]\}$.
De igual manera $\products{N_{P_2}}=\{[\feature{A}],[\feature{A},\feature{B}]\}$ y la probabilidad
asociada con cada producto es calculada tomando en cuenta las distintas ramas.
También se muestra $\products{P_3}=\{[\feature{A}],[\feature{A},\feature{B}] \}$
donde $\prob{[\feature{A}],P_3]}=p$ y $\prob{[\feature{A},\feature{B}],P_3]}=1-p$.

En el siguiente término se muestra la noción del dominio probabilístico.
Considerando $P_4=(\feature{A};\checkmark \choice_{p} \feature{B};\nil)$,
se tiene que $\products{P_4}=\{[\feature{A}]\}$
donde $\prob{[\feature{A}],P]}=1$.

Finalmente consideremos $P_5=(\feature{A};\checkmark \paral \feature{B};\checkmark)$,
donde $\products{P_5}=\{[\feature{A},\feature{B}]\}$ y 
 $\prob{[\feature{A},\feature{B}],P_5}= 0.5+0.5=1$.




% \bdfn\label{dfn:equiv}
%   Let $P,Q\in\fodaPA$. We define that they are functional equivalent, written $P\equiv_{func} Q$ if
% the products derived from both \SPLs\ are the same: $\products{P}=\products{Q}$.

% We define that they are probabilistically equivalent, written $P\equiv_{prob} Q$ if
% $P\equiv_{func} Q$ and for each $[\sigma]\in \products{P}$ we have that $\prob{[\sigma],P} =\prob{[\sigma],Q}$ .
% \edfn 


% \bprop
%   Let $P,Q,R\in\fodaPA$. The following properties hold:
%   \begin{itemize}
%   \item $P\equiv_{func} P$.
%   \item $P\equiv_{prob} P$.
%   \item If $P\equiv_{func} Q$ then $Q\equiv_{func} P$.
%   \item If $P\equiv_{prob} Q$ then $Q\equiv_{prob} P$.
%   \item If $P\equiv_{func} Q$ and $Q\equiv_{func} R$ then $P\equiv_{func} R$.
%   \item If $P\equiv_{prob} Q$ and $Q\equiv_{prob} R$ then $P\equiv_{prob} R$.

%           \end{itemize}
% \eprop 

%%% Local Variables: 
%%% mode: latex
%%% TeX-master: "../../main"
%%% ispell-local-dictionary: "castellano"
%%% End: 
