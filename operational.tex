%\subsection{Operational Semantics}
%\label{sec:stat:oper}


\begin{figure*}[t]
        \linefigure

        \centering\scalebox{0.95}{%
        $
        \begin{array}{*{3}{l@{}c@{\hspace{4em}}}}
        \nombreRegla{tick} & \checkmark\tranp{\checkmark}{1}\nil &
        \nombreRegla{feat} & \feature{A};P\tranp{\feature{A}}{1}P \\
        \nombreRegla{ofeat1} & \ofeature{A};_{p}P\tranp{\feature{A}}{p}P &
        \nombreRegla{ofeat2} & \ofeature{A};_{p}P\tranp{\checkmark}{(1-p)}\nil\\
        \nombreRegla{cho1} & \displaystyle \frac{P\tranp{\feature{A}}{p} P_1}{P\choice_{q} Q\tranp{\feature{A}}{p\cdot q} P_1}&
        \nombreRegla{cho2} & \displaystyle\frac{Q\tranp{\feature{A}}{q} Q_1}{P\choice_{p} Q\tranp{\feature{A}}{(1-p)\cdot q}Q_1}\\
        \nombreRegla{con1} & \displaystyle\frac{P\tranp{\feature{A}}{p} P_1}{P\paral Q\tranp{\feature{A}}{\frac{p}{2}}P_1\paral Q} &
        \nombreRegla{con2} & \displaystyle\frac{Q\tranp{\feature{A}}{q} Q_1}{P\paral Q\tranp{\feature{A}}{\frac{q}{2}}P\paral Q_1}\\
        \nombreRegla{con3} & \displaystyle\frac{P\tranp{\checkmark}{q}\nil, Q\tranp{\checkmark}{p}\nil}{P\paral Q\tranp{\checkmark}{p\cdot q}\nil} &\\
        \nombreRegla{con4} & \displaystyle\frac{P\tranp{\feature{A}}{p} P_1, Q\tranp{\checkmark}{q}\nil}{P\paral Q\tranp{\feature{A}}{\frac{p\cdot q}{2}} P_1} &
        \nombreRegla{con5} & \displaystyle\frac{P\tranp{\checkmark}{p}\nil,Q\tranp{\feature{A}}{q} Q_1}{P\paral Q\tranp{\feature{A}}{\frac{p\cdot q}{2}} Q_1} \\
        %
        %
        \nombreRegla{req1} & \displaystyle
        \frac{P \tranp{\feature{C}}{p} P_1,\ \feature{C}\neq\feature{A}}{\require{A}{B}{P}
                \tranp{\feature{C}}{p} \require{A}{B}{P_1}} &
        \nombreRegla{req2} &  \displaystyle
        \frac{P \tranp{\feature{A}}{p} P_1}{\require{A}{B}{P}\tranp{\feature{A}}{p} \mandatory{B}{P_1}} & \\
        % \nombreRegla{req3} & \displaystyle
        % \frac{P \tranp{\feature{B}}{p} P_1}{\require{A}{B}{P}
        %   \tranp{\feature{B}}{p} P_1} &
        \nombreRegla{req3} &  \displaystyle
        \frac{P \tranp{\feature{\checkmark}}{p} \nil}{\require{A}{B}{P}
                \tranp{\feature{\checkmark}}{p} \nil}\\
        \nombreRegla{excl1} &    \displaystyle
        \frac{P \tranp{\feature{C}}{p} P_1,\ \feature{C}\neq\feature{A},\  \feature{C}\neq\feature{B}}{
                \exclude{A}{B}{P}\tranp{\feature{C}}{p} \exclude{A}{B}{P_1}} &
        \nombreRegla{excl2} & \displaystyle
        \frac{P \tranp{\feature{A}}{p} P_1}{\exclude{A}{B}{P}
                \tranp{\feature{A}}{p}\forbid{B}{P_1}} \\
        \nombreRegla{excl3} & \displaystyle
        \frac{P \tranp{\feature{B}}{p} P_1}{\exclude{A}{B}{P}\tranp{\feature{B}}{p}\forbid{A}{P_1}}
        &
        \nombreRegla{excl4} & \displaystyle
        \frac{P \tranp{\checkmark}{p} \nil}{\exclude{A}{B}{P}\tranp{\checkmark}{p}\nil}\\

        \nombreRegla{forb1} & \displaystyle
        \frac{P \tranp{\feature{B}}{p} P_1,\ \feature{B}\neq\feature{A}}{\forbid{A}{P}
                \tranp{\feature{B}}{p} \forbid{A}{P_1}} &
        \nombreRegla{forb2} & \displaystyle
        \frac{P \tranp{\checkmark}{p} \nil}{\forbid{A}{P}
                \tranp{\checkmark}{p} \nil}  \\
        %      \nombreRegla{forb3} & \displaystyle
        %     \frac{P\tran{\feature{A}}P_2 \tran{\feature{F}} P_1}{\forbid{A}{P}
        %       \tran{\feature{B}} \forbid{A}{P_1}},\ \feature{B}\neq\feature{A} &
        %      \nombreRegla{forb4} & \displaystyle
        %     \frac{P\tran{\feature{A}}P_2 \tran{\feature{\checkmark}} P_1}{\forbid{A}{P}
        %       \tran{\checkmark} \nil}\\
        %
        %
        %
        \nombreRegla{mand1} &  \displaystyle\frac{P\tranp{\checkmark}{p} \nil}{\mandatory{A}{P}
                \tranp{\feature{A}}{p} \checkmark}  &
        \nombreRegla{mand2} &  \displaystyle\frac{P\tranp{\feature{A}}p P_1}{\mandatory{A}{P}\tranp{\feature{A}}p P_1} \\
        \nombreRegla{mand3} &  \displaystyle\frac{P\tranp{\feature{B}}p
                P_1,\ \feature{A}\neq\feature{B}}{\mandatory{A}{P}\tranp{\feature{B}}p \mandatory{A}{P_1}}
        \\
        \multicolumn{4}{c}{\feature{A},\feature{B},\feature{C}\in\calF,\ a\in\calF\cup\{\checkmark\}}

        \end{array}
        $}

        \noindent

        \linefigure

        \caption{\fodaPAp\ operational semantics.  \label{fig:sos-rules}}
\end{figure*}

In Figure~\ref{fig:sos-rules} we present the set of rules formally defining  the operational behavior of
\fodaPAp. These rules essentially coincide with the ones corresponding to
\fodaPA~\cite{acl13}, but with the addition of probabilities. Next we focus
on the explanation of the role of probabilities. Rules
$\nombreRegla{tick}$ and $\nombreRegla{feat}$
show the corresponding feature with probability 1.
%
Rules  $\nombreRegla{ofeat1}$ and $\nombreRegla{ofeat2}$ deal with the
probabilistic optional feature. The feature can be chosen with probability
$p$ and can be rejected with probability  $1-p$. Let us note that both probabilities
are not null.
%
Rules $\nombreRegla{cho1}$ and $\nombreRegla{cho2}$ define the
behavior of the probabilistic choice operator. The left branch is
selected with probability~$p$ and the right one with probability~$1-p$.
%
It is important to note that the rules for the conjunction operator,
$\nombreRegla{con1}$,
$\nombreRegla{con2}$,
$\nombreRegla{con4}$ and
$\nombreRegla{con5}$,
equitably distribute the probability between both
branches, that is, $\frac{1}{2}$. We have preferred to use a simple definition of this operator, but it is easy to replace it by a more involved version of a probabilistic conjunction operator~\cite{ahk98}.
%
Rule $\nombreRegla{con3}$ requires that both branches agree on the
termination of a product. Figures~\ref{example:op1}
and~\ref{example:op2} contain some examples of the operational semantics.
%

\begin{figure}[t]
  \linefigure
  \centering
    \scalebox{0.6}{%
      \begin{tikzpicture}[->,>=stealth',node distance=1.5cm]
        \node[syntax] (A)
        {$\checkmark\paral(\fB;\checkmark\paral\fC;\checkmark)$};
        \node[syntax] (B) [below left=1cm and 4cm of A]{$\checkmark\paral(\checkmark\paral\fC;\checkmark)$};
        \node[syntax] (C) [below left=1cm and -.5cm of A]{$\checkmark\paral\fC;\checkmark$};
        \node[syntax] (D) [below right=1cm and 5cm of A]{$\checkmark\paral(\fB\checkmark\paral\checkmark)$};
        \node[syntax] (E) [below right=1cm and -.5cm of A]{$\fB;\checkmark\paral\checkmark$};

        \node[syntax] (B1) [below left=1cm and 1cm of B]{$\checkmark\paral(\checkmark\paral\checkmark)$};
        \node[syntax] (B2) [below left=1cm and -0.7cm of B]{$\checkmark\paral(\checkmark)$};
        \node[syntax] (B3) [below right=1cm and -0.7cm of B]{$(\checkmark\paral\checkmark)$};
        \node[syntax] (B4) [below right=1cm and 1cm of B]{$\checkmark$};
        \node[syntax] (D1) [below left=1cm and 1cm of D]{$\checkmark\paral(\checkmark\paral\checkmark)$};
        \node[syntax] (D2) [below left=1cm and -0.7cm of D]{$\checkmark\paral(\checkmark)$};
        \node[syntax] (D3) [below right=1cm and -0.7cm of D]{$(\checkmark\paral\checkmark)$};
        \node[syntax] (D4) [below right=1cm and 1cm of
        D]{$\checkmark$};

        \node[syntax] (C1) [below left=1cm and -.5cm of C]{$\checkmark\paral\checkmark$};
        \node[syntax] (C2) [below right=1cm and -.5cm of C]{$\checkmark$};
        \node[syntax] (E1) [below left=1cm and -.5cm of E]{$\checkmark\paral\checkmark$};
        \node[syntax] (E2) [below right=1cm and -.5cm of E]{$\checkmark$};

        \node[syntax] (B11) [below=1cm of B1] {$\nil$};
        \node[syntax] (B21) [below=1cm of B2] {$\nil$};
        \node[syntax] (B31) [below=1cm of B3] {$\nil$};
        \node[syntax] (B41) [below=1cm of B4] {$\nil$};
        \node[syntax] (D11) [below=1cm of D1] {$\nil$};
        \node[syntax] (D21) [below=1cm of D2] {$\nil$};
        \node[syntax] (D31) [below=1cm of D3] {$\nil$};
        \node[syntax] (D41) [below=1cm of D4] {$\nil$};
        \node[syntax] (C11) [below=1cm of C1] {$\nil$};
        \node[syntax] (C21) [below=1cm of C2] {$\nil$};
        \node[syntax] (E11) [below=1cm of E1] {$\nil$};
        \node[syntax] (E21) [below=1cm of E2] {$\nil$};

        \path
        (A) edge[->] node[above, opacity=0, text opacity=1]
        {$\fB,\frac{1}{4}$}
        (B);
        \path
        (A) edge[->] node[left, opacity=0, text opacity=1]
        {$\fB,\frac{1}{4}$}
        (C);
        \path (A) edge[->] node[above, opacity=0, text opacity=1]
        {$\fC,\frac{1}{4}$}
        (D);
        \path (A) edge[->] node[right, opacity=0, text opacity=1]
        {$\fC,\frac{1}{4}$}
        (E);

        \path (B) edge[->] node[left, opacity=0, text opacity=1]
        {$\fC,\frac{1}{4}$} (B1);
        \path (B) edge[->] node[left, opacity=0, text opacity=1]
        {$\fC,\frac{1}{4}$} (B2);
        \path (B) edge[->] node[left, opacity=0, text opacity=1]
        {$\fC,\frac{1}{4}$} (B3);
        \path (B) edge[->] node[right, opacity=0, text opacity=1]
        {$\fC,\frac{1}{4}$} (B4);
        \path (D) edge[->] node[left, opacity=0, text opacity=1]
        {$\fB,\frac{1}{4}$} (D1);
        \path (D) edge[->] node[left, opacity=0, text opacity=1]
        {$\fB,\frac{1}{4}$} (D2);
        \path (D) edge[->] node[left, opacity=0, text opacity=1]
        {$\fB,\frac{1}{4}$} (D3);
        \path (D) edge[->] node[right, opacity=0, text opacity=1]
        {$\fB\frac{1}{4}$} (D4);

        \path (E) edge[->] node[left, opacity=0, text opacity=1]
        {$\fB,\frac{1}{2}$} (E1);
        \path (E) edge[->] node[right, opacity=0, text opacity=1]
        {$\fB,\frac{1}{2}$} (E2);

       \path (C) edge[->] node[left, opacity=0, text opacity=1]
        {$\fC,\frac{1}{2}$} (C1);
        \path (C) edge[->] node[right, opacity=0, text opacity=1]
        {$\fC,\frac{1}{2}$} (C2);
        \path (C1) edge[->] node[right, opacity=0, text opacity=1]
        {$\checkmark$} (C11);
        \path (C2) edge[->] node[right, opacity=0, text opacity=1]
        {$\checkmark$} (C21);
        \path (E1) edge[->] node[right, opacity=0, text opacity=1]
        {$\checkmark$} (E11);
        \path (E2) edge[->] node[right, opacity=0, text opacity=1]
        {$\checkmark$} (E21);
        \path (B1) edge[->] node[right, opacity=0, text opacity=1]
        {$\checkmark$} (B11);
        \path (B2) edge[->] node[right, opacity=0, text opacity=1]
        {$\checkmark$} (B21);
        \path (B3) edge[->] node[right, opacity=0, text opacity=1]
        {$\checkmark$} (B31);
        \path (B4) edge[->] node[right, opacity=0, text opacity=1]
        {$\checkmark$} (B41);
        \path (D1) edge[->] node[right, opacity=0, text opacity=1]
        {$\checkmark$} (D11);
        \path (D2) edge[->] node[right, opacity=0, text opacity=1]
        {$\checkmark$} (D21);
        \path (D3) edge[->] node[right, opacity=0, text opacity=1]
        {$\checkmark$} (D31);
        \path (D4) edge[->] node[right, opacity=0, text opacity=1]
        {$\checkmark$} (D41);


      \end{tikzpicture}}
        \caption{Examples of the operational semantics (1/3).\label{example:op1}}
\end{figure}
\begin{figure}[t]
  \linefigure
  \begin{minipage}{.7\hsize}
    \centering
    \scalebox{0.6}
    {
      \begin{tikzpicture}[->,>=stealth',node distance=1.5cm]
        \node[syntax] (A) {$\fA;\checkmark\paral(\fB;\checkmark\paral
          \fC;\checkmark)$};
        \node[syntax] (B) [below left=2cm of A] {$\checkmark\paral(\fB;\checkmark\paral
          \fC;\checkmark)$};
        \node[syntax] (C) [below right=3.5cm and -1cm of A] {$\fA;\checkmark\paral(\checkmark\paral
          \fC;\checkmark)$};
        \node[syntax] (D) [below right=2cm of A] {$\fA;\checkmark\paral(\fB;\checkmark\paral
          \checkmark)$};
        \node[isosceles triangle, shape border rotate=90] (B1) [below=0cm of B] {Figure~\ref{example:op1}};
        \node[] (D1)
        [below=0.1cm of D] {\parbox{3cm}{\centering Similar to\\ $\fA;\checkmark\paral(\checkmark\paral
            \fC;\checkmark)$}};
        \node[]
        [below=2cm of C] {\parbox{5cm}{\centering Similar to the ones in Figure~\ref{example:op1}}};

        \node[syntax] (C1) [below left=1cm and 1.5cm of C]{$\checkmark\paral(\checkmark\paral\fC;\checkmark)$};
        \node[syntax] (C2) [below left=1cm and -0.5cm of C]{$\checkmark\paral(\fC\checkmark)$};
        \node[syntax] (C3) [below right=1cm and -1.5cm of C]{$\fA;\checkmark\paral(\checkmark\paral\checkmark)$};
        \node[syntax] (C4) [below right=1cm and 1.5cm of
        C]{$\fA;\checkmark\paral(\checkmark)$};

        % \node[syntax] (B11) [below left=1cm and 0cm of B1] {$\checkmark$};
        % \node[syntax] (B12) [below right=1cm and -0.5cm of B1] {$\checkmark\paral\checkmark$};
        % \node[syntax] (B21) [below left=1cm and 0cm of B2] {$\checkmark$};
        % \node[syntax] (B22) [below right=1cm and -0.5cm of B2] {$\checkmark\paral\checkmark$};
        % \node[syntax] (C1) [below left=1cm and -0.5cm of C] {$\checkmark\paral(\checkmark\paral
        %   \fC;\checkmark)$};
        % \node[syntax] (C2) [below right=1cm and -0.5cm of C] {$\fA;\checkmark\paral\checkmark$};
        % \node[syntax] (C11) [below left=1cm and 0cm of C1]
        % {$\checkmark$};
        % \node[syntax] (C12) [below right=1cm and -0.5cm of C1] {$\checkmark\paral\checkmark$};
        % \node[syntax] (C21) [below left=1cm and 0cm of C2] {$\checkmark$};
        % \node[syntax] (C22) [below right=1cm and -0.5cm of C2] {$\checkmark\paral\checkmark$};
        % \node[syntax] (D1) [below left=1cm and -0.5cm of D] {$\checkmark\paral(\fB;\checkmark\paral
        %   \checkmark)$};
        % \node[syntax] (D2) [below right=1cm and -0.5cm of D] {$\fA;\checkmark\paral\checkmark$};
        % \node[syntax] (D11) [below left=1cm and 0cm of D1] {$\checkmark$};
        % \node[syntax] (D12) [below right=1cm and -0.5cm of D1] {$\checkmark\paral\checkmark$};
        % \node[syntax] (D21) [below left=1cm and 0cm of D2] {$\checkmark$};
        % \node[syntax] (D22) [below right=1cm and -0.5cm of D2] {$\checkmark\paral\checkmark$};
        % \node[syntax] (B111) [below=1cm of B11] {$\nil$};
        % \node[syntax] (B121) [below=1cm of B12] {$\nil$};
        % \node[syntax] (B211) [below=1cm of B21] {$\nil$};
        % \node[syntax] (B221) [below=1cm of B22] {$\nil$};
        % \node[syntax] (C111) [below=1cm of C11] {$\nil$};
        % \node[syntax] (C121) [below=1cm of C12] {$\nil$};
        % \node[syntax] (C211) [below=1cm of C21] {$\nil$};
        % \node[syntax] (C221) [below=1cm of C22] {$\nil$};
        % \node[syntax] (D111) [below=1cm of D11] {$\nil$};
        % \node[syntax] (D121) [below=1cm of D12] {$\nil$};
        % \node[syntax] (D211) [below=1cm of D21] {$\nil$};
        % \node[syntax] (D221) [below=1cm of D22] {$\nil$};

        \path
        (A) edge[->] node[above, opacity=0, text opacity=1]
        {$\feature{A},\frac{1}{2}$}
        (B);

        \path
        (A) edge[->] node[right, opacity=0, text opacity=1]
        {$\feature{B},\frac{1}{4}$}
        (C);
        \path
        (A) edge[->] node[right, opacity=0, text opacity=1]
        {$\feature{C},\frac{1}{4}$}
        (D);
        \path
        (C) edge[->] node[right, opacity=0, text opacity=1]
        {$\feature{A},\frac{1}{4}$}
        (C1);
        \path
        (C) edge[->] node[right, opacity=0, text opacity=1]
        {$\feature{A},\frac{1}{4}$}
        (C2);
        \path
        (C) edge[->] node[right, opacity=0, text opacity=1]
        {$\feature{C},\frac{1}{4}$}
        (C3);
        \path
        (C) edge[->] node[right, opacity=0, text opacity=1]
        {$\feature{C},\frac{1}{4}$}
        (C4);
        % \path
        % (B) edge[->] node[right, opacity=0, text opacity=1]
        % {$\feature{C},\frac{1}{2}$}
        % (B1);
        % \path
        % (B) edge[->] node[right, opacity=0, text opacity=1]
        % {$\feature{B},\frac{1}{2}$}
        % (B2);
        % \path
        % (B2)
        % edge[->] node[left, opacity=0, text opacity=1]
        % {$\feature{C},\frac{1}{2}$}
        % (B21);
        % \path
        % (B2)
        % edge[->] node[right, opacity=0, text opacity=1]
        % {$\feature{C},\frac{1}{2}$}
        % (B22);
        % \path
        % (B1)
        % edge[->] node[left, opacity=0, text opacity=1]
        % {$\feature{B},\frac{1}{2}$}
        % (B11);
        % \path
        % (B1)
        % edge[->] node[right, opacity=0, text opacity=1]
        % {$\feature{B},\frac{1}{2}$}
        % (B12);
        % \path
        % (C) edge[->] node[right, opacity=0, text opacity=1]
        % {$\feature{A},\frac{1}{2}$}
        % (C1);
        % \path
        % (C) edge[->] node[right, opacity=0, text opacity=1]
        % {$\feature{C},\frac{1}{2}$}
        % (C2);
        % \path
        % (C2)
        % edge[->] node[left, opacity=0, text opacity=1]
        % {$\feature{A},\frac{1}{2}$}
        % (C21);
        % \path
        % (C2)
        % edge[->] node[right, opacity=0, text opacity=1]
        % {$\feature{A},\frac{1}{2}$}
        % (C22);
        % \path
        % (C1)
        % edge[->] node[left, opacity=0, text opacity=1]
        % {$\feature{C},\frac{1}{2}$}
        % (C11);
        % \path
        % (C1)
        % edge[->] node[right, opacity=0, text opacity=1]
        % {$\feature{C},\frac{1}{2}$}
        % (C12);
        % \path
        % (D) edge[->] node[right, opacity=0, text opacity=1]
        % {$\feature{A},\frac{1}{2}$}
        % (D1);
        % \path
        % (D) edge[->] node[right, opacity=0, text opacity=1]
        % {$\feature{B},\frac{1}{2}$}
        % (D2);
        % \path
        % (D2)
        % edge[->] node[left, opacity=0, text opacity=1]
        % {$\feature{A},\frac{1}{2}$}
        % (D21);
        % \path
        % (D2)
        % edge[->] node[right, opacity=0, text opacity=1]
        % {$\feature{A},\frac{1}{2}$}
        % (D22);
        % \path
        % (D1)
        % edge[->] node[left, opacity=0, text opacity=1]
        % {$\feature{B},\frac{1}{2}$}
        % (D11);
        % \path
        % (D1)
        % edge[->] node[right, opacity=0, text opacity=1]
        % {$\feature{B},\frac{1}{2}$}
        % (D12);
        % \path
        % (B11)
        % edge[->] node[right, opacity=0, text opacity=1]
        % {$\checkmark,1$}
        % (B111);
        % \path
        % (B12)
        % edge[->] node[right, opacity=0, text opacity=1]
        % {$\checkmark,1$}
        % (B121);
        % \path
        % (B21)
        % edge[->] node[right, opacity=0, text opacity=1]
        % {$\checkmark,1$}
        % (B211);
        % \path
        % (B22)
        % edge[->] node[right, opacity=0, text opacity=1]
        % {$\checkmark,1$}
        % (B221);
        % \path
        % (C11)
        % edge[->] node[right, opacity=0, text opacity=1]
        % {$\checkmark,1$}
        % (C111);
        % \path
        % (C12)
        % edge[->] node[right, opacity=0, text opacity=1]
        % {$\checkmark,1$}
        % (C121);
        % \path
        % (C21)
        % edge[->] node[right, opacity=0, text opacity=1]
        % {$\checkmark,1$}
        % (C211);
        % \path
        % (C22)
        % edge[->] node[right, opacity=0, text opacity=1]
        % {$\checkmark,1$}
        % (C221);
        % \path
        % (D11)
        % edge[->] node[right, opacity=0, text opacity=1]
        % {$\checkmark,1$}
        % (D111);
        % \path
        % (D12)
        % edge[->] node[right, opacity=0, text opacity=1]
        % {$\checkmark,1$}
        % (D121);
        % \path
        % (D21)
        % edge[->] node[right, opacity=0, text opacity=1]
        % {$\checkmark,1$}
        % (D211);
        % \path
        % (D22)
        % edge[->] node[right, opacity=0, text opacity=1]
        % {$\checkmark,1$}
        % (D221)
        % ;
      \end{tikzpicture}
    }
  \end{minipage}
  \begin{minipage}{0.25\hsize}
                \centering
                \bigskip
                %$\feature{A};\ofeature{B};\checkmark \leadsto$
                % \begin{minipage}{0.35\hsize}
                %         \centering
                \scalebox{0.6}
                {
                        \begin{tikzpicture}[->,>=stealth',node distance=1.5cm]
                        \node[syntax] (A)   {$\feature{A};\ofeature{B};_{p}(\feature{C};\ofeature{D};_{q}\checkmark)$};
                        \node[syntax] (B) [below=1.5cm of A]   {$\ofeature{B};_{p}(\feature{C};\ofeature{D};_{q}\checkmark)$};
                        \node[syntax] (C) [below left=1.5cm and 0cm of B]   {$\nil$};
                        \node[syntax] (D) [below=1.5cm of B]   {$\feature{C};\ofeature{D};_{q}\checkmark$};
                        \node[syntax] (F) [below=1.5cm of D]   {$\ofeature{D};_{q}\checkmark$};
                        \node[syntax] (G) [below left=1.5cm and 0cm of F]   {$\nil$};
                        \node[syntax] (H) [below=1.5cm of F] {$\checkmark$};
                        \node[syntax] (I) [below=1.5cm of H]   {$\nil$};

                        \path (A) edge[->] node[right, opacity=0, text opacity=1]
                        {$\feature{A},1$}
                        (B)
                        (B) edge[->] node[above left, opacity=0, text opacity=1]
                        {$\checkmark,1-p$}
                        (C)
                        (B) edge[->] node[right, opacity=0, text opacity=1]
                        {$\feature{B},p$}
                        (D)
                        (D) edge[->] node[right, opacity=0, text opacity=1]
                        {$\feature{C},1$}
                        (F)
                        (F) edge[->] node[above left, opacity=0, text opacity=1]
                        {$\checkmark,1-q$}
                        (G)
                        (F) edge[->] node[right, opacity=0, text opacity=1]
                        {$\feature{D},q$}
                        (H)
                        (H) edge[->] node[right, opacity=0, text opacity=1]
                        {$\checkmark$}
                        (I);
                        \end{tikzpicture}
                }
        \end{minipage}
        \vspace{0.5cm}
        \linefigure

        \caption{Examples of the operational semantics (2/3).\label{example:op2}}
\end{figure}
\begin{figure}[t]
  \linefigure
  \centering
  \begin{minipage}{0.4\hsize}
    \centering
    % $\feature{A};\ofeature{B};\checkmark \leadsto$
    % \begin{minipage}{0.35\hsize}
    %   \centering
    \scalebox{0.6}{
      \begin{tikzpicture}[->,>=stealth',node distance=1.5cm]
        \node[syntax] (A)   {$\feature{A};\ofeature{B};_{p}(\feature{C};(\feature{D};\checkmark\choice_{q}(\ofeature{E};_{r}\checkmark)))$};
        \node[syntax] (B) [below=1.5cm of A]   {$\ofeature{B};_{p}(\feature{C};(\feature{D};\checkmark\choice_{q}(\ofeature{E};_{r}\checkmark))$};
        \node[syntax] (C) [below right=1.5cm and 0cm of B]   {$\nil$};
        \node[syntax] (D) [below=1.5cm of B]   {$\feature{C};(\feature{D};\checkmark\choice_{q}(\ofeature{E};_{r}\checkmark))$};
        \node[syntax] (F) [below=1.5cm of D]   {$\feature{D};\checkmark\choice_{q}(\ofeature{E};_{r}\checkmark)$};
        \node[syntax] (G) [below left=1.5cm and 0.5cm of F]   {$\checkmark$};
        \node[syntax] (H) [right=2.5cm of G] {$\checkmark$};
        \node[syntax] (I) [below right=1.5cm and 1cm of F]   {$\nil$};
        \node[syntax] (J) [below=1.5cm of G]   {$\nil$};
        \node[syntax] (K) [below=1.5cm of H]   {$\nil$};


        \path (A) edge[->] node[right, opacity=0, text opacity=1]
        {$\feature{A}$}
        (B)
        (B) edge[->] node[above right, opacity=0, text opacity=1]
        {$\checkmark,1-p$}
        (C)
        (B) edge[->] node[left, opacity=0, text opacity=1]
        {$\feature{B},p$}
        (D)
        (D) edge[->] node[right, opacity=0, text opacity=1]
        {$\feature{C},1$}
        (F)
        (F) edge[->] node[above left, opacity=0, text opacity=1]
        {$\feature{D},q$}
        (G)
        (F) edge[->] node[left, near end, opacity=0, text opacity=1]
        {$\feature{E}, (1-q)r$}
        (H)
        (F) edge[->] node[above right, opacity=0, text opacity=1]
        {$\checkmark, (1-q)(1-r)$}
        (I)
        (G) edge[->] node[right, opacity=0, text opacity=1]
        {$\checkmark,1$}
        (J)
        (H) edge[->] node[right, opacity=0, text opacity=1]
        {$\checkmark,1$}
        (K)

        ;
      \end{tikzpicture}
    }
  \end{minipage}
  \begin{minipage}{0.4\hsize}
    % $\feature{A};\ofeature{B};\checkmark \leadsto$
                % \begin{minipage}{0.35\hsize}
                %   \centering
    \scalebox{0.62}{
      \begin{tikzpicture}[->,>=stealth',node distance=1.5cm]
        \node[syntax] (A)   {$\ofeature{A};_{p}\feature{B};(\ofeature{C};_{q}\checkmark\paral\feature{D};\checkmark)$};
        \node[syntax] (B) [below left=1.5cm and 1.5cm of A]   {$\nil$};
        \node[syntax] (C) [below=1.5cm of A]   {$\feature{B};(\ofeature{C};_{q}\checkmark\paral\feature{D};\checkmark)$};

        \node[syntax] (D) [below=1.5cm of C]   {$\ofeature{C};_{q}\checkmark\paral\feature{D};\checkmark$};

        \node[syntax] (F) [below left=1.5cm and 1cm of D]   {$\checkmark\paral\feature{D};\checkmark$};

        \node[syntax] (G) [below=1.5cm of D]
        {$\ofeature{C};_{q}\checkmark\paral\checkmark$};

        \node[syntax] (D3) [below right=1.5cm and 1cm of D]
        {$\checkmark$};
        \node[syntax] (D31) [below right=1.5cm and 1cm of D3]
        {$\checkmark$};

        \node[syntax] (H) [below=1.5cm of F]   {$\checkmark\paral\checkmark$};

        \node[syntax] (I) [below=1.5cm of G]   {$\checkmark\paral\checkmark$};

        \node[syntax] (J) [below right=1.5cm and 1cm of G]
        {$\nil$};

        \node[syntax] (K) [below=1.5cm of H]   {$\nil$};
        \node[syntax] (L) [below=1.5cm of I]   {$\nil$};


        \path (A) edge[->] node[above left, opacity=0, text opacity=1]
        {$\checkmark, (1-p)$}
        (B)
        (A) edge[->] node[right, opacity=0, text opacity=1]
        {$\feature{A},p$}
        (C)
        (C) edge[->] node[right, opacity=0, text opacity=1]
        {$\feature{B},1$}
        (D)
        (D) edge[->] node[above left, opacity=0, text opacity=1]
        {$\feature{C}, \frac{q}{2}$}
        (F)
        (D) edge[->] node[right, opacity=0, text opacity=1]
        {$\feature{D},\frac{1}{2}$}
        (G)
        (F) edge[->] node[left, opacity=0, text opacity=1]
        {$\feature{D},1$}
        (H)
        (G) edge[->] node[left, opacity=0, text opacity=1]
        {$\feature{C},q$}
        (I)
        (G) edge[->] node[right, opacity=0, text opacity=1]
        {$\checkmark,1-q$}
        (J)
        (H) edge[->] node[right, opacity=0, text opacity=1]
        {$\checkmark,1$}
        (K)
        (I) edge[->] node[right, opacity=0, text opacity=1]
        {$\checkmark,1$}
        (L)
        (D) edge[->] node[right, opacity=0, text opacity=1]
        {$\fD,\frac{1-q}{2}$}
        (D3)
        (D3) edge[->] node[right, opacity=0, text opacity=1]
        {$\checkmark,1$}
        (D31)
        ;
      \end{tikzpicture}
    }
  \end{minipage}
  \vspace{0.5cm}
  \linefigure

  \caption{Examples of the operational semantics (3/3).\label{example:op3}}
\end{figure}

%%% Local Variables:
%%% mode: latex
%%% TeX-master: "main"
%%% End:


We use \emph{multisets} of transitions to consider different occurrences of the same transition. Thus, if a transition can be derived in several ways, then each derivation generates a different
instance of this transition~\cite{h96}. For example, let us consider the
term
$P=\feature{A};\checkmark \choice_{\frac12} \feature{A};\checkmark$. If we were not careful, then we would have
the transition $P\tranp{\feature{A}}{\frac12}\checkmark$ only once, while we
should have this transition twice. So, if a transition can be derived in
several ways, then we consider that each derivation generates a
different instance. In particular, we will later consider multisets of computations as well.
%
We will use the delimiters~$\lbag$ and~$\rbag$ to denote
multisets and $\uplus$ to denote the union of multisets.

The following result, whose proof is immediate, shows that successful termination leads to $\nil$.

\blem\label{lem:check}
Let $P,Q\in\fodaPAp$ and $p\in(0,1]$. We have $P\tranp{\checkmark}{p}{Q}$ if and only if $Q=\nil$.
\elem

Next we present some notions associated with the composition of consecutive transitions.

\bdfn\label{def:trtrantions} Let $P,Q\in\fodaPAp$. We write  $P\vtranp{s}{p}Q$ if there exists a sequence of consecutive transitions
\begin{displaymath}
    P=P_0\tranp{a_1}{p_1}P_1\tranp{a_2}{p_2}P_2\cdots P_{n-1}\tranp{a_n}{p_n} P_n=Q
\end{displaymath}
where $n\geq 0$, $s=a_1a_2\cdots a_n$ and $p=p_1\cdot p_2\cdot \cdots
p_{n}$. We say that $s$ is a trace of~$P$.

Let $s\in\calF^*$ be a trace of $P$. We define the product
$\product{s}\subseteq\calF$ as the set consisting of all features belonging to
$s$.

Let $P\in\fodaPAp$. We define the set of probabilistic products of $P$, denoted by $\prodp(P)$, as the set
\begin{displaymath}
 \prodp(P) = \{(pr,p)\ |\ p>0 \wedge p=\sum\lbag q\ |\
  P\vtranp{s\checkmark}{q} Q \y \product{s}=pr \rbag\}
\end{displaymath}
\item We define the total probability of $P$, denoted by $\ham(P)$, as the value $\sum\lbag p\ |\ \exists pr: (pr,p)\in\prodp(P)\rbag$. In addition, we define $\waste(P) = 1-\ham(P)$.
\edfn

The following result shows some properties, concerning probabilities, of the operational semantics. In particular, we have that the probability of (sequences of) transitions is greater than zero and that the probabilities of products belong to $[0,1]$.

\blem\label{lem:sum:prob}
  Let  $P,\ Q\in\fodaPAp$.We have the following results.
  \begin{enumerate}
  \item If $P\tranp{\feature{A}}{p}Q$ then $p\in(0,1]$.
        If $P\vtranp{s}{p}Q$ then $p\in(0,1]$.
  \item
    $\sum \lbag p\, |\ \exists \feature{A}\in\calF,\ Q\in\fodaPAp:\
    P\tranp{\feature{A}}{p}Q \rbag\in[0,1]$.
  \item
    $\sum \lbag p\, |\ \exists s\in\calF^*,\ Q\in\fodaPAp:\
    P\tranp{s\checkmark}{p}Q \rbag\in[0,1]$.
  \item $\ham(P)\in [0,1]$.
  \end{enumerate}
\elem


%%% Local Variables:
%%% mode: latex
%%% TeX-master: "main"
%%% End:
