%\subsection{Operational Semantics}
%\label{sec:stat:oper}


\begin{figure*}[h]
        \linefigure

        \centering\scalebox{0.95}{%
        $
        \begin{array}{*{3}{l@{}c@{\hspace{4em}}}}
        \nombreRegla{tick} & \checkmark\tranp{\checkmark}{1}\nil &
        \nombreRegla{feat} & \feature{A};P\tranp{\feature{A}}{1}P \\
        \nombreRegla{ofeat1} & \ofeature{A};_{p}P\tranp{\feature{A}}{p}P &
        \nombreRegla{ofeat2} & \ofeature{A};_{p}P\tranp{\checkmark}{(1-p)}\nil\\
        \nombreRegla{cho1} & \displaystyle \frac{P\tranp{\feature{A}}{p} P_1}{P\choice_{q} Q\tranp{\feature{A}}{p\cdot q} P_1}&
        \nombreRegla{cho2} & \displaystyle\frac{Q\tranp{\feature{A}}{q} Q_1}{P\choice_{p} Q\tranp{\feature{A}}{(1-p)\cdot q}Q_1}\\
        \nombreRegla{con1} & \displaystyle\frac{P\tranp{\feature{A}}{p} P_1}{P\paral Q\tranp{\feature{A}}{\frac{p}{2}}P_1\paral Q} &
        \nombreRegla{con2} & \displaystyle\frac{Q\tranp{\feature{A}}{q} Q_1}{P\paral Q\tranp{\feature{A}}{\frac{q}{2}}P\paral Q_1}\\
        \nombreRegla{con3} & \displaystyle\frac{P\tranp{\checkmark}{q}\nil, Q\tranp{\checkmark}{p}\nil}{P\paral Q\tranp{\checkmark}{p\cdot q}\nil} &\\
        \nombreRegla{con4} & \displaystyle\frac{P\tranp{\feature{A}}{p} P_1, Q\tranp{\checkmark}{q}\nil}{P\paral Q\tranp{\feature{A}}{\frac{p\cdot q}{2}} P_1} &
        \nombreRegla{con5} & \displaystyle\frac{P\tranp{\checkmark}{p}\nil,Q\tranp{\feature{A}}{q} Q_1}{P\paral Q\tranp{\feature{A}}{\frac{p\cdot q}{2}} Q_1} \\
        %
        %
        \nombreRegla{req1} & \displaystyle
        \frac{P \tranp{\feature{C}}{p} P_1,\ \feature{C}\neq\feature{A}}{\require{A}{B}{P}
                \tranp{\feature{C}}{p} \require{A}{B}{P_1}} &
        \nombreRegla{req2} &  \displaystyle
        \frac{P \tranp{\feature{A}}{p} P_1}{\require{A}{B}{P}\tranp{\feature{A}}{p} \mandatory{B}{P_1}} & \\
        % \nombreRegla{req3} & \displaystyle
        % \frac{P \tranp{\feature{B}}{p} P_1}{\require{A}{B}{P}
        %   \tranp{\feature{B}}{p} P_1} &
        \nombreRegla{req3} &  \displaystyle
        \frac{P \tranp{\feature{\checkmark}}{p} \nil}{\require{A}{B}{P}
                \tranp{\feature{\checkmark}}{p} \nil}\\
        \nombreRegla{excl1} &    \displaystyle
        \frac{P \tranp{\feature{C}}{p} P_1,\ \feature{C}\neq\feature{A},\  \feature{C}\neq\feature{B}}{
                \exclude{A}{B}{P}\tranp{\feature{C}}{p} \exclude{A}{B}{P_1}} &
        \nombreRegla{excl2} & \displaystyle
        \frac{P \tranp{\feature{A}}{p} P_1}{\exclude{A}{B}{P}
                \tranp{\feature{A}}{p}\forbid{B}{P_1}} \\
        \nombreRegla{excl3} & \displaystyle
        \frac{P \tranp{\feature{B}}{p} P_1}{\exclude{A}{B}{P}\tranp{\feature{B}}{p}\forbid{A}{P_1}}
        &
        \nombreRegla{excl4} & \displaystyle
        \frac{P \tranp{\checkmark}{p} \nil}{\exclude{A}{B}{P}\tranp{\checkmark}{p}\nil}\\

        \nombreRegla{forb1} & \displaystyle
        \frac{P \tranp{\feature{B}}{p} P_1,\ \feature{B}\neq\feature{A}}{\forbid{A}{P}
                \tranp{\feature{B}}{p} \forbid{A}{P_1}} &
        \nombreRegla{forb2} & \displaystyle
        \frac{P \tranp{\checkmark}{p} \nil}{\forbid{A}{P}
                \tranp{\checkmark}{p} \nil}  \\
        %      \nombreRegla{forb3} & \displaystyle
        %     \frac{P\tran{\feature{A}}P_2 \tran{\feature{F}} P_1}{\forbid{A}{P}
        %       \tran{\feature{B}} \forbid{A}{P_1}},\ \feature{B}\neq\feature{A} &
        %      \nombreRegla{forb4} & \displaystyle
        %     \frac{P\tran{\feature{A}}P_2 \tran{\feature{\checkmark}} P_1}{\forbid{A}{P}
        %       \tran{\checkmark} \nil}\\
        %
        %
        %
        \nombreRegla{mand1} &  \displaystyle\frac{P\tranp{\checkmark}{p} \nil}{\mandatory{A}{P}
                \tranp{\feature{A}}{p} \checkmark}  &
        \nombreRegla{mand2} &  \displaystyle\frac{P\tranp{\feature{A}}p P_1}{\mandatory{A}{P}\tranp{\feature{A}}p P_1} \\
        \nombreRegla{mand3} &  \displaystyle\frac{P\tranp{\feature{B}}p
                P_1,\ \feature{A}\neq\feature{B}}{\mandatory{A}{P}\tranp{\feature{B}}p \mandatory{A}{P_1}}
        \\
        \multicolumn{4}{c}{\feature{A},\feature{B},\feature{C}\in\calF,\ a\in\calF\cup\{\checkmark\}}

        \end{array}
        $}

        \noindent

        \linefigure

        \caption{\fodaPAp\ operational semantics.  \label{fig:sos-rules}}


\end{figure*}

In Figure~\ref{fig:sos-rules} we present the set of rules formally defining  the operational behavior of
\fodaPAp. These rules essentially coincide with the ones corresponding
\fodaPA~\cite{acl13}, but with the addition of probabilities. Next we focus 
on the explanation of the role of probabilities. Rules
$\nombreRegla{tick}$ and $\nombreRegla{feat}$
show the corresponding feature with probability 1.
%
Rules  $\nombreRegla{ofeat1}$ and $\nombreRegla{ofeat2}$ deal with the
probabilistic optional feature. The feature can be chosen with probability
$p$ and can be rejected with probability  $1-p$. Let us note that both probabilities
are not null.
%
Rules $\nombreRegla{cho1}$ and $\nombreRegla{cho2}$ define the
behavior of the probabilistic choice operator. The left branch is
selected with probability $p$ and the right with probability $1-p$.
%
It is important to note rules for the conjunction operator
(Rules $\nombreRegla{con1}$ to $\nombreRegla{con4}$)
distribute the probability equitably between both
branches, that is, $\frac{1}{2}$.
%
Rule $\nombreRegla{con5}$ requires that both branches agree on the
termination of a product.
%

\blem\label{lem:check}
  Dados los términos $P,Q\in\fodaPA$ y la probabilidad $p\in(0,1]$, $P\tranp{\checkmark}{p}{Q}$ si y sólo si $Q=\nil$.
\elem


\begin{itemize}
\item Hay que introducir la notatición de multiconjuntos
\item Hay que explicar la necesidad de los multiconjuntos.
\item $\ham(P)$ es lo que suman las probabilidades de los productos
  del término.
\end{itemize}

\bdfn\label{def:trtrantions}
\begin{enumerate}
\item $P\vtranp{s}{p}Q$ iff si $s=a_1a_2\cdots a_n$ y $p=p_1\cdot
  p_2\cdot \cdots p_{n}$ and
  \begin{displaymath}
    P=P_0\tranp{a_1}{p_1}P_1\tranp{a_2}{p_2}P_2\cdots P_{n-1}\tranp{a_n}{p_n} P_n=Q
  \end{displaymath}
\item
  $(pr,p)\in\prodp(P)$ si y sólo si $p>0$ y $p=\sum\lbag q\ |\
  P\vtranp{s\checkmark}{q} Q \y \product{s}=pr \rbag$
\item $\ham(P)=\rbag p\ |\ \exists pr: (pr,p)\in\prodp(P)$
\item $\waste(P) = 1-\ham(P)$
\end{enumerate}

\edfn


\begin{itemize}
\item Si tengo una traza su probabilidad es mayor que 0.
\item La suma de las probabilidades de todos los productos están en el
  intervalo [0,1].
\end{itemize}



\blem\label{lem:sum:prob}
  Let  $P,\ Q\in\fodaPAp$,
  \begin{enumerate}
  \item $P\tranp{\feature{A}}{p}Q$ then $p\in(0,1]$.
  \item $P\vtranp{s}{p}Q$ then $p\in(0,1]$.
  \item
    $\sum \lbag p\, |\ \exists \feature{A}\in\calF,\ Q\in\fodaPA:\
    P\tranp{\feature{A}}{p}Q \rbag\in[0,1]$

  \item
    $\sum \lbag p\, |\ \exists \feature{s}\in\calF^*,\ Q\in\fodaPA:\
    P\tranp{\feature{s}}{p}Q \rbag\in[0,1]$
  \item
    $\ham(P)\in [0,1]$
  \end{enumerate}
\elem





%%% Local Variables:
%%% mode: latex
%%% TeX-master: "main"
%%% End:
