\section{Semántica operacional}
\label{sec:stat:oper}

Las reglas de la semántica operacional de $\fodaPAp$ está
basada en las reglas descritas en la figura~\ref{fig:sos-rules},
sin embargo, estas reglas deben  ser modificadas y extendidas para permitir
representar probabilidades dentro del nuevo modelo.

\begin{figure*}[h]
        \linefigure
        
        \centering\scalebox{0.95}{%
        $
        \begin{array}{*{3}{l@{}c@{\hspace{4em}}}}
        \nombreRegla{tick} & \checkmark\tranp{\checkmark}{1}\nil &
        \nombreRegla{feat} & \feature{A};P\tranp{\feature{A}}{1}P \\
        \nombreRegla{ofeat1} & \ofeature{A};_{p}P\tranp{\feature{A}}{p}P &
        \nombreRegla{ofeat2} & \ofeature{A};_{p}P\tranp{\checkmark}{(1-p)}\nil\\
        \nombreRegla{cho1} & \displaystyle \frac{P\tranp{\feature{A}}{p} P_1}{P\choice_{q} Q\tranp{\feature{A}}{p\cdot q} P_1}&  
        \nombreRegla{cho2} & \displaystyle\frac{Q\tranp{\feature{A}}{q} Q_1}{P\choice_{p} Q\tranp{\feature{A}}{(1-p)\cdot q}Q_1}\\
        \nombreRegla{con1} & \displaystyle\frac{P\tranp{\feature{A}}{p} P_1}{P\paral Q\tranp{\feature{A}}{\frac{p}{2}}P_1\paral Q} &
        \nombreRegla{con2} & \displaystyle\frac{Q\tranp{\feature{A}}{q} Q_1}{P\paral Q\tranp{\feature{A}}{\frac{q}{2}}P\paral Q_1}\\
        \nombreRegla{con3} & \displaystyle\frac{P\tranp{\checkmark}{q}\nil, Q\tranp{\checkmark}{p}\nil}{P\paral Q\tranp{\checkmark}{p\cdot q}\nil} &\\
        \nombreRegla{con4} & \displaystyle\frac{P\tranp{\feature{A}}{p} P_1, Q\tranp{\checkmark}{q}\nil}{P\paral Q\tranp{\feature{A}}{\frac{p\cdot q}{2}} P_1} &
        \nombreRegla{con5} & \displaystyle\frac{P\tranp{\checkmark}{p}\nil,Q\tranp{\feature{A}}{q} Q_1}{P\paral Q\tranp{\feature{A}}{\frac{p\cdot q}{2}} Q_1} \\  
        %
        %
        \nombreRegla{req1} & \displaystyle
        \frac{P \tranp{\feature{C}}{p} P_1,\ \feature{C}\neq\feature{A}}{\require{A}{B}{P}
                \tranp{\feature{C}}{p} \require{A}{B}{P_1}} &
        \nombreRegla{req2} &  \displaystyle
        \frac{P \tranp{\feature{A}}{p} P_1}{\require{A}{B}{P}\tranp{\feature{A}}{p} \mandatory{B}{P_1}} & \\
        % \nombreRegla{req3} & \displaystyle
        % \frac{P \tranp{\feature{B}}{p} P_1}{\require{A}{B}{P}
        %   \tranp{\feature{B}}{p} P_1} & 
        \nombreRegla{req3} &  \displaystyle  
        \frac{P \tranp{\feature{\checkmark}}{p} \nil}{\require{A}{B}{P}
                \tranp{\feature{\checkmark}}{p} \nil}\\
        \nombreRegla{excl1} &    \displaystyle
        \frac{P \tranp{\feature{C}}{p} P_1,\ \feature{C}\neq\feature{A},\  \feature{C}\neq\feature{B}}{
                \exclude{A}{B}{P}\tranp{\feature{C}}{p} \exclude{A}{B}{P_1}} &
        \nombreRegla{excl2} & \displaystyle
        \frac{P \tranp{\feature{A}}{p} P_1}{\exclude{A}{B}{P}
                \tranp{\feature{A}}{p}\forbid{B}{P_1}} \\
        \nombreRegla{excl3} & \displaystyle
        \frac{P \tranp{\feature{B}}{p} P_1}{\exclude{A}{B}{P}\tranp{\feature{B}}{p}\forbid{A}{P_1}}
        &
        \nombreRegla{excl4} & \displaystyle
        \frac{P \tranp{\checkmark}{p} \nil}{\exclude{A}{B}{P}\tranp{\checkmark}{p}\nil}\\
        
        \nombreRegla{forb1} & \displaystyle
        \frac{P \tranp{\feature{B}}{p} P_1,\ \feature{B}\neq\feature{A}}{\forbid{A}{P}
                \tranp{\feature{B}}{p} \forbid{A}{P_1}} &
        \nombreRegla{forb2} & \displaystyle
        \frac{P \tranp{\checkmark}{p} \nil}{\forbid{A}{P}
                \tranp{\checkmark}{p} \nil}  \\
        %      \nombreRegla{forb3} & \displaystyle
        %     \frac{P\tran{\feature{A}}P_2 \tran{\feature{F}} P_1}{\forbid{A}{P}
        %       \tran{\feature{B}} \forbid{A}{P_1}},\ \feature{B}\neq\feature{A} &
        %      \nombreRegla{forb4} & \displaystyle
        %     \frac{P\tran{\feature{A}}P_2 \tran{\feature{\checkmark}} P_1}{\forbid{A}{P}
        %       \tran{\checkmark} \nil}\\ 
        %     
        % 
        % 
        \nombreRegla{mand1} &  \displaystyle\frac{P\tranp{\checkmark}{p} \nil}{\mandatory{A}{P}
                \tranp{\feature{A}}{p} \checkmark}  &
        \nombreRegla{mand2} &  \displaystyle\frac{P\tranp{\feature{A}}p P_1}{\mandatory{A}{P}\tranp{\feature{A}}p P_1} \\
        \nombreRegla{mand3} &  \displaystyle\frac{P\tranp{\feature{B}}p
                P_1,\ \feature{A}\neq\feature{B}}{\mandatory{A}{P}\tranp{\feature{B}}p \mandatory{A}{P_1}}
        \\
        \multicolumn{4}{c}{\feature{A},\feature{B},\feature{C}\in\calF,\ a\in\calF\cup\{\checkmark\}}
        
        \end{array}  
        $}
        
        \noindent
        
        \linefigure
        
        \caption{Reglas para definir la semántica operacional de \fodaPA.  \label{fig:sos-rules-p}}


        
\end{figure*}

Las reglas descritas en la figura \ref{fig:sos-rules-p} describen el comportamiento
del sistema al procesar términos siguiendo la extensión
probabilística \fodaPAp. Como se puede observar, la definición de estas reglas parte
de la definición de figura \ref{fig:sos-rules}, sin embargo, las reglas de la 
extensión probabilística permiten modificar la información relacionada a las
probabilidades a medida que son procesadas las característica.

Las reglas $\nombreRegla{tick}$ y $\nombreRegla{feat}$ muestran
que al procesar la característica $\checkmark$, $\feature{A}$ y $\feature{A}$ respectivamente,
podrá realizarse siempre con probabilidad 1 y que debe ser procesada la característica
correspondiente.
%
Las transiciones $\nombreRegla{ofeat1}$ y $\nombreRegla{ofeat2}$ permiten procesar el
operador opcional, si
es procesada la característica se computa con
probabilidad $p$ y en caso contrario $1-p$
respectivamente.
%
El operador de selección única ($\choice$)
está compuesto por dos reglas semánticas
$\nombreRegla{cho1}$ y $\nombreRegla{cho2}$,
para el caso de la regla $\nombreRegla{cho1}$
se calcula la probabilidad multiplicando
la probabilidad de ambos
lados del operador ($p.q$) y para la regla $\nombreRegla{cho2}$
al ser un operador simétrico es multiplicado 1 menos
la probabilidad de la característica a procesar
por la probabilidad acumulada del término del otro
lado del operador ($(1-p).q$).
%
Los operadores de conjunción $\nombreRegla{con1}$ y
$\nombreRegla{con2}$ 
procesan el operador de paralelo siempre que
puedan seguirse procesando características, en donde,
deber ejecutarse ambos lados de la conjunción
la probabilidad de procesarlos es la mitad ($\frac{p}{2}$
o $\frac{q}{2}$) dependiendo de que lado del paralelo
se esté ejecutando.
%
La regla $\nombreRegla{con3}$
permite procesar la característica $\checkmark$
cuando pueda ser procesada en ambos lados del paralelo
y su probabilidad será la multiplicación  de las
probabilidades de ambos lados ($p.q$).
%
Las reglas $\nombreRegla{con4}$ y $\nombreRegla{con5}$ permiten
procesar el operador de conjunción si alguno de los extremos del término puede generar
la característica $\checkmark$ y su probabilidad
será la mitad de la multiplicación de las probabilidades de ambos lados
del operador ($\frac{p.q}{2}$).
%
Todos los operadores restantes
($\nombreRegla{req1}$, $\nombreRegla{req2}$,
$\nombreRegla{req3}$,  $\nombreRegla{excl1}$,
$\nombreRegla{excl2}$, $\nombreRegla{excl3}$,
$\nombreRegla{excl4}$, $\nombreRegla{forb1}$,
$\nombreRegla{forb2}$, $\nombreRegla{mand1}$,
$\nombreRegla{mand2}$ y $\nombreRegla{mand3}$)
no intervienen ni modifican la probabilidad
de los elementos sintácticos al ser procesados.

A continuación serán descritas las definiciones correspondientes
al procesamiento de las probabilidades.

\bdfn\label{def:trantions}
  Dados los términos $P,Q\in\fodaPA$ y la característica $\feature{A}\in\calF\cup\{\checkmark\}$.
  Existe una transición de $P$ a $Q$ etiquetada
  con el símbolo $\feature{A}$ y la probabilidad $p$,
  representada por $P\tranp{\feature{A}}{p}Q$,
  si puede ser deducida de las reglas descritas en la
  figura~\ref{fig:sos-rules-p}.  
\edfn

\bdfn\label{def:trtrantions}
$P\vtranp{s}{p}Q$ si y sólo si $s=a_1a_2\cdots a_n$ y $p=p_1\cdot p_2\cdot \cdots p_{n}$ y
\begin{displaymath}
P=P_0\tranp{a_1}{p_1}P_1\tranp{a_2}{p_2}P_2\cdots P_{n-1}\tranp{a_n}{p_n} P_n=Q
\end{displaymath}

$(pr,p)\in\prodp(P)$ si y sólo si $p>0$ y $p=\sum[q\ |\ P\vtranp{s\checkmark}{q} Q \y \product{s}=pr ]$
\edfn

Las definiciones \ref{def:trantions} y \ref{def:trtrantions}
muestran como es modelada la información probabilística
dentro de \fodaPAp. Esta, es relativa al procesamiento
de las características que pertenezcan a cualquier término
válido y a trazas de características.

%\todo{Hay que explicar despacio las cosas}
%\todo{explicar los lemas y poner más texto}

\blem\label{lem:prob}
  Dados los términos $P,\ Q\in\fodaPA$, 
  \begin{itemize}
  \item Dada la característica  $\feature{A}\in\calF\cup\{\checkmark\}$, si
    $P\tranp{\feature{A}}{p}Q$ entonces $p\in(0,1]$.
  \item Dada la característica $s\in\calF^*$, si $P\vtranp{s}{p}Q$ entonces $p\in(0,1]$.
  \end{itemize}
\elem

Es importante una vez descrito como será modelada la información probabilística,
definir el dominio matemático de estos nuevos datos.
El lema \ref{lem:prob} muestra que la información relativa a la probabilidad
de cualquier característica aislada o parte de cualquier traza probabilística
estará comprendida por números reales  entre $(0,1]$
con lo que se plantea el siguiente lema.

\blem\label{lem:sum:prob}
  Para cualquier término $P\in\fodaPA$ se cumple
  \begin{displaymath}
    \sum \{p\, |\ \exists \feature{A}\in\calF,\ Q\in\fodaPA:\ P\tranp{\feature{A}}{p}Q \}\in[0,1]
  \end{displaymath}
\elem

Así el lema \ref{lem:sum:prob} acota a que la sumatoria
de todas las probabilidades del procesamiento de
las características deba ser representado como un número
real entre $[0,1]$.

\blem\label{lem:check}
  Dados los términos $P,Q\in\fodaPA$ y la probabilidad $p\in(0,1]$, $P\tranp{\checkmark}{p}{Q}$ si y sólo si $Q=\nil$.
\elem

Debe ser descrito el comportamiento del símbolo terminal
al generar un producto válido de la extensión probabilística,
el lema \ref{lem:check} muestra que solo es posible procesar
la característica $\checkmark$ cuando el término restante es
$\nil$ lo que significa que ha sido generado un producto válido.


Una vez definido como es modelada la información probabilística
en la semántica operacional de \fodaPAp\ debe describirse
el proceso del cálculo de las probabilidades, esto es logrado
mediante la definición de las funciones que permitirán calcular
la probabilidad de una traza probabilística dentro de un término
que pertenezca al álgebra.

Ya definida la noción del almacenamiento
de la información
relacionada a las probabilidades dentro
de los términos del álgebra, se procede
a definir el concepto de traza probabilística.

 \bdfn\label{dfn:pr:tr}
   Dado el término $P\in\fodaPA$ y la traza
   probabilística
   $\sigma =\langle (a_{1},p_{1}),\ldots,(a_{n},p_{n})\rangle \in \ptraces{N_P}$,
   decimos que $\sigma = \langle (a_{1},p_{1}),\ldots,(a_n,p_n)\rangle$
   es una traza satisfactoria de $P$ si  $a_n=\checkmark$ y en caso
   contrario se dice que es una traza no satisfactoria.
   \begin{itemize}
   	\item Se denota el conjunto de trazas satisfactorias de $P$ como $\straces{P}$. 
   	\item Se denota el conjunto de trazas no satisfactorias de $P$ como $\untraces{P}$.
    \item Se denota la unión de ambos conjuntos de trazas satisfactorias
    y no satisfactoiras como $\ctraces{P}$.
   \end{itemize}
 \edfn

La definición \ref{dfn:pr:tr} define  el
proceso de construcción de tanto las trazas probabilísticas
como el resultado de la función $\straces{P}$ y en la 
definición \ref{dfn:pr:tr:cn} la agrupación de todas las
características para cada traza probabilística sin
importar el orden de los mismos.

 \bdfn\label{dfn:pr:tr:cn}
 Dada la traza probabilística $\sigma$. El \emph{producto inducido por la traza},
 escrito $[\sigma]$, es el conjunto obtenido
 de los elementos del alfabeto de la traza sin tomar en cuenta su posición dentro de la
 traza: 
 $$
 [\sigma] = \{ a_k |\ \sigma=\langle (a_{1},p_{1}),\ldots ,(a_{n},p_{n})\rangle \land 1 \leq k \leq n \land\ a_k\in\calF\}
 $$
 \edfn
 
Podemos tomar como ejemplo la siguiente traza $\sigma=\langle(\feature{A},p_{1}),(\feature{B},p_{2}),(\feature{C},p_{3}),(\checkmark,p_{4})\rangle$, 
 se tiene que
  $[\sigma]=\{\feature{A}, \feature{B}, \feature{C}\}$.

De esta manera la definición \ref{dfn:pr:tr:pr}
muestra 
el significado de la función 
encargada de calcular todos
los productos válidos generados
a partir
de un término cualquiera.

 \bdfn\label{dfn:pr:tr:pr}
 Dado un término $P\in\fodaPA$, se definen \emph{los productos de $P$}, escrito
 $\products{P}$, como:
 $$\products{P}=\{[\sigma]\ | \ \sigma\in\straces{N_P}\}$$
 \edfn
 
 Una vez calculadas las trazas probabilísticas de 
 los productos válidos y el conjunto de todas estas,
 se deben tomar en cuenta aquellas trazas probabílisticas
 que no han generado productos válidos.
 
 
 \bdfn\label{dfn:pr:tr:waste}
 Dado el término $P\in\fodaPA$, se define el \emph{desecho de $P$} como
 la suma de las probabilidades de las trazas no satisfactorias de $P$
 $$\waste{P}= \hspace*{-5em}\sum_{\sigma\in\ctraces{N_P}\setminus\straces{N_P}}\hspace*{-5em}\completeprob{\sigma}$$
 \edfn

La definición \ref{dfn:pr:tr:waste} permite calcular la probabilidad de aquellas trazas
no satisfactorias para cualquier término $P$.

 \bdfn
   Dado el producto $[\sigma]\in\products{P}$, 
   definimos la probabilidad de este producto en $P$ descrita $\prob{[\sigma],P}$ como:
         $$
         \prob{[\sigma],P} = \frac{\sum_{ \sigma' \in \straces{N_P} \land  [\sigma'] =  [\sigma]} \completeprob{\sigma'}}{1-\waste{P}}$$
 \edfn


 Es importante destacar que $\checkmark\not\in\calF$ entonces $\checkmark\not\in [\sigma]$.
 La función $\prob{[\sigma],P}$ induce un espacio probabilístico en el conjunto
 de productos de $P$.
 \bprop\label{dfn:pr:sum}
   Dado el producto $P\in\fodaPA$, entonces
   $$1 = \sum_{[\sigma]\in\products{P}} \prob{[\sigma],P}$$
 \eprop

La definición \ref{dfn:pr:sum} muestra que la suma de las probabilidades del producto
$[\sigma]\in\products{P}$ debe ser igual a 1.




\begin{figure}[h]
\linefigure
\vspace*{0.5em}
\begin{minipage}{0.2\hsize}
\centering 
\texttt{$P_{1}$}
\end{minipage}
\begin{minipage}{0.4\hsize}
\centering
\texttt{$P_{2}$}
\end{minipage}
\vspace*{1em}
\begin{minipage}{0.4\hsize}
\centering
\texttt{$P_{3}$}
\end{minipage}

\begin{minipage}{0.2\hsize}
\centering 
%$\feature{A};\ofeature{B};\checkmark \leadsto$
% \begin{minipage}{0.35\hsize}
%         \centering
\scalebox{0.7}
{
        \begin{tikzpicture}[->,>=stealth',node distance=1.5cm]
                \node[syntax] (A)   {$\feature{A};\ofeature{B};_{p}(\feature{C};\ofeature{D};_{q}\checkmark)$};
                \node[syntax] (B) [below=1.5cm of A]   {$_{1}(\ofeature{B};_{p}(\feature{C};\ofeature{D};_{q}\checkmark))$};
                \node[syntax] (C) [below left=1.5cm and 0cm of B]   {$_{(1-p)}\nil$};
                \node[syntax] (D) [below=1.5cm of B]   {$_{p}(\feature{C};\ofeature{D};_{q}\checkmark)$};
                \node[syntax] (F) [below=1.5cm of D]   {$_{1}(\ofeature{D};_{q}\checkmark)$};
                \node[syntax] (G) [below left=1.5cm and 0cm of F]   {$_{(1-q)}\nil$};
                \node[syntax] (H) [below=1.5cm of F] {$_{q}\checkmark$};
                \node[syntax] (I) [below=1.5cm of H]   {$_{1}\nil$};       
                         
                \path (A) edge[->] node[right, opacity=0, text opacity=1]
                          {$\feature{A}$~$\nombreRegla{feat}$}
                      (B)
                      (B) edge[->] node[above left, opacity=0, text opacity=1]
                          {$\checkmark$~$\nombreRegla{ofeat2}$}
                      (C)
                      (B) edge[->] node[right, opacity=0, text opacity=1]
                          {$\feature{B}$~$\nombreRegla{ofeat1}$}
                      (D)
                      (D) edge[->] node[right, opacity=0, text opacity=1]
                          {$\feature{C}$~$\nombreRegla{feat}$}
                      (F)
                      (F) edge[->] node[above left, opacity=0, text opacity=1]
                          {$\checkmark$~$\nombreRegla{ofeat2}$}
                      (G)
                      (F) edge[->] node[right, opacity=0, text opacity=1]
                          {$\feature{D}$~$\nombreRegla{ofeat1}$}
                      (H)
                      (H) edge[->] node[right, opacity=0, text opacity=1]
                          {$\checkmark$~$\nombreRegla{tick}$}
                      (I);                      
        \end{tikzpicture}
}
\end{minipage}
\begin{minipage}{0.4\hsize}
\centering
%$\feature{A};\ofeature{B};\checkmark \leadsto$
% \begin{minipage}{0.35\hsize}
%         \centering
\scalebox{0.7}{
        \begin{tikzpicture}[->,>=stealth',node distance=1.5cm]
        \node[syntax] (A)   {$\feature{A};\ofeature{B};_{p}(\feature{C};(\feature{D};\checkmark\choice_{q}(\ofeature{E};_{r}\checkmark)))$};
        \node[syntax] (B) [below=1.5cm of A]   {$_{1}(\ofeature{B};_{p}(\feature{C};(\feature{D};\checkmark\choice_{q}(\ofeature{E};_{r}\checkmark)))$};
        \node[syntax] (C) [below right=1.5cm and 0cm of B]   {$_{(1-p)}\nil$};
        \node[syntax] (D) [below=1.5cm of B]   {$_{p}(\feature{C};(\feature{D};\checkmark\choice_{q}(\ofeature{E};_{r}\checkmark)))$};
        \node[syntax] (F) [below=1.5cm of D]   {$_{1}(\feature{D};\checkmark\choice_{q}(\ofeature{E};_{r}\checkmark))$};
        \node[syntax] (G) [below left=1.5cm and 0cm of F]   {$_{(1.q)}\checkmark$};
        \node[syntax] (H) [below=1.5cm of F] {$_{(1-p).q}\checkmark$};
        \node[syntax] (I) [below right=1.5cm and 0cm of F]   {$_{(1-q).(1-r)}\nil$};       
        \node[syntax] (J) [below=1.5cm of G]   {$_{1}\nil$};     
        \node[syntax] (K) [below=1.5cm of H]   {$_{1}\nil$};        

              
        \path (A) edge[->] node[right, opacity=0, text opacity=1]
        {$\feature{A}$~$\nombreRegla{feat}$}
        (B)
        (B) edge[->] node[above right, opacity=0, text opacity=1]
        {$\checkmark$~$\nombreRegla{ofeat2}$}
        (C)
        (B) edge[->] node[left, opacity=0, text opacity=1]
        {$\feature{B}$~$\nombreRegla{ofeat1}$}
        (D)
        (D) edge[->] node[right, opacity=0, text opacity=1]
        {$\feature{C}$~$\nombreRegla{feat}$}
        (F)
        (F) edge[->] node[above left, opacity=0, text opacity=1]
        {$\feature{D}$~$\nombreRegla{cho1}$}
        (G)
        (F) edge[->] node[right, opacity=0, text opacity=1]
        {$\feature{E}$~$\nombreRegla{cho2}$}
        (H)
        (F) edge[->] node[above right, opacity=0, text opacity=1]
        {$\checkmark$~$\nombreRegla{ofeat2}$$\nombreRegla{cho2}$}
        (I)
        (G) edge[->] node[right, opacity=0, text opacity=1]
        {$\checkmark$~$\nombreRegla{tick}$}
        (J)
        (H) edge[->] node[right, opacity=0, text opacity=1]
        {$\checkmark$~$\nombreRegla{tick}$}
        (K)

        ;                      
        \end{tikzpicture}
}
\end{minipage}
\begin{minipage}{0.4\hsize}
	\centering
	%$\feature{A};\ofeature{B};\checkmark \leadsto$
	% \begin{minipage}{0.35\hsize}
	%         \centering
	\scalebox{0.7}{
		\begin{tikzpicture}[->,>=stealth',node distance=1.5cm]
        \node[syntax] (A)   {$\ofeature{A};_{p}\feature{B};(\ofeature{C};_{q}\checkmark\paral\feature{D};\checkmark)$};
		\node[syntax] (B) [below left=1.5cm and 2cm of A]   {$_{(1-p)}\nil$};
		\node[syntax] (C) [below=1.5cm of A]   {$_{p}\feature{B};(\ofeature{C};_{q}\checkmark\paral\feature{D};\checkmark)$};

		\node[syntax] (D) [below=1.5cm of C]   {$_{1}(\ofeature{C};_{q}\checkmark\paral\feature{D};\checkmark)$};

		\node[syntax] (F) [below left=1.5cm and 1cm of D]   {$_{\frac{q}{2}}(\checkmark\paral\feature{D};\checkmark)$};
		
		\node[syntax] (G) [below=1.5cm of D]
		{$_{\frac{q}{2}}(\ofeature{C};\checkmark\paral\checkmark)$};
		
		\node[syntax] (H) [below=1.5cm of F]   {$_{\frac{\frac{q}{2}.p}{2}}(\checkmark\paral\checkmark)$};	   
		
		\node[syntax] (I) [below=1.5cm of G]   {$_{\frac{\frac{q}{2}.p}{2}}(\checkmark\paral\checkmark)$};
		
		\node[syntax] (J) [below right=1.5cm and 1cm of G]
		{$_{(1-(\frac{q}{2}))}(\nil)$};

		\node[syntax] (K) [below=1.5cm of H]   {$_{p.q}(\nil)$};		   
		\node[syntax] (L) [below=1.5cm of I]   {$_{p.q}(\nil)$};
		
						
		\path (A) edge[->] node[above left, opacity=0, text opacity=1]
		{$\checkmark$~$\nombreRegla{ofeat2}$}
		(B)
		(A) edge[->] node[right, opacity=0, text opacity=1]
		{$\feature{A}$~$\nombreRegla{ofeat1}$}
		(C)
		(C) edge[->] node[right, opacity=0, text opacity=1]
		{$\feature{B}$~$\nombreRegla{feat1}$}
		(D)
		(D) edge[->] node[above left, opacity=0, text opacity=1]
		{$\feature{C}$~$\nombreRegla{con1}$}
		(F)
		(D) edge[->] node[right, opacity=0, text opacity=1]
		{$\feature{D}$~$\nombreRegla{con2}$}
		(G)
		(F) edge[->] node[left, opacity=0, text opacity=1]
		{$\feature{D}$~$\nombreRegla{con5}$}
		(H)
		(G) edge[->] node[left, opacity=0, text opacity=1]
		{$\feature{C}$~$\nombreRegla{con4}$}
		(I)
		(G) edge[->] node[right, opacity=0, text opacity=1]
		{$\checkmark$~$\nombreRegla{ofeat2}$}
		(J)
		(H) edge[->] node[right, opacity=0, text opacity=1]
		{$\checkmark$~$\nombreRegla{con3}$}
		(K)
		(I) edge[->] node[right, opacity=0, text opacity=1]
		{$\checkmark$~$\nombreRegla{con3}$}
		(L)		
		;                      
		\end{tikzpicture}
	}
\end{minipage}
\linefigure

\caption{Ejemplos de la ejecución de la reglas de la semántica operacional del modelo probabilístico.\label{example:op}}
\end{figure}

Los ejemplos descritos en la figura~\ref{example:op} muestran el
cálculo de la información probabilística al procesar los términos $P_1$, $P_2$ y $P_3$.


Consideremos el término $P_1$=
$\feature{A};\ofeature{B};_{p}(\feature{C};\ofeature{D};_{q}\checkmark)$
cuyo conjunto $\straces{N_{P_1}}$ es:

$$
\begin{array}{ll}
\straces{N_{P_1}}=&\{ \\
                 &  \langle(\feature{A},1),(\checkmark,(1-p))\rangle,\\ 
                 &\langle(\feature{A},1),(\feature{B},p),(\feature{C},1),(\checkmark,1-q)\rangle\\ 
                 &\langle(\feature{A},1),(\feature{B},p),(\feature{C},1),(\feature{D},q),(\checkmark,1)\rangle\\
                 & \}
\end{array}     
$$

Donde $\products{P_1}=\{[\feature{A}],[\feature{A}\feature{B}\feature{C}],[\feature{A}\feature{B}\feature{C}\feature{D}]\}$

De igual manera el término $P_2$=
$\feature{A};\ofeature{B};_{p}(\feature{C};(\feature{D};\checkmark\choice_{q}(\ofeature{E};_{r}\checkmark)))$
cuyo conjunto $\straces{N_{P_2}}$ es:

$$
\begin{array}{ll}
\straces{N_{P_2}}=&\{ \\
&  \langle(\feature{A},1),(\checkmark,(1-p))\rangle,\\ 
&\langle(\feature{A},1),(\feature{B},p),(\feature{C},1),(\feature{D},1.q),(\checkmark,1)\rangle\\ 
&\langle(\feature{A},1),(\feature{B},p),(\feature{C},1),(\feature{E},((1-p).q)),(\checkmark,1)\rangle\\
&\langle(\feature{A},1),(\feature{B},p),(\feature{C},1),(\checkmark,(1-q).(1-r))\rangle\\
& \}
\end{array}     
$$

Donde $\products{P_2}=\{[\feature{A}],[\feature{A}\feature{B}\feature{C}\feature{D}],[\feature{A}\feature{B}\feature{C}\feature{E}],[\feature{A}\feature{B}\feature{C}]\}$




También $P_3$=
$\ofeature{A};_{p}\feature{B};(\ofeature{C};_{q}\checkmark\paral\feature{D};\checkmark)$
cuyo conjunto $\straces{N_{P_3}}$ es:

$$
\begin{array}{ll}
\straces{N_{P_3}}=&\{ \\
&  \langle(\checkmark,(1-p))\rangle,\\ 
&\langle(\feature{A},p),(\feature{B},1),(\feature{C},\frac{q}{2}),(\feature{D},\frac{\frac{q}{2}.p}{2}),(\checkmark,p.q)\rangle\\ 
&\langle(\feature{A},p),(\feature{B},1),(\feature{D},\frac{q}{2}),(\feature{C},\frac{\frac{q}{2}.p}{2}),(\checkmark,p.q)\rangle\\ 
&\langle(\feature{A},p),(\feature{B},1),(\feature{D},\frac{q}{2}),(\checkmark,1-(\frac{q}{2}))\rangle\\ 
& \}
\end{array}     
$$

Donde $\products{P_3}=\{[\feature{A}\feature{B}\feature{C}\feature{D}],[\feature{A}\feature{B}\feature{D}]\}$

De esta manera es posible dentro de la semántica operacional, calcular
las probabilidades para cada traza en cada elemento.